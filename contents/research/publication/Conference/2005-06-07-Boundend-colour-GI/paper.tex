%\documentclass[times,10pt,twocolumn]{ieee}


\documentclass[11pt]{article}

%\usepackage{floatsty}
%\usepackage{fullpage}

%\usepackage{showlabels}

\usepackage{amsthm}
\usepackage{amssymb}
\usepackage{algorithm2e}

\newcommand{\Section}[1]{\section{#1}}
\newcommand{\SubSection}[1]{\subsection{#1}}

\newtheorem{theorem}{Theorem}[section]

\newtheorem{lemma}[theorem]{Lemma}
\newtheorem{conjecture}[theorem]{Conjecture}
\newtheorem{fact}[theorem]{Fact}
\newtheorem{corollary}[theorem]{Corollary}
\newtheorem*{claim}{Claim}
\newtheorem*{remark}{Remark}
\newtheorem{proposition}[theorem]{Proposition}
\newtheorem{observation}{Observation}[theorem]
\newtheorem{definition}[theorem]{Definition}
\newtheorem{problem}[theorem]{Problem}
%\newtheorem{remark}[theorem]{Remark}

%\newcommand {\proof}{{\bf Proof:\newline \indent}}
\newcommand{\bproof}{\noindent{\it Proof}}
\newcommand{\cproof}{\noindent{\it Proof of Claim}}
\newcommand{\eproof}{\hspace*{\fill}$\rule{2mm}{2mm}$~~~~~\bigskip}
\renewenvironment{proof}{\bproof. }{\eproof}
\newenvironment{psketch}{\bproof~{\it Sketch}. }{\eproof}
\newenvironment{claimproof}{\cproof. }{\hspace*{\fill}\vspace{5mm}}

\newcommand{\FuncProblem}[3][]
{
        \begin{problem}#1 
        \end{problem}
        \begin{tabbing}
        \ \ \ \ \={\bf Input:\ \=}#2\\
        \ \ \ \ \={\bf Output:\ \=}#3 \\
        \end{tabbing}
}
\newcommand{\DecProblem}[3][]
{
        \begin{problem}#1 
        \end{problem}
        \begin{tabbing}
        \ \ \ \ \={\bf Input:\ \=}#2\\
        \ \ \ \ \={\bf Question:\ \=}#3\\
        \end{tabbing}
}


%\input{complexityclass}
%%
%%      End of Header file
%%


\newcommand{\UP}{\mbox{\rm UP}}
\newcommand{\PP}{\mbox{\rm PP}}
\newcommand{\GapP}{\mbox{\rm GapP}}
\newcommand{\NP}{\mbox{\rm NP}}
\newcommand{\FP}{\mbox{\rm FP}}
\newcommand{\coNP}{\mbox{\rm co-NP}}
\newcommand{\ParityP}{\mbox{$\oplus$\rm P}}
\newcommand{\ModkP}{\mbox{\rm Mod$_k$P}}
\newcommand{\ModKL}[1]{\ensuremath{\mathrm{Mod}_{#1}\mathrm{L}}}
\newcommand{\numP}{\mbox{\rm \#P}}
\newcommand{\LWPP}{\mbox{\rm LWPP}}
\newcommand{\UPSV}{\mbox{\rm UPSV}}
\newcommand{\SPP}{\mbox{\rm SPP}}
\newcommand{\AM}{\mbox{\rm AM}}
\newcommand{\coAM}{\mbox{\rm coAM}}
\newcommand{\ModP}{\mbox{\rm ModP}}
\newcommand{\DSPACE}{\mbox{\rm DSPACE}}
\newcommand{\BPL}{\mbox{\rm BPL}}
\newcommand{\CeqP}{\mbox{\rm C$_=$P}} 
\newcommand{\gap}{\mbox{\it gap}}
\newcommand{\BQP}{\mbox{\rm BQP}}
\newcommand{\AWPP}{\mbox{\rm AWPP}}

\newcommand{\Mod}[1]{\ensuremath{\mathrm{Mod}_{#1}\mathrm{L}}}

\newcommand{\Lo}{\ensuremath{\mathrm{L}}}
\newcommand{\NL}{\ensuremath{\mathrm{NL}}}
\newcommand{\SL}{\mbox{\rm SL}}
\newcommand{\GapL}{\mbox{\rm GapL}}
\newcommand{\ModkL}{\mbox{\rm Mod$_k$L}}
\newcommand{\ModpL}{\mbox{\rm Mod$_p$L}}
\newcommand{\ModpaL}{\mbox{\rm Mod$_{p^\alpha}$L}}
\newcommand{\FL}{\mbox{\rm FL}}
\newcommand{\NC}{\mbox{\rm NC}}
\newcommand{\TC}{\mbox{\rm TC}}
\newcommand{\sharpL}{\mbox{$\#$\rm L}}

\newcommand{\NCone}{\ensuremath{\NC^1}}
\newcommand{\NCtwo}{\ensuremath{\NC^2}}
\newcommand{\RNC}{\ensuremath{\rm RNC}}
\newcommand{\TCone}{\ensuremath{\TC^1}}

\newcommand{\CeqL}{\mbox{\rm C$_=$L}}
\newcommand{\PL}{\mbox{\rm PL}}
\newcommand{\ACzeroPL}{\mbox{\ensuremath{\textrm{AC}^0\textrm{PL}}}}
\newcommand{\ACzeroCeqL}{\mbox{\ensuremath{\textrm{AC}^0\textrm{C}_=\textrm{L}}}}
\newcommand{\ACzero}{\mbox{\ensuremath{\textrm{AC}^0}}}

\newcommand{\ModL}{\mbox{\rm ModL}}
\newcommand{\LModL}{\mbox{\ensuremath{\L^{\ModL}}}}
\newcommand{\FLModL}{\mbox{\ensuremath{\FL^{\ModL}}}}
\newcommand{\FLModLpoly}{\mbox{\ensuremath{\FL^{\ModL}{\rm /poly}}}}
\newcommand{\RLModL}{\mbox{\ensuremath{\RL^{\ModL}}}}
%\newcommand{\ZPLModL}{\mbox{\ensuremath{\ZPL^{\ModL}}}}
%\newcommand{\ZPLModL}{\mbox{\ensuremath{\ZPL}}}
\newcommand{\poly}{\mbox{\rm poly}}
\newcommand{\End}{\mbox{\rm End}}


\newcommand{\iso}{\ensuremath{\cong} }
\newcommand{\niso}{\ensuremath{\ncong}}
\newcommand{\ord}[1]{\ensuremath{{\it ord}_p(#1)}}
\newcommand{\diag}{\ensuremath{\it diag}}
\newcommand{\cont}{\ensuremath{\it cont}}

\newcommand{\BCGI}{\mbox{\rm BCGI}}

\newcommand{\Sift}[1]{\ensuremath{\mathrm{Sift}({#1})}}

\newcommand{\red}{\mbox{\rm RED}}
\newcommand{\blue}{\mbox{\rm BLUE}}

\newcommand{\commu}[1]
{\ensuremath{\left\lbrack #1 \right\rbrack}}

%\newcommand{\Sym}[1]{\ensuremath{\mathrm{Sym}(#1)}}
\newcommand{\NCL}[2]{\ensuremath{\mathrm{NCL}_{#1}({#2})}}

\newcommand{\Q}{\mathbb{Q}}
\newcommand{\Z}{\mathbb{Z}}
\newcommand{\Zpe}{\mathbb{Z}_{p^e}}
\newcommand{\M}{\mathcal{M}}
\newcommand{\F}{\ensuremath{\mathbb{F}}}

\newcommand{\x}{\mbox{\bf x}}
\newcommand{\y}{\mbox{\bf y}}
\newcommand{\z}{\mbox{\bf z}}
\renewcommand{\b}{\mbox{\bf b}}
\renewcommand{\v}{\mbox{\bf v}}

\newcommand{\sym}[1]{{\rm Sym(}#1{\bf )}}

\renewcommand{\angle}[1]{\langle #1\rangle}

\newcommand{\acc}{\mbox{\it acc}}
\newcommand{\rej}{\mbox{\it rej}}
%\newcommand{\gap}{\mbox{\it gap}}

\newcommand{\p}[1]{\ensuremath{p_{#1}^{\alpha_#1}}}

\newcommand{\q}{\ensuremath{q=\prod_{j=1}^n p_j^{\alpha_j}}}

\newcommand{\LCON}{\mbox{\small\rm LCON}}
\newcommand{\LCONX}{\mbox{\small\rm LCONX}}
\newcommand{\LCONL}{\mbox{\small\rm LCONNULL}}
\newcommand{\AGM}{\mbox{\small\rm AGM}}
\newcommand{\AORDER}{\mbox{\small\rm AORDER}}
\newcommand{\AISO}{\mbox{\small\rm AISO}}
\newcommand{\AGMX}{\mbox{\small\rm AGMX}}
\newcommand{\AINTER}{\mbox{\small\rm AINTER}}
\newcommand{\AGP}{\mbox{\small\rm AGP}}
\newcommand{\ORBIT}{\mbox{\small\rm ORBIT}}
\newcommand{\STAGMX}{\mbox{\small\rm STAGMX}}
\newcommand{\APISO}{\mbox{\small\rm APISO}}
\newcommand{\APGM}{\mbox{\small\rm APGM}}
\newcommand{\APORDER}{\mbox{\small\rm APORDER}}

\newcommand{\Ker}{\mbox{\rm Ker}}
\newcommand{\Prob}{\mbox{\rm Prob}}
\newcommand{\mod}{\mbox{\textrm mod}}


\newcommand{\psb}[2]{\mbox{\ensuremath{\rm{#1~}{POINTSET}_{#2}}}}
\newcommand{\prpsb}[1]{\ensuremath{\textrm{primitive-POINTSET}_{#1}}}

\newcommand{\Res}[2]{\ensuremath{\mathrm{Res}_{#1}\left({#2}\right)}}

\newcommand{\Sym}[1]{\ensuremath{\mathrm{Sym}\left({#1}\right)}}
\newcommand{\Soc}[1]{\ensuremath{\mathrm{Soc}\left(#1\right)}}
\newcommand{\Diag}[1]{\ensuremath{\mathrm{Diag}\left({#1}\right)}}
\newcommand{\pr}[2]{\ensuremath{pr_{#1}\left({#2}\right)}}

\title{Bounded Color Multiplicity Graph Isomorphism\\ is in 
the $\#\Lo$ Hierarchy\thanks{Part of this work was supported 
by a DST-DAAD project supporting exchange visits.}}

%\title{Bounded Color Multiplicity Graph Isomorphism is in
%$\Lo^{\GapL}$}

\author{
  V. Arvind, Piyush P Kurur, and T.C. Vijayaraghavan \\
  Institute of Mathematical Sciences, C.I.T Campus\\ 
  Chennai 600113, India\\
  email: {\tt\{arvind,ppk,tcvijay\}@imsc.res.in}\\
}

\date{}

\begin{document}

\maketitle

\begin{abstract}
In this paper we study the complexity of \emph{Bounded Color
Multiplicity Graph Isomorphism} $\BCGI_b$: the input is a pair of
vertex-colored graphs such that the number of vertices of a given
color in an input graph is bounded by $b$. We show that $\BCGI_b$ is
in the $\#\Lo$ hierarchy (more precisely, the $\Mod{k}$ hierarchy for
some constant $k$ depending on $b$). Combined with the fact that
Bounded Color Multiplicity Graph Isomorphism is logspace many-one hard
for every set in the $\Mod{k}$ hierarchy for any constant $k$, we get
a tight classification of the problem using logspace-bounded counting
classes.
\end{abstract}

\section{Introduction}

In the last decade or so, logspace bounded classes, particularly
logspace counting classes, have played an important role in
classifying natural problems in $\NCtwo$. A well-known example is the
problem of computing the determinant. We know from Toda and Vinay's
work~\cite{toda91counting,vinay91counting} that the complexity class
GapL \emph{exactly} captures the complexity of computing integer
determinants. Likewise, computing determinants over finite fields of
characteristic $p$ is captured by the class $\ModpL$ \cite{DH}. More
recently, the results of~\cite{allender99complexity} classify
important linear algebraic problems like finding the rank and checking
feasibility of linear equations. Also, the complexity of perfect
matching is now quite well characterized in~\cite{allender99isolation}
using logspace counting classes and the isolation lemma. For other
results in this area, including connections to circuit complexity
questions, we refer the reader to the recent survey article by
Allender~\cite{quaderni}.

\subsection{Graph Isomorphism and Logspace Counting Classes}

In this paper we study the following restricted version of Graph
Isomorphism problem: The input is a pair of vertex-colored graphs
$(G_1,G_2)$ with the additional property that the number of vertices
of a given color in the graphs is less than a prescribed constant
bound $b$. And, the restricted Graph Isomorphism problem is to check
if there is a color-preserving isomorphism between $G_1$ and $G_2$.

This problem is known as the \emph{Bounded Color Multiplicity Graph
Isomorphism} problem (which we denote by $\BCGI_b$). It is a
well-studied restriction of Graph Isomorphism: This was the first
version of Graph Isomorphism studied using group-theoretic methods
when Babai gave a randomized polynomial time algorithm for $\BCGI_b$
for each constant $b$~\cite{babai}. This result was improved to a
deterministic polynomial time algorithm in~\cite{FHL}. Subsequently,
Luks in \cite{luks} gave a remarkable NC algorithm for the $\BCGI_b$
problem. Building on earlier work by McKenzie and Cook \cite{MC}) and
Luks and McKenzie \cite{LM}, in \cite{luks} Luks introduced
group-theoretic techniques to handle the nonabelian parts of
permutation groups in the class NC. The earlier papers \cite{MC,LM}
developed techniques to show that several permutation group problems
for abelian and solvable groups are in the class NC. These methods are
linear-algebraic and do not extend to general permutation groups.

Luks' NC algorithm in \cite{luks} is actually for a more general
problem (of which $\BCGI_b$ is a special case). In \cite{luks} the
actual parallel time taken by the algorithm is not specified. But we
can notice that the NC algorithm runs in $O(\log n)$ stages and in
each stage it uses Luby's $\NCtwo$ algorithm for the maximal
independent set problem. Thus the running time of the parallel
algorithm appears to be at least $\log^3 n$.

In a different line of research, Tor\'an in \cite{tor} developed a
nice graph gadget that enables the simulation of addition modulo $k$
using automorphisms of the graph gadget. As a consequence, in
\cite{tor} several hardness results for Graph Isomorphism are
shown. In particular, it is shown that $\BCGI_b$ is $\ACzero$-many one
hard for the logspace counting class $\Mod{k}$ for each constant
$k$. The construction in \cite{tor} requires $b$ to be $k^2$.

In a related paper~\cite{jkmt}, the BCGI problem is examined for color
classes of sizes $2$ and $3$, and show that in these cases it is in SL
(which puts it in L as $\SL=\Lo$ \cite{rein}).

\subsection{The results of this paper}

The question that naturally arises is whether the gap between
the $\NC$ upper bound result of \cite{luks} and the lower bound result
of \cite{tor} for $\BCGI_b$ can be closed. In this paper, we examine
Luks' NC algorithm carefully using logspace complexity classes. As our
main result we show that $\BCGI_b$ is in $\Mod{k}$ hierarchy, where
the constant $k$ and the level of the hierarchy in which BCGI sits
depends on $b$.

It turns out that the we can decompose the problem into a constant
number of stages. Each stage corresponds to either handling an abelian
simple group or a nonabelian simple group. We are able to show that
the stages involving abelian groups can be solved in the complexity
class $\Mod{k}$ for a suitable constant $k$. Surprisingly for handling
nonabelian groups, although the group theory is more involved, the
actual computation can be done in deterministic logspace $\Lo$. This
is because in the nonabelian stages we can do away with finding
maximal independent sets and show that the computations involved are
actually reducible to \emph{undirected graph connectivity} (which is
recently shown to be in $\Lo$ \cite{rein}).

Since the $\Mod{k}$ hierarchy is contained in the $\#\Lo$ hierarchy
which, in turn, is contained in $\NCtwo$ (even in $\TCone$)
\cite{allender99complexity}, our result puts $\BCGI_b$ in $\NCtwo$
(indeed in $\TCone$).

Below we depict a chain known inclusions of the relevant complexity
classes inside $\NCtwo$.

\begin{eqnarray*}
\BCGI_b\in\Mod{k}^{\Mod{k}^{~^{\cdot^{\cdot^{\cdot}}}}}\subseteq  
\#\Lo^{\#\Lo^{~^{\cdot^{\cdot^{\cdot}}}}}\subseteq  
\TCone \subseteq \NCtwo.
\end{eqnarray*}

While the ideas and the underlying group-theoretic machinery in our
result is based on Luks \cite{luks} and \cite{LM}, to prove the
$\Mod{k}$ hierarchy upper bound we had to make several crucial
changes in the NC algorithm of \cite{luks}.

%\subsection{$\BCGI_b$ is classified by the $\Mod{k}$ hierarchy}

\begin{remark}
How well does our $\Mod{k}$ hierarchy upper bound classify the problem
$\BCGI_b$? 

We can apply Tor\'an's result \cite[Lemma 4.7]{tor} to show that for
every $k$ if $A$ is a set in the $j$th level of the $\Mod{k}$
hierarchy, then $A$ is logspace many-one reducible to $\BCGI_b$, where
$b$ depends only on $k$ and $j$. Thus, $\BCGI_b$ is intimately related
to the $\Mod{k}$ hierarchy. The question whether the $\Mod{k}$
hierarchy collapses for composite $k$ is open (for prime $k$ it
is known to collapse to $\Mod{k}$). 
\end{remark}

\section{Preliminaries and Notation}\label{defs}

We assume some familiarity with basic complexity theory and basic
notions in group theory and linear algebra. For details on the
complexity classes discussed here we refer the reader to
\cite{allender99complexity} and for group-theoretic definitions we
refer the reader to \cite{hallbook,wielandt64finite}.

\subsection{Complexity Classes}

We write $\Lo$ and $\FL$ for the class of languages (respectively,
functions) computable by deterministic logspace-bounded Turing
machines. The class of languages accepted by nondeterministic logspace
machines is denoted by NL, and the class of languages accepted by
symmetric logspace machines is denoted by SL (equal to $\Lo$ in
\cite{rein}). The class NL is known to be closed under complement.

We now recall the definitions for the complexity classes $\GapL$ and
$\ModkL$ with $k\geq2$. For a nondeterministic Turing machine $M$ we
denote by $\acc_M(x)$ and $\rej_M(x)$ the number of accepting and
rejecting computation paths of $M$ on input $x$, respectively. We
define $\gap_M(x)$ as the difference $\acc_M(x)-\rej_M(x)$.

\begin{definition}\label{GapL}
 The class $\GapL$ is defined as the class of functions
  $f:\Sigma^*\longrightarrow \Z$ such that for some nondeterministic
  logspace-bounded Turing machine $M$, $f(x)=\gap_M(x)$ for every
  $x\in\Sigma^*$.
\end{definition}

$\GapL$ can also be defined equivalently as the closure of $\sharpL$
under subtraction, where $\sharpL$ denote the set of all functions
$\acc_M(x)$ for nondeterministic logspace-bounded Turing machines $M$.

\begin{definition}\label{ModkL}
  Let $k\geq 2$ be an integer. $\ModkL$ is the class of sets L such
  that there is an $f\in\sharpL$ with $x\in L\Leftrightarrow
  f(x)\not\equiv 0(\mbox{mod }k)$.
\end{definition}

We now define the reductions that are discussed in this paper: for
languages $L$ and $L'$, we say that $L$ is \emph{logspace many-one
reducible} to $L'$ if there is an $f\in\FL$ such that $x\in L$ if and
only if $f(x)\in L'$. We say that $L$ is \emph{logspace Turing
reducible} to $L'$ if there is a deterministic logspace oracle Turing
machine $M$ such that $M$ with oracle $L'$ accepts $L$.

Nondeterministic logspace oracle Turing machines are defined using the
``Ruzzo-Simon-Tompa'''' oracle access~\cite{RST} (also see
\cite{allender96relationships,allender99complexity}) which restricts
the machine to write the oracle queries deterministically. We define
the relativized classes $\#\Lo^A$ and $\Mod{k}^A$ in this model. We
can now inductively define the $\Mod{k}$ hierarchy: A language $L$ is
in the $j$th level of the $\Mod{k}$ hierarchy if there is an oracle
set $A$ in the $j-1$th level such that $L\in\Mod{k}^A$.

For languages $L$ and $L'$, we say that $L$ is $\NCone$-Turing
reducible to $L'$ if there is a logspace uniform circuit family of
polynomial size and logarithmic depth with bounded fanin gates that
computes $L$ with oracle gates for $L'$. AC$^0$-Turing reductions are
similarly defined using logspace uniform AC$^0$ circuit families. It
is known that the $\#\Lo$ hierarchy equals $\ACzero(\#\Lo)$
\cite{allender96relationships}.

\subsection{Permutation Groups}

We recall useful definitions and basic facts about permutation
groups. For details we refer the reader to the book by
Wielandt~\cite{wielandt64finite}.

A permutation group $G$ is a subgroup of $\Sym{\Omega}$, where
$\Sym{\Omega}$ is the group of all permutations on $\Omega$, and
$\Omega$ is an $n$-element set. We write $H\leq G$ when $H$ is a
subgroup of $G$. The subgroup $H$ of $G$ is \emph{normal}, denoted
$H\lhd G$, if $gHg^{-1}=H$ for all $g\in G$. 

The image of a point $\alpha\in\Omega$ under $g\in G$ is denoted
$\alpha^g$, and we apply permutations from left to right so that
$\alpha^{gh}=(\alpha^g)^h$.  The set $\alpha^G=\{\alpha^g\mid g\in
G\}$ is the \emph{$G$-orbit} of $\alpha$, and $G$ is \emph{transitive}
on $\Omega$ if $\alpha^G=\Omega$. The \emph{group generated} by a set
of permutations $S\subseteq\Sym{\Omega}$ is the smallest subgroup of
$\Sym{\Omega}$ containing $S$, denoted $\angle{S}$. For a group $G$
and $S\subseteq G$, the \emph{normal closure} of $S$ in $G$ is
the smallest normal subgroup of $G$ containing $S$ and is denoted
by $\NCL{G}{S}$.

A group $G$ is \emph{cyclic} if $G=\angle{g}$ for some $g$, $G$ is
\emph{abelian} if $gh=hg$ for all $g, h\in G$.
%, and $G$ is a $p$-group
%if $|G|$ is a power of the prime $p$.

Let $G\leq\Sym{\Omega}$ be transitive on $\Omega$. A subset $\Delta
\subseteq \Omega$ is called a \emph{$G$-block} if for every $g \in G$
either $\Delta^g = \Delta$ or $\Delta^g \cap \Delta = \emptyset$. For
any transitive group $G$, the set $\Omega$ and the singleton sets $\{
\omega \}$, $\omega \in \Omega$ are called \emph{trivial} blocks. A
transitive group $G$ is called \emph{primitive} if it does not have
any nontrivial blocks otherwise it is called \emph{imprimitive}.

Consider a permutation group $G$ that is transitive on $\Omega$. If
$\Delta$ is a $G$-block, then for every $g \in G$ $\Delta^g$ is also a
$G$-block. The collection of blocks $\{ \Delta^g : g \in G\}$ forms a
partition of $\Omega$ and is called the \emph{$\Delta$ block system}.
Notice that the permutation group $G$ acts transitively on the
$\Delta$ block system (since every element of $G$ in its natural
action maps blocks to blocks). We call $\Delta\subset\Omega$ a
\emph{maximal block} if there is no other block $\Sigma$ such that
$\Delta\subset\Sigma\subset\Omega$. In this case, we call $\{ \Delta^g
: g \in G\}$ a \emph{maximal block system}.

If $G$'s action is imprimitive on a maximal block system, then we can
again find a maximal block system here and continue further, resulting
in a hierarchy of block systems. We formalize this process below.

Let $\Delta$ and $\Sigma$ be two $G$-blocks in $\Omega$, such that
$\Delta \subseteq \Sigma$. Notice that the collection of blocks $\{
\Delta^g : g \in G \textrm{ and } \Delta^g \subseteq \Sigma \}$ forms
a partition of $\Sigma$ which we call the \emph{$\Delta$ block system}
of $\Sigma$.

We now define a structure tree of the transitive group $G$.  Structure
trees play an important role in algorithms for permutation groups
\cite{Lu93}. The \emph{structure tree} of $G$ is a rooted tree nodes
of which are labeled by $G$-blocks satisfying the following
conditions:

\begin{enumerate} 
\item The root is labeled $\Omega$.  
\item The leaves are labeled with singleton sets $\{ \alpha\}$, 
$\alpha \in \Omega$.
\item For each internal node labeled by $\Sigma$, the labels of its
children constitute a $\Delta$ block system of $\Sigma$, where
$\Delta$ is a maximal block properly contained in $\Sigma$.
\end{enumerate}

Note that there is a natural action of $G$ on nodes at each level: $g
\in G$ maps node $u$ to $v$ iff there is a $G$-block $\Delta$ such
that the labels of $u$ and $v$ are $\Delta$ and $\Delta^g$
respectively. Notice that $G$ acts primitively on the children of the
root.

Let $G$ be a permutation group on $\Omega$ with orbits
$\Omega_1,\ldots,\Omega_r$. The \emph{structure forest} is a
collection of structure trees $T_1,\ldots,T-r$, where $T_i$ is the
structure tree of the transitive action of $G$ restricted to
$\Omega_i$.


\section{Overview of the algorithm}\label{section3}

Let $\Omega=\{1,~2,\ldots,~n\}$ and $G\leq Sym(\Delta)$ be a group.
The group $G_{\{\Delta\}}$ is the \emph{pointwise stabilizer} of
$\Delta$, that is the subgroup of elements in $G$ that fix each point
in $\Delta$.

We now define BCGI and a closely related problem $\psb{}{b}$.

\begin{definition}{}
\begin{itemize}
\item The problem $\BCGI_b$ has input instances as pairs of graphs
$(G_1,G_2)$ where the two graphs $G_1$ and $G_2$ are vertex colored
such that each color class is of size at most $b$. The problem is to
test if there is a color-preserving isomorphism between $G_1$ and
$G_2$. We sometimes write BCGI to stand for the problem when the
constant $b$ is not specified.

\item The problem $\psb{}{b}$ has input instances consisting of a
subset $\Delta \subseteq \Omega$ along with a permutation group $G\leq
Sym(\Omega)$ given by its generator set, such that $G$ has orbits of
size bounded by a constant $b$. The problem is to find the pointwise
stabilizer subgroup $G_{\{\Delta\}}$ of $G$.
\end{itemize}
\end{definition}

It is well known that for each constant $b$ there is a constant $c$
such that $\BCGI_b$ reducible to $\psb{}{c}$ \cite{LM,luks}. If we
examine the reduction it is easy to see that it is indeed a logspace
many-one reduction. Thus we have the following proposition.

\begin{proposition}\label{thm-red}
$\BCGI_b$ is logspace many-one reducible to $\psb{}{c}$, where the
constant $c$ depends on the size of the color classes $b$.
\end{proposition}

Thus, in order to prove the upper bound result for BCGI, it suffice to
show that the problem $\psb{}{b}$ is in the $\ModkL$ hierarchy for
some constant $k$. The rest the paper is devoted to this. We first
give an overview of the algorithm, breaking it up into different
tasks. In the subsequent sections we describe algorithms for these
tasks in detail.

Let $\Omega=\cup_{i=1}^{m}\Omega_i$ be the partition of $\Omega$ into
$G$-orbits, where $|\Omega_i|\leq b$ for each i. Let $G_i$ be the
permutation group obtained by projecting $G$ onto $\Omega_i$ for each
$i$. Since $|G_i|\leq b!$ for each $i$, the following proposition is
immediate.

\begin{proposition}
  Given an instance $G\leq Sym(\Omega)$ of $\psb{}{b}$, the orbits
  $\Omega_i$ and the groups $G_i$ obtained by projecting $G$ on
  $\Omega_i$ can be computed in deterministic logspace.
\end{proposition}

Furthermore, for each $G_i$(acting transitively on $\Omega_i$) we can
compute a structure tree of $G_i$ in deterministic logspace: we can
simply search for a minimal block $B_1$ and find the corresponding
block system by letting $G_i$ act on it. Thus we obtain
$\Omega_i=B_1\cup\cdots\cup B_l$, where $\{B_i\}$ is the block system.
Either $G_i$ acts primitively on $\{B_1,\ldots,B_l\}$ or we can again
find a minimal block and a corresponding block system. Continuing
thus, we finally get a list of blocks on which $G_i$ acts primitively.
We get a rooted tree corresponding to the above algorithm: the leaves
are elements of $\Omega_i$, the next level has $l$ nodes corresponding
to each of $B_1,\ldots,B_l$ and, the children of $B_j$ are the
elements of $\Omega_i$ it contains and so on. At the $k^{th}$ level of
the tree are the nodes corresponding to a minimal block system for
$G_i$'s action on the block system at level $k-1$.

\begin{proposition}
For each orbit of $G$, an instance of $\psb{}{b}$ a structure tree can
be computed in deterministic logspace.
\end{proposition}

Since the orbit sizes are constant, the above proposition is
immediate.

Let $T_1,\ldots,~T_m$ be the computed structure trees for the
respective orbits $\Omega_1,\ldots,\Omega_m$ of $G$. Every $T_i$ has
constant height (bounded by $\log b$). Notice that $G_{\{\Delta\}}$,
the subgroup we need to compute, fixes every ancestor of points in
$\Delta$ of a structure tree $T_i,~(1\leq i\leq m)$. This property
will allow us to reduce the problem further as follows:

Consider $G$'s action on the set
$\Omega^{'}=\cup_{i=1}^m\Omega_i^{'}$, where $\Omega_i^{'}$ is the set
of children of the root of $T_i,~(1\leq i\leq m)$. By definition $G$
acts primitively on each $\Omega_i^{'}$.

Let $\Delta^{'}\subseteq\Omega^{'}$ be the ancestors of $\Delta$ in
all the structure trees $T_i$. Then $(G,~\Delta^{'},~\Omega^{'})$ is
an instance of $\psb{}{b}$ with the further restriction that $G$ is
primitive on its orbits. We refer to this restricted problem as
$\prpsb{b}$. Suppose we have an algorithm for $\prpsb{b}$ in the
$\ModkL$ hierarchy for some constant $k$. This algorithm will compute
$G_{\{\Delta'\}}=H$. We will assume that the algorithm will actually
compute a generating set $\{h_1,\ldots,~h_s\}$ for $H$, where it keeps
track of $h_i$'s action on the entire set $\Omega$.

Notice that the group $G_{\{\Delta\}}$ is a subgroup of $H$.  Now
consider the structure forest of $H$. For $\delta\in\Delta$ let
$T_{\delta}$ be the tree with $\delta$ as a leaf in the structure
forest for $H$. It is easy to see that the height of $T_{\delta}$ is
$1$ less than the height of the corresponding tree in the structure
forest for $G$.

Thus, in computing $H$ we have made progress. We can replace $G$ by
$H$, and recompute the structure forest for $H$ and repeat the above.
If we invoke the $\ModkL$ hierarchy upper bound for $\prpsb{b}$ a
constant number of nested oracle levels, where the constant is bounded
by $\log b$, then we would have computed $G_{\{\Delta\}}$.

Putting it together, we have shown the following reduction.

\begin{theorem}
  If $\prpsb{b}$ is in the $\ModkL$ hierarchy then $\psb{}{b}$ is
  also in the $\ModkL$ hierarchy.
\end{theorem}

\begin{remark} 
Both the level of the hierarchy and $k$ are constants that depend on
the orbit bound $b$. We do not make it explicit here.
\end{remark}

Thus the problem that we need to upper bound is $\prpsb{b}$.  In the
remaining sections we will show that this problem is in the $\Mod{k}$
hierarchy for a constant $k$.
 
\section{Basic group theoretic lemmas}

For proving our upper bound on $\prpsb{b}$ we need to recall some
group-theoretic machinery from \cite{luks}. We shall also need to
assemble specific properties of permutation groups with bounded size
orbits.

Let $G \leq \Sym{\Omega}$ and let $\Omega_i$, $1 \leq i \leq m$ be its
orbits such that $\vert \Omega_i \vert \leq b$ for all $i$. Let $G_i$
be the projection of $G$ onto $\Omega_i$. Clearly, $G \leq G_1\times
\ldots\times G_m$. For any $h \in G_1 \times \ldots \times G_m$ let
$\pr{i}{h}$ mean the action of $h$ restricted to $\Omega_i$. For a
subgroup $H$ of $G_1 \times \ldots \times G_m$, we will use
$\pr{i}{H}$ to denote projection of the subgroup $H$ on $\Omega_i$.

We have $G_i \leq \Sym{\Omega_i}$ and $\vert \Omega_i \vert \leq b$
for each i, the set of possible simple subgroups that can occur as
quotients of consecutive groups in any composition series of
$G_1\times \ldots \times G_m$ is a constant sized collection
$\mathcal{T} = \{ T_1,\ldots, T_c\}$ where $ c \leq 2 ^{b!}$. A
deterministic logspace machine can explicitly compute and keep
$\mathcal{T}$ as a table.

\begin{definition}[T-semisimple groups]
  A finite group $G$ is \emph{$T$-semisimple}, for a simple group $T$,
  if $G = G_1 \times \ldots\times G_r$ where each $G_i$ is isomorphic
  to $T$.
\end{definition}

\begin{lemma}{\rm\cite{luks}}\label{lem-unique-res}
  For any simple group $T$ and any finite group $G$, there is a unique
  minimal normal subgroup $N$ such that $G/N$ is $T$-semisimple.
\end{lemma}

 We call $N$ as the $T$-residue of $G$ and denote it by $\Res{T}{G}$.
By Lemma~\ref{lem-unique-res}, notice that $\Res{T}{G}$ is invariant
under any automorphism of $G$.

\begin{definition}
  A subgroup $H$ of a finite group $G$ is a \emph{characteristic subgroup} if
  $H$ is invariant under all automorphisms of $G$.
\end{definition}

We recall two useful properties.

\begin{proposition}\label{prop-char-subgroup}
  \begin{enumerate}
  \item If $H$ is a characteristic subgroup of $G$ then $H$ is a normal
    subgroup of $G$.
  \item If $K$ is a characteristic subgroup of $H$ and $H$ is a
    characteristic subgroup of $G$ then $K$ is a characteristic
    subgroup of $G$.
\end{enumerate}
\end{proposition}

For any simple group $T$, notice that $\Res{T}{G}$ is a characteristic
subgroup of $G$.

\begin{definition}
  A \emph{residual tower} of $G$ is a normal series $G = R_0 \rhd R_1
  \ldots\rhd R_l = 1$ where for all $1 \leq i \leq l$, $R_i =
  \Res{T_i}{R_{i-1}}$ for some simple group $T$.
\end{definition}

Using Proposition~\ref{prop-char-subgroup} each $R_i$ is a
characteristic subgroup of $G$ and hence is normal.

\begin{definition}
  The \emph{socle} $\Soc{G}$ of a finite group $G$ is the subgroup
  generated by the minimal normal subgroups of $G$.
\end{definition}

\begin{proposition}
  The socle $\Soc{G}$ is a characteristic subgroup of $G$.
\end{proposition}

The following lemma is a crucial property of residual towers for
primitive groups.

\begin{lemma}{\rm\cite{luks}}\label{lem-smallest}
  If $H$ is a primitive permutation group acting on $\Omega$ then the
  smallest nontrivial group in any residual tower is always $\Soc{H}$.
\end{lemma}

Now we consider an instance $(G,\Delta,\Omega)$ of $\prpsb{b}$. As the
orbits $\Omega_i$ are of constant size, we can in deterministic
logspace compute a residual tower $\{ R_{i,j} \}_{j=1}^l$ for each
$G_i$. We assume, without loss of generality, that the length $l$ of
residue sequences of $G_i$'s are all same. Notice that $l$ is a
constant bounded by $\log b$.

We describe a series of groups for $G$ obtained from these residual
towers.\footnote{This is different from the series used in
\cite{luks}, and is more convenient for proving the results of this
paper.}

First consider the groups $S_j = R_{1,j} \times \ldots \times
R_{m,j}$.  Notice that $S_0 = G_1 \times \ldots \times G_m$.
Furthermore, since $R_{i,j} \lhd G_i$ for $1 \leq i \leq m$ and $1
\leq j \leq l$, it follows easily that we have a normal series
\[
1 = S_l \lhd S_{l-1} \lhd \ldots \lhd S_1 \lhd S_0 = G_1 \times \ldots
\times G_m.
\]

We refine this normal series further as follows. Consider the list of
simple groups $\{ T_1,\ldots, T_c\}$ as an ordered list, we insert a
series of length at most c between $S_j$ and $S_{j+1}$ for each $j$.
\[
S_j=Q_0 \rhd Q_1 \rhd Q_2 \ldots Q_c = S_{j+1}.
\]
where, for $1\leq k\leq c$, the $Q_k = R_{1,j}^{(k)} \times \ldots
\times R_{m,j}^{(k)}$ is given by
\[
R_{i,j}^{(k)} = \left\{ \begin{array}{cc}
    R_{i,j+1}& \textrm{if } R_{i,j}/R_{i,j+1} \cong T_k\\
    R^{(k-1)}_{i,j} & \textrm{otherwise}
    \end{array}
  \right.
\]

Notice that each $Q_k$ is a normal subgroup of $S_0$. Furthermore
$Q_k/Q_{k+1}$ is either trivial or is $T_k$-semisimple. As a result we
have a normal series of length at most $cl$ (a constant) of the
following form
\begin{equation}\label{restower}
G_1 \times \ldots \times G_m = R_0(G) \rhd R_1(G) \rhd \ldots \rhd
R_k(G) = 1,
\end{equation}
where $R_i(G)/R_{i+1}(G)$ is $T$-semisimple for some $T \in
\{T_1,\ldots T_c\}$.

Taking $N_i = G \cap R_i(G)$ we have the following \emph{residue
series}
\begin{equation}\label{res}
G = N_0 \rhd \ldots \rhd N_k = 1,
\end{equation} 
where $N_i/N_{i+1}$ can be embedded $R_i(G)/R_{i+1}(G)$. It
follows that $N_i/N_{i+1}$ is also $T$-semisimple for some $T \in
\mathcal{T}$. Furthermore, $N_i \lhd G$ for all $i$.

We need the following properties of this series (as in \cite{luks}).

\begin{lemma}{\rm\cite{luks}}\label{lem-N-soc}
  Let $N$ be a group in the normal series of
  Equation~\ref{res}. Then for any subgroup $H \leq G$
  the projection $\pr{i}{H} = G_i$ if and only if $\pr{i}{H \cap N} =
  \pr{i}{N}$.
\end{lemma}

The following characterization of the socles of primitive permutation
groups is also needed.

\begin{theorem}[O'Nan-Scott theorem]\label{onan}
  Let $G \leq \Sym{\Omega}$ be a primitive permutation group with
  socle $\Soc{G} = K$. Let $\omega \in \Omega$. Then $K$ is
  $T$-semisimple for some simple group $G$ and one of the following
  holds:
  \begin{enumerate}
  \item[(i)] $K$ is abelian in which case $K$ is elementary abelian,
    regular on $\Omega$, and is the unique minimal normal subgroup of
    $G$. Furthermore $K_{\{\omega\}} = 1$.
  \item[(ii)] K is nonabelian, transitive on $\Omega$, and is the unique
    minimal normal subgroup of $G$.
  \item[(iii)] $K$ is nonabelian and is a product $K = K_1 \times
    K_2$, where $K_1$ and $K_2$ are isomorphic. $K_1$ and $K_2$ are
    the only minimal normal subgroup of $G$, and each $K_i$ acts
    regularly on $\Omega$. For each $\omega\in\Omega$, $K_{\{\omega\}}
    =\Diag{K_1\times K_2}$.
  \end{enumerate}
 \end{theorem}
 
Another crucial result is Scott's lemma as stated in \cite{luks}. We
state a version useful for us. Let $G$ and $H$ be finite isomorphic
groups. Then for an isomorphism $\varphi$ from $G$ to $H$, the
diagonal subgroup $\Diag{G\times H}$ of $G\times H$ defined as follows
\[
\Diag{G\times H}=\{(x,\varphi(x))\mid x\in G\}.
\]
Clearly, $\Diag{G\times H}$ is isomorphic to each of $G$ and $H$.  If
$\{H_i\mid 1\leq i\leq r\}$ is a set of mutually isomorphic finite
groups, we can similarly define $\Diag{\times_{i=1}^r H_i}$ as a
subgroup of $\prod_{i=1}^r H_i$, given isomorphisms from $H_1$ to each
of the $H_i$, $i\neq 1$.

\begin{lemma}[Scott's Lemma]\label{scott}
Let $T$ be a finite simple group and let $G$ be any subgroup of
$\prod_{i=1}^r T_i$ such that each $T_i$ is isomorphic to $T$.  Then
$G$ is a direct product of diagonal subgroups. More precisely, there
is a partition $\cup_{j=1}^s I_s$ of $\{1,\ldots,r\}$ such that
\[
G=\prod_{j=1}^s\Diag{\times_{i\in I_j}T_i}.
\]
\end{lemma}

We now give a brief overview of the algorithm for $\prpsb{b}$. Let
$(G,\Delta)$ be an instance of $\prpsb{b}$ and let $G = N_0 \rhd
\ldots \rhd N_l$ be the residue series for $G$ as defined in
Equation~\ref{res}. The algorithm claimed in the following
theorem is the key subroutine for solving $\prpsb{b}$.

\begin{theorem}\label{thm-cutting-down}
Given a instance $(G,\Delta)$ of $\prpsb{b}$ and a residue series $G =
N_0 \rhd \ldots \rhd N_l=\{1\}$ for the group $G$, there is a sequence of
subgroups $G=H_0,H_1,\ldots,H_l$ of $G$ such that each $H_s$ in the
sequence contains $G_{\{\Delta\}}$, and $|\pr{i}{H_s}|\leq
|G_i|/2$ for all indices $i$ such that $\pr{i}{N_s}=\{1\}$ and
$\Omega_i$ contains a point of $\Delta$. Furthermore, for each $s\geq
0$, given the subgroup $H_{s}$ of this sequence by a generating set
the subgroup $H_{s+1}$ can be computed by an $\Lo^{\ModKL{t}}$
algorithm.
\end{theorem}

For an instance $(G,\Delta)$ of $\prpsb{b}$, call an orbit $\Omega_i$
a \emph{target orbit} if $\Omega_i$ contains a point of $\Delta$.

We compute the sequence of subgroups $H_0, \ldots, H_{l}$ as follows:
To begin with $H_0 = G$. Then, $H_{i+1}$ is computed from $H_i$ by
applying the $\Lo^{\ModKL{t}}$ algorithm of
Theorem~\ref{thm-cutting-down}.

As a consequence of this theorem, the subgroup $H_{l}$ is such that
for all target orbits $\Omega_i$, $|\pr{i}{H_l}|\leq |G_i|/2$. Thus,
if we consider the group $H=H_l$, then notice that $G_{\{\Delta\}}\leq
H$, and $H$ has been cut down on all target orbits. We now recompute
the residue series with $G$ replaced by $H=H_l$ and repeat the above
process.  Since in every such iteration the group is cut down in size
on each target orbit, in a \emph{constant} number of rounds the
algorithm will have converged to $G_{\{\Delta\}}$. This requires only
a constant number of iterations as each $G_i$ is a constant-sized
group. We will prove Theorem~\ref{thm-cutting-down} in
Section~\ref{six}.

For putting $\prpsb{b}$ in the $\Mod{k}$ hierarchy we first need to
compute the residue series of $G$ for an instance $(G,\Delta)$ of
$\prpsb{b}$. In the next section we explain how to compute it in the
$\Mod{k}$ hierarchy using a general notion of strong generating sets
as in \cite{LM}. Combined with the algorithm sketched above this
will prove the upper bound for $\prpsb{b}$.

\section{The Strong Generating Set problem}\label{five}

%\begin{definition}[Bounded color Permutation groups]
%  Let $G$ be a permutation group on $\Omega$. We say that $G$ is a
%  \emph{$c$-bounded permutation group} on $\Omega$ if all the
%  $G$-orbits are of size less than or equal to $c$. A bounded color
%  permutation group $G$ is a $O(1)$-bounded permutation group. The
%  orbits of $G$ will sometimes be called color classes.
%\end{definition}

Let $G\leq\Sym{\Omega}$, where $G$ has orbits $\Omega_i, 1\leq i\leq
m$ of size bounded by a constant $b$. Consider the \emph{residue
series} for $G$ defined by Equation~\ref{res}. In this section we show
that computing a generating set for each subgroup in the residue
series is in the $\Mod{k}$ hierarchy. The method we use is via strong
generating sets and sifts, as invented by Luks and McKenzie
\cite{LM,luks}. We begin with a general definition.
  
\begin{definition}[Strong generator set]
  Let $G$ be a permutation group and let $H$ be a subgroup of $G$. For
  a tower of subgroup $G = G_0 \geq \ldots \geq G_r = H$ let $B_i$, $0
  \leq i < r$, be the set of all left coset representatives of
  $G_{i+1}$ in $G_i$. Every element $g \in G$ can be uniquely
  expressed as a product $g = g_0g_2\ldots g_{r-1}h$ where $g_i \in
  B_i$ and $h \in H$. The collection of sets $B_i$, $0 \leq i < r$ is
  called a \emph{strong generating set} of $G$ relative to $H$
  (hereafter abbreviated as SGS for $G$ rel $H$). If $H =1$ then $\cup
  B_i$ is a generating set for $G$ which is called a strong generating
  set for $G$.
\end{definition}

Given an element $g \in G$ and an SGS (say, $\cup B_i$) of $G$ rel
$H$, there is a unique $h \in H$ such that $g$ can be written as a
product $g = g_0 g_1 \ldots g_{r-1}h$ where $g_i \in B_i$. By the
\emph{sift} of $g$, denoted by $\Sift{g}$, we mean this element
$h$. Note that the sift of an element depended on the choice of coset
representatives $B_i$ in the SGS of $G$.

We are interested in the special case when $H \lhd G$. For an SGS of
$G$ rel $H$ arising {from} a \emph{normal series} between $G$ and $H$,
we have the following lemma.

\begin{lemma}\label{lem-sift}
  Let $G$ be a permutation group and $H$ be any normal subgroup of
  $G$. Let $G = G_0 \rhd G_1\rhd \ldots \rhd G_k = H$ be a normal series
  between $G$ and $H$ and let $B_i$'s be the left (right) coset
  representatives of $G_{i+1}$ in $G_i$.  Let $A = \{
  g_1,g_2,\ldots,g_l\}$ be a set of generators for $G$ Let $S$ be the
  set consisting of the following elements
\begin{enumerate}
  \item $\Sift{g}$ for all $g \in A$.
  \item \label{prop-commutator}$\Sift{x^{-1}yx}$ for all $x \in B_i$ and
    $y \in B_j$, $i \leq j$.
  \item \label{prop-product} $\Sift{xy}$ for all $x,y \in B_i$ for all
    $i$.
\end{enumerate}
Then $\NCL{G}{S} = H$.
\end{lemma}

\begin{proof}
  Since $H$ is a normal subgroup of $G$ and since all elements of $S$
  are contained in $H$ it is clear that $\NCL{G}{S} \leq H$. To prove
  the converse consider any element $h \in H$. There exists elements
  $y_1,y_2,\ldots,y_m$ in $A$ such that $h = y_1y_2\ldots y_l$.  Now
  since $S$ contains the sifts of all $y$'s we have
  \[
  h = \prod_{i=1}^m x_{i,1}x_{i,2}\ldots x_{i,r} s_i.
  \]
  where $x_{i,j} \in B_j$ and $s_i$ in $\NCL{G}{S}$. Using the fact
  that $\NCL{G}{S}$ is a normal subgroup and using properties
  \ref{prop-commutator} and \ref{prop-product} repeatedly we can
  rewrite the above product as $h = x_1x_2\ldots x_r s$ where $x_i \in
  B_i$ and $s \in \NCL{G}{S}$. Now since $h \in H$ we have $x_1 x_2
  \ldots x_r =1$ and hence $h = s$.
\end{proof}

\subsection{Computing the Residue series}

Let $G\leq \Sym{\Omega}$ with bounded orbits, given by a set of
generators $A$. Let $\Omega_1,\ldots,\Omega_m$ be its orbits of size
at most $b$. Consider the residue series 
\[ 
G = N_0 \rhd N_1 \rhd \cdots \rhd N_k=\{1\}.
\] 
Recall that $k$ is a constant depending on $b$, and each quotient
$N_i/N_{i+1}$ is $T$-semisimple for some simple group
$T\in\{T_1,\ldots,T_c\}$, where $c$ is another constant depending on
$b$.

The algorithm proceeds in stages: at the $i$th stage, we assume that
an SGS for $G$ rel $N_i$ is already computed, and the task at this
stage is to extend it to an SGS for $G$ rel $N_{i+1}$. To do this,
following the approach of \cite{LM}, we will compute an SGS for $N_i$
rel $N_{i+1}$ and append it to the SGS for $G$ rel $N_{i}$. If we can
show that this step can be carried out in $\Lo^{\Mod{t}}$ for a
suitable constant $t$, the claimed $\Mod{k}$ hierarchy upper bound
will follow (where the level of the hierarchy is the number of
stages).

In order to compute an SGS for $N_i$ rel $N_{i+1}$, notice that it
 suffices to solve the problem defined in the theorem below (we can
 choose, in the theorem statement below, $L=R_i(G)$ and $M=R_{i+1}(G)$
 as defined in Equation~\ref{restower}).

\begin{theorem}\label{thm-normal-closure}
  Let $G\leq\sym{\Omega}$ with orbits $\Omega_i$, $1 \leq i \leq m$
  given by generator set $A$. Let $L = L_1 \times \ldots \times L_r$
  and $M = M_1 \times \ldots \times M_r$ be such that $\Sym{\Omega_i}
  \geq L_i \geq M_i$ and $G$ normalizes both $L$ and $M$ and $L_i/M_i$
  is $T$-semisimple for some simple group $T$. Given a set $S$
  of elements of $G$ such that $\NCL{G}{S} = G \cap L$ there is an
  $\Lo^{\Mod{t}}$ algorithm to compute an SGS for $G \cap L$ rel $G \cap
  M$, for some constant $t$.

Furthermore, for any $x\in G\cap L$ there is an $\Lo^{\Mod{t}}$
algorithm to compute the $\Sift{x}$ w.r.t.\ the SGS for $G \cap L$ rel
$G \cap M$ computed above.
\end{theorem}

\begin{proof}
  The group $L/M$ is $T$-semisimple as each $L_i/M_i$ is
  $T$-semisimple. Consequently, $L/M$ is of the form $T_1 \times T_2
  \times \ldots T_s$ where $T_i \cong T$ for all $1 \leq i \leq s$,
  for some $s=O(m)$, where the constant factor depends on
  $b$. Moreover, $G\cap L /G \cap M$ can be faithfully embedded into
  $\prod_{i=1}^m L_i/M_i$.  

We handle the proof of the theorem in two cases:

\subsection*{$T$ is nonabelian.} 

By Scott's Lemma~\ref{scott} we know that $G\cap L / G \cap M$ is a
product of diagonal groups, each isomorphic to $T$. Let $\phi_i : L
\mapsto T_i$ be the map obtained by composing the quotient map {from}
$L$ to $L/M$ and the projection map to $T_i$. We say that $T_i$ and
$T_j$ are \emph{linked} if $G \cap L/ G\cap M$ restricted to $T_i
\times T_j$ is the diagonal group. Consider an undirected graph
$\mathcal{G}$ with vertex set $V = \{ T_i : 1 \leq i \leq t \}$ and
edge set $ \{(T_i,T_j) : T_i \textrm{ and } T_j \textrm{ are
linked}\}$. The groups $T_i$ and $T_j$ are of bounded size hence we
can check whether $T_i$ and $T_j$ are linked in deterministic logspace
as follows:
  
Find $G\cap L/G\cap M$ restricted to $T_i \times T_j$. Since this is a
constant size group one can easily determine whether the group is $T_i
\times T_j$ or $\mathrm{diag}(T_i\times T_j)$ (by checking the order
for example). To find $G \cap L / G \cap M$ we first compute the set
$\phi_{ij}(S) = \{ \langle \phi_i(s) , \phi_j(s) \rangle : s \in
S\}$. We keep on adding new elements by taking the $G$ conjugates of
elements of $S$ and restricting it to the $T_i$ and $T_j$ until no
more elements can be added. This will require only a constant number
of such additions, and hence can be done in logspace.
  
Now in the graph $\mathcal{G}$ we find the connected components (in
$\Lo^{\SL}=\Lo$). Let the connected components be $C_k$, $1 \leq k \leq
l$. For each connected component $C_k$ of $\mathcal{G}$ we pick a
vertex, say $T_{i_k}$.

%In the next step, that can be carried out in deterministic logspace,
%we to compute the SGS for $G\cap L/G\cap M$.

The main algorithmic step is the computation of elements $g_k \in G
\cap L$, $1 \leq k \leq l$ such that $\phi_{i_j}(g_k) \in T_{i_j}$ is
identity for all $j$ except $j = k$. Assuming that we have these
elements, consider the normal subgroup $\NCL{G}{g_k}$ of $G\cap
L$. Since $\phi_{i_k}$ is an onto homomorphism and $T_{i_k}$ is
simple, it follows that $\phi_{i_k}(\NCL{G}{g_k})$, being a nontrivial
normal subgroup of $T_{i_k}$, must in fact be $T_{i_k}$. We compute a
set $B_k$ of distinct inverse images of $\phi_{i_k}(\NCL{G}{g_k})$, $1
\leq k \leq l$. It is easy to see that $\cup_{k=1}^l B_k$ is an SGS of
$G \cap L$ rel $G \cap M$.

In the sequel, we explain how to compute the $g_k$'s. It suffices to
explain how to compute $g_1$.

To begin with we define \emph{iterated commutators}. Let
$\commu{h_1,\ldots,h_k}$ be defined as
\begin{eqnarray*}
\commu{h_1,h_2} &=& h_1^{-1}h_2^{-1}h_1h_2,\\
\commu{h_1,\ldots,h_i,h_{i+1}} &=& \left[\commu{h_1,\ldots,h_i},h_{i+1}\right].
\end{eqnarray*}

We will compute a sequence of elements $h_1, h_2,\ldots,h_l$
satisfying the following property.

  \begin{enumerate}
  \item $\phi_{i_1}(h_1) \neq 1$,
  \item $\phi_{i_1}(\commu{h_1,h_2,\ldots,h_k}) \neq 1$ for all $k$
    and
  \item $\phi_{i_k}(h_k) = 1$.
  \end{enumerate}
  
  It is easy to see that given $h_1, h_2,\ldots,h_l$ as above, $g_1 =
  \commu{h_1,h_2,\ldots,h_l}$ has the required properties:
  $\phi_{i_1}(g_1) \neq 1$, and $\phi_{i_k}(g_1) =1$ for $2 \leq k
  \leq l$. 

We now give a logspace algorithm to actually find such a sequence
$h_1, h_2,\ldots,h_l$.

  \begin{algorithm}
    $x$ := $h_1$ such that $\phi_{i_1}(h_1) \neq 1$ \;
    $\alpha$ := $\phi_{i_1}(h_1)$\;
    \For{$k = 2$ \KwTo $l$} 
    { 
      compute $h_i$ such that $\phi_{i_1}(h_k)$ does not commute with
      $\alpha$ and $\phi_{i_k}(h_k) = 1$\; 
      $\beta$ := $\phi_{i_1}(h_1)$ \;
      $\alpha$ := $[\alpha,\beta]$ \;
      output $h_k$\;
    }
\end{algorithm}

Having got the sequence $h_1,\ldots,h_l$ we now need to compute the
iterated commutator $\commu{h_1,\ldots,h_l}$ in logspace. Then,
combining the two logspace computations we get a logspace algorithm
for computing $g_1$.

\begin{claim}{\label{lem-iter-commu}}
  Let $G$ be a permutation group with bounded-size orbits. Given
  $h_1,\ldots,h_n \in G$, the iterated commutator
  $\commu{h_1,\ldots,h_n}$ can be computed in deterministic logspace.
\end{claim}

It suffices to compute $\commu{h_1,\ldots,h_n}$ restricted to each
$G$-orbit $\Omega$ separately. Since $G$ has bounded-size orbits, and
since the iterated commutator $\commu{h_1,\ldots,h_n}$ is a formula
over the $h_i$'s, we can evaluate $\commu{h_1,\ldots,h_n}$ restricted
to each orbit $\Omega_i$ in $\NCone$ (and hence deterministic
logspace). The claim follows.

Finally, we claim that $B_k$ can be computed in deterministic logspace
from $g_k$. This is possible because it suffices to carry out the
normal closure restricted to one of orbits on which $g_k$ is
nontrivial. However, notice that the computation will obtain the
elements of $B_k$ as permutations on the whole set $\Omega$. There
will be only a constant number of elements to be included in each
$B_k$. Thus, we have computed $\cup_{k=1}^l B_k$ which is an SGS of
$G\cap L$ rel $G\cap M$.

We now explain how to compute $\Sift{x}$ for any $x\in G\cap L$
w.r.t.\ this SGS:

Let $x\in G\cap L$ to be sifted. We apply $\phi_{i_k}(x)$ which is in
$T_{i_k}$ for each $k$, corresponding to the vertices $T_{i_k}$ picked
from each connected component $C_k$ of $\mathcal{G}$, $1\leq k\leq l$.
Since $\phi_{i_k}(B_k)=T_{i_k}$, for each $k$, in logspace we can find
an $x_k$ in the SGS such that
$\phi_{i_k}(x_k)=(\phi_{i_k}(x))^{-1}$. It is now easy to see that
$\Sift{x}=x\prod_{k=1}^lx_k$.

This completes the proof for the nonabelian $T$.
 
\subsection*{$T$ is abelian.}

As $T$ is abelian and simple, for some prime $p\leq b$, $T$ is
isomorphic to the additive group $\Z_p$. Since $L/M$ to be
$T$-semisimple, we have $L/M = T_1 \times T_2 \times \ldots T_s$ where
$T_i \cong \Z_p$ for each $i$, and $s=O(m)$. Thus, $L/M$ is
essentially $\prod_{i=1}^m \Z_p$. But $\prod_{i=1}^m \Z_p$ has more
structure because it is an $m$-dimensional vector space over the
finite field $\F_p$. Furthermore, since $G\cap L /G \cap M$ can be
faithfully embedded in $\prod_{i=1}^m L_i/M_i$, we know that $G\cap L
/G \cap M$ is essentially a subspace of $\prod_{i=1}^m \Z_p$.

Notice that the vector space structure of $L/M=\prod_{i=1}^m \Z_p$ is
effectively obtained: we have the generating set of $L/M=\prod
L_i/M_i$, say $\{s_1M,\ldots,s_rM\}$, in terms of the generating sets
for $L_i/M_i$, where we know the permutations $s_i$ on the entire set
$\Omega$. Thus, in deterministic logspace we can convert an element
$sM$ of $L/M$ into the corresponding vector $\v_s\in\prod_{i=1}^m
\Z_p$. Although the computation of SGS for $G\cap L$ rel $G\cap M$ and
the Sift operation will be solved as a linear algebra problem modulo
$p$, alongside we need to compute the SGS for $G\cap L$ rel $G\cap M$
in as elements of $\Sym{\Omega}$. Similarly, we need to do the Sift
operation on elements of $\Sym{\Omega}$. Thus, it is important for the
algorithm to keep track of the permutation $\pi$ corresponding to each
vector $\v_{\pi}$ that is computed. However, notice that this is easy
to do because for $\v_{\pi}$ expressed as an $\F_p$ linear combination
of the $\v_{s_i}$'s. More precisely, we know that if $s$ and $s'$
correspond to $\v_s$ and $\v_{s'}$, then $ss'$ corresponds to
$\v_s+\v_{s'}$ and $s^k$ corresponds to $k\v_s$ for $k\in\Z_p$.

Now, given $S\subset G$ such that $\NCL{G}{S} = G \cap L$, we will
show how to compute an SGS for $G \cap L$ rel $G \cap M$ in
$\FL^{\Mod{p}}$. We need the following observation from \cite{LM}
about the normal closure that will convert the problem to a simple
linear algebra problem over $\F_p$ that can be solved in
$\FL^{\Mod{p}}$ \cite{allender99complexity,DH}.

Define $\tau_g:L/M\rightarrow L/M$ as $\tau_g(h)=g^{-1}hg$ where $g$
is a generator of $G$ and $h\in L/M$.  The mapping $\tau_g$ is clearly
an automorphism of the group $L/M$. Since $L/M$ is the vector space
$\prod_{i=1}^m \Z_p$, $\tau_g$ is an invertible linear transformation
for each $g\in G$. Furthermore, given $g\in G$ in deterministic
logspace we can write down the $m\times m$ matrix $t_g$ corresponding
to $\tau_g$. It is easy to see that the matrix $t_g$ will actually be
a block diagonal matrix, with one block for each $L_i/M_i$, $1\leq
i\leq s$. The logspace algorithm just needs to figure out the block
corresponding to $L_i/M_i$ for each $i$. Since $L_i/M_i$ are constant
sized, this can be easily done by exhaustive search. 

Thus, if $G=\angle{g_1,\ldots,g_q}$, then in logspace we can compute
the list of matrices $t_{g_1},\ldots,t_{g_q}$. Each matrix $t_{g_i}$
here is block diagonal, with each block being a square matrix of size
at most $c$ (where $c$ is a constant depending on $b$). The following
claim is from \cite{LM}.\footnote{In \cite{LM} this is used for
showing that the membership testing problem over permutation groups is
in NC. We have adapted this to the BCGI setting where we crucially use
the fact that the matrices are block diagonal with constant-size
blocks to derive our upper bound.}

\begin{claim}
$\NCL{G}{S}/G\cap M$ is the smallest subspace of $L/M$ that contains
the set of vectors $S$ and is closed under the linear transformations
$\tau_{g_1},\ldots,\tau_{g_q}$.
\end{claim}

Thus, the problem of computing the SGS for $G\cap L$ rel $G\cap M$
boils down to finding the smallest subspace of $\prod_{i=1}^m \Z_p$
containing the vectors $\{\v_s\mid s\in S\}$ and closed under the
linear maps defined by $t_{g_1},\ldots,t_{g_q}$. Since each matrix
$t_{g_i}$ is block diagonal with each block of size bounded by the
constant $c$, the matrix algebra $\mathcal{A}$ that is generated by
$t_{g_1},\ldots,t_{g_q}$ can be characterized by the following claim.

\begin{claim}
Let $\mathcal{M}$ consist of the set of products of all choices of $j$
matrices from $\{t_{g_1},\ldots,t_{g_q}\}$, for each $j$ such that
$1\leq j\leq c^2$. Then the matrix algebra $\mathcal{A}$ is the
$\F_p$ linear span of the set $\mathcal{M}$.
\end{claim}

The above claim follows easily from the fact that the subalgebra
generated by each block can have dimension at most $c^2$. Thus, it
suffices to consider at most $c^2$ many matrix products for all the
blocks to obtain an $\F_p$ linear basis for $\mathcal{A}$.

With the above claim the $\FL^{\Mod{p}}$ algorithm is now easy to
describe. First, a deterministic logspace algorithm can output the set
$\mathcal{M}$, since multiplying a constant number of matrices can be
done in logspace. For each matrix $t\in\mathcal{M}$, notice that the
logspace algorithm can also keep track of the corresponding element
$g\in G$, by carrying out the corresponding multiplication.

Next, a deterministic logspace algorithm can list out the vectors
$t(\v_s)\mid s\in S, t\in\mathcal{M}\}$. The corresponding elements as
permutations of the form $g^{-1}sg$ can be kept track of again. To
obtain the SGS from this set of vectors, we need to pick a maximal
linearly independent subset. Since testing feasibility of linear
equations over $\F_p$ can be done in $\Mod{p}$
\cite{allender99complexity}, it follows that a basis (and hence an
SGS) can be computed in $\FL^{\Mod{p}}$. The corresponding
permutations are also available. Thus, we have the SGS both as
permutations and in vector form. Let the SGS as permutations be
denoted by the list $\{\pi_1,\ldots,\pi_t\}$.

Finally, we need to describe $\Sift{g}$ for any $g\in G$ using this
SGS. We first write $g$ as a vector $\v_g$ in $L/M$ in logspace (by
examining the action of $g$ on each orbit). We write down a system of
linear equations $AX=\v_g$, where the $i$th column $A_i$ of $A$ is
$\v_{\pi_i}$ for $1\leq i\leq t$, and $\pi_i$ are the elements of the
SGS. We compute the unique solution $x_1,\ldots,x_r$ to this in
$\FL^{\Mod{p}}$. Notice that $\Sift{g}=g\prod_{i=1}^r g_i^{-x_i}$.
\end{proof}

Now we are ready to give the overall algorithm for computing the SGS
of a permutation group with bounded orbits via its residue series.

\begin{theorem}\label{thm-SGS}
  Let $G\leq\Sym{\Omega}$ be a permutation group with orbits
  $\Omega_i$ of size bounded by a constant such that $\pr{i}{G}$ is
  primitive on $\Omega_i$ for each $i$. Then an SGS (strong generating
  set) for $G$ w.r.t.\ its residue series can be computed in the
  $\Mod{k}$ hierarchy.
\end{theorem}

\begin{proof}  

Let $G_i$ be the projection $\pr{i}{G}$ of $G$ onto the $i$th orbit
$\Omega_i$. Recall from Equation~\ref{res} that the residue
series of $G$ is
\[
G = N_0 \rhd \ldots \rhd N_k = 1,
\]
which the groups $N_i$ are defined w.r.t.\ the other series given by
Equation~\ref{restower}:
\[
G_1 \times \ldots \times G_m = R_0(G) \rhd R_1(G) \rhd \ldots \rhd
R_k(G) = 1.
\]

Here, $N_i = G \cap R_i(G)$ for each $i$.

We assume as inductive hypothesis that we already have the strong
generating set of $G$ relative to $N_i$, and we have a sifting
procedure for elements of $G$ with respect to this strong generating
set. Furthermore, we assume inductively that computing this SGS and
the sifting procedure can be carried out in the $\Mod{k}$ hierarchy of
some level $j$. Here $k$ is suitably chosen constant depending only on
the orbit bound $b$.

We show using the above inductive assumption and
Theorem~\ref{thm-normal-closure} that the SGS of $G$ rel $N_{i+1}$ and
can be computed in the $j+1$st level of the $\Mod{k}$ hierarchy, and
the corresponding sifting procedure can be carried out in the $j+1$st
level of the $\Mod{k}$ hierarchy.

Using the already computed the SGS of $G$ relative to $N_i$ and the
corresponding sifting procedure, we can compute (in the $j$th level of
the $\Mod{k}$ hierarchy) a subset $S \subseteq G$ such that
$\NCL{G}{S} = N_{i}$ (Lemma~\ref{lem-sift}). Now, using this set $S$,
notice that $N_i=G\cap R_i(G)$ and $N_{i+1}=G\cap R_{i+1}(G)$ are
groups to which we can apply Theorem~\ref{thm-normal-closure}, with
$L=R_i(G)$ and $M=R_{i+1}(G)$, to find both the SGS of $N_i$ rel
$N_{i+1}$ and be able to sift w.r.t.\ this SGS in $\FL^{\Mod{k}}$.
The union of the SGS of $G$ rel $N_i$ with the SGS of $N_i$ rel
$N_{i+1}$ will give the SGS of $G$ rel $N_{i+1}$. Clearly, the
computation can be done in the $j+1$st level of the $\Mod{k}$
hierarchy.

We next need to show how to sift w.r.t.\ this SGS of $G$ rel
$N_{i+1}$. Given an element $g \in G$ we first compute its sift $g'$
w.r.t the SGS of $G$ rel $N_i$. By induction hypothesis, this
computation can be done in $j$th level of the $\Mod{k}$ hierarchy.
Next, since the element $g'\in G\cap N_i$, we can apply the
$\FL^{\Mod{k}}$ sift procedure of Theorem~\ref{thm-normal-closure}
w.r.t.\ to the SGS of $N_i$ rel $N_{i+1}$ to finally obtain the sift
of $g$ w.r.t.\ the SGS of $G$ rel $N_{i+1}$. Again, the overall complexity
lies in the $j+1$st level of the $\Mod{k}$ hierarchy.

Continuing this for the $l$ levels of the residue series, we obtain
the required SGS for $G$. The computation can be carried out in the
$l$th level of the $\Mod{k}$ hierarchy. It is easy to see that the
constant $k$ can be chosen as the product of all distinct primes $p$
such that $N_i/N_{i+1}$ is $\Z_p$-semisimple. This completes the
proof.
\end{proof}

\section{The Pointwise Stabilizer problem}\label{six}

The goal of this section is to prove Theorem~\ref{thm-cutting-down}
which would complete the upper bound proof for $\prpsb{b}$, and hence
for BCGI.

Let $(G,\Delta)$ be an instance of $\prpsb{b}$. For each orbit
$\Omega_i$ consider the projection of the residual tower given by
Equation~\ref{restower} to the orbit $\Omega_i$ resulting in
the following series for $G_i$:
\[
G_i = \pr{i}{R_0(G)} \rhd\pr{i}{R_1(G)} \rhd \ldots \rhd
\pr{i}{R_k(G)} = 1.
\]

Now, since each $G_i$ is primitive, from Lemma~\ref{lem-smallest}, it
follows that the smallest nontrivial group in the above series is
$\Soc{G_i}$, for each $i$. Consequently, in the residue series $G =
N_0 \rhd \ldots \rhd N_l$ of $G$, for each $\Omega_i$ there is some
$N_s$ such that $\pr{i}{N_s}=\Soc{G_i}$ and
$\pr{i}{N_{s+1}}=\{1\}$. Given an index $s$, let $X$ be the set of
target orbit indices $i$ such that $\pr{i}{N_s}=\Soc{G_i}$ and
$\pr{i}{N_{s+1}}=\{1\}$.

Assume, as inductive hypothesis, that we have computed a generating
set for the group $H_s$ in the sequence of groups
$G=H_0,H_1,\ldots,H_l$ claimed in Theorem~\ref{thm-cutting-down} with
the desired properties. Namely, $H_s$ contains $G_{\{\Delta\}}$ as
subgroup and $|\pr{i}{H_s}|\leq |G_i|/2$ for all target orbit indices
$i$ such that $\pr{i}{N_s}=\{1\}$. We will find $H_{s+1}$ as a
subgroup of $H_sN_s$ such that $|\pr{i}{H_s}|\leq |G_i|/2$ for all
$i\in X$. Clearly, $H_{s+1}$ will satisfy
Theorem~\ref{thm-cutting-down}.
 
We first give a broad outline of the $\FL^{\ModKL{t}}$ algorithm for
finding $H_{s+1}$. The actual details of the algorithm varies depends
upon whether the socle is abelian or not. It will be taken up in the
two subsections after the outline.

The algorithm will pick a \emph{critical subset} $Y$ of $X$ and
compute the following: For each generator $x$ of $H_s$, it computes an
element $x'\in G$ obtained as $xy\in H_sN_s$ for a suitably chosen
$y\in N_s$ such that $x'$ fixes $\Delta\cap \Omega_i$ for each $i\in
Y$. Next, the algorithm computes the subgroup $N$ of $N_s$ which
pointwise fixes $\Delta\cap \Omega_i$ for each $i\in Y$. Let
$H=\angle{\{x'\mid x\textrm{ generator of } H_s\}}$. By construction,
$HN$ will pointwise fix $\Delta\cap \Omega_i$ for each $i\in
Y$. Furthermore, the algorithm will choose $Y$ suitably (which will be
explained in the details) so that $|\pr{i}{HN}|\leq |\pr{i}{G}|/2$ for
all $i\in X$.  The subgroup $H_{s+1}$ is defined to be $HN$.

Since the residue series for $G$ has the property that for each $i$
there is an $s$ such that $\pr{i}{N_s}$ is the socle of $G_i$, it
follows that $|\pr{i}{H_l}|\leq |G_i|/2$ for all the target orbits
$\Omega_i$.

\subsection*{When the socle is nonabelian}

Suppose $N_s/N_{s+1}$ is $T$-semisimple for a nonabelian simple group
$T$. Based on the O'Nan Scott Theorem~\ref{onan}, we can partition $X$
into $X_1\cup X_2$, where for each $i\in X_1$, $\Soc{G_i}$ is of type
(ii), and for each $i\in X_2$, $\Soc{G_i}$ is of type (iii). For $i\in
X_1$, $\Soc{G_i}$ is the unique minimal normal subgroup of $G_i$. For
$i\in X_2$, $\Soc{G_i}$ is of the form $K_1\times K_2$, where the two
\emph{socle parts} $K_1$ and $K_2$ are the only two minimal normal
subgroups of $G_i$.

Furthermore, if $K$ is a socle for some $i\in X_1$, or $K$ is a socle
part for for some $i\in X_2$, then $K$ is $T$-semisimple. More
precisely, it is of the form $T_1\times\ldots\times T_d$ for a
constant $d$ with each $T_j\cong T$, and $G$ acts transitively by
conjugation on $\{T_1,\ldots,T_d\}$.

In the following lemma we summarize observations from \cite{luks}
(that follow from the O'Nan Scott Theorem~\ref{onan} and Scott's
Lemma~\ref{scott}). This lemma identifies the group $N_s$ projected on
orbits in $X$ using the socle parts of $\Soc{G_i}$ for $i\in X$.  More
precisely, if $K$ and $K'$ are two socle parts corresponding to any
two indices $i, j\in X$, then $N_s$ projected on $\Omega_i\cup
\Omega_j$ either induces the group $\Diag{K\times K'}$ or $K\times
K'$. In the former case we say that the socle parts $K$ and $K'$ are
\emph{linked} in $N_s$.

\begin{lemma}\label{lem-link}
Let $G = N_0 \rhd \ldots \rhd N_l$ be a residue series for the
permutation group $G$, where $(G,\Delta)$ is an instance of
$\prpsb{b}$. Let $X$ be the set of indices $i$ such that
$\pr{i}{N_s}=\Soc{G_i}$ and $\pr{i}{N_{s+1}}=\{1\}$ where
$N_s/N_{s+1}$ is $T$-semisimple for a nonabelian simple group $T$.
For any pair of indices $i, j\in X$ exactly one of the following
holds:
\begin{enumerate}
\item[(a)] $N_s$ projected on $\Omega_i\cup \Omega_j$ is
of the form $\Soc{G_i}\times\Soc{G_j}$.

\item[(b)] If $i,j\in X_1$ and $N_s$ projected on $\Omega_i\cup \Omega_j$
is of the form $\Diag{K\times K'}$, where $K=\Soc{G_i}$ and
$K'=\Soc{G_j}$.

\item[(c)] If $i,j\in X_2$, and $\Soc{G_i}=K_1\times K_2$ and
$\Soc{G_j}=K'_1\times K'_2$ then $N_s$ projected on $\Omega_i\cup
\Omega_j$ is either of the form $\Diag{K_1\times
K'_1}\times\Diag{K_2\times K'_2}$ or of the form $\Diag{K_1\times
K'_1}\times K_2\times K'_2$.

\item[(d)] If $i\in X_1$ and $j\in X_2$, $\Soc{G_i}=K$ and
$\Soc{G_i}=K_1\times K_2$, then $N_s$ projected on $\Omega_i\cup
\Omega_j$ is of the form $\Diag{K\times K_1}\times K_2$.
\end{enumerate}

Furthermore, observe that if $K=T_1\times\ldots\times T_d$ and
$K'=T'_1\times\ldots\times T'_e$ are two linked socle parts, then $e=d$
and in $\Diag{K\times K'}$, $T_j$ is linked to $T'_{\pi(j)}$ for
some permutation $\pi$ of the indices.
\end{lemma}

%For each $i\in X$ consider $\Soc{G_i}$ which is nonabelian of type
%(ii) or (iii) and is $T$-semisimple for a nonabelian $T$.

In deterministic logspace we first construct an undirected graph
$\mathcal{G}$ with vertex set $V=\{K\mid K\textrm{ is a socle part of
}\Soc{G_i}, i\in X\}$. The edge set $E$ is partitioned into RED and
BLUE edges defined as follows: $\red=\{(K,K')\mid K\textrm{ and }
K'\textrm{ are linked}\}$ and $\blue=\{(K,K')\mid \exists i\in
X:~\Soc{G_i}=K\times K'\}$.

Next, in deterministic logspace the algorithm will find the connected
components in the red subgraph $(V,\red)$. Notice that by Scott's
Lemma~\ref{scott} the red subgraph is transitive. Thus, the connected
components are just cliques and can be easily determined in
deterministic logspace.

\paragraph{Finding $Y$} 
We partition the cliques of the red subgraph of $\mathcal{G}$ into two
sets $\mathcal{T}_1$ and $\mathcal{T}_2$, where a red clique $C$ is
put in $\mathcal{T}_1$ if $C$ contains an element $K=\Soc{G_i}$ for
some $i\in X$. The remaining cliques are put in $\mathcal{T}_2$.

We will find the critical subset $Y$ of $X$ in two stages. In the
first stage we pick $Y_1$ defined as follows: for every clique
$C\in\mathcal{T}_1$ pick a $K=\Soc{G_i}\in C$ and include the index
$i$ in $Y_1$.

We now take the graph $\mathcal{G}$ and delete all the cliques in
$\mathcal{T}_1$ (and the incident edges). Call the new graph
$\mathcal{G}'$. In $\mathcal{G}'$, considering the red cliques as
vertices, we compute the lexicographically first spanning forest of
blue edges in $\Lo^{\SL}=\Lo$.\footnote{Alternatively, we can shrink
the cliques in $\mathcal{G}'$ into vertices and find a spanning forest
of blue edges.}

Let $E'$ be the blue edges in the spanning forest. Recall that each
$e=(K,K')\in E'$ corresponds to the orbit $\Omega_i$ where
$\Soc{G_i}=K\times K'$. The remaining part $Y_2$ of the critical
subset $Y$ consist of such indices $i$ corresponding to edges in $E'$.

Thus, we have determined $Y$ in $\Lo^{\SL}=\Lo$.

\paragraph{Finding $N$ and $H$}
The subgroup $N$ of $N_s$ pointwise fixes $\Delta\cap\Omega_i$ for
each $i\in Y$. I.e.\ $N=\{g\in N_s\mid \delta^g=\delta~,
\forall~\delta\in\cup_{i\in Y}(\Delta\cap\Omega_i)\}$. Notice that,
since $N_{s+1}$ fixes $\Delta\cap\Omega_i$ for each $i\in X$, we have
$N_s\geq N\geq N_{s+1}$. 

We find $N$ in two stages corresponding to $Y_1$ and $Y_2$.

Let $N'=\{g\in N_s\mid \delta^g=\delta, ~\forall~\delta\in\cup_{i\in
Y_1}(\Delta\cap\Omega_i)\}$.

Let the set $S$ denote the SGS for $N_s$ rel $N_{s+1}$. By the
construction of the SGS described in Theorem~\ref{thm-normal-closure}
for the nonabelian case, it is easy to see that for every clique
$C\in\mathcal{T}_1$, there is a subset $S_C$ of the SGS $S$ with the
following property:

\noindent
For a $K\in C$ let the corresponding orbit index be $i$ (note that
$i\in Y_1$). Either $\Soc{G_i}=K$ or $\Soc{G_i}=K\times K'$, and we
have $\pr{i}{S_C}=K$ (note that $K'$ has to be in a different
clique). Furthermore, for each index $j\neq i$ we have
$\pr{j}{S_C}=\{1\}$.

Now, from the elements of $S_C$ for each $C\in\mathcal{T}_1$, we can
compute the subgroup $N'$ of $N_s$ in deterministic logspace. Let
$S'_C\subset S_C$ be those elements that fix $\Delta\cap\Omega_i$ for
each $i\in Y_1$ corresponding to some $K\in C$. Then $N'$ is generated
by the union of the subsets $S'_C$ of $S_C$ for different
$C\in\mathcal{T}_1$.

Furthermore, for each generator $x$ of $H_s$, we can also compute an
element $y\in S$ such that $xy=x'$ pointwise fixes $\cup_{i\in
Y_1}(\Delta\cap\Omega_i)$. More specifically, $y$ can be chosen by
picking appropriate elements from different $S_C$, $C\in\mathcal{T}_1$
and multiplying them. Denote by $H'$ the group generated by these
$x'$'s computed as above.

Next we explain how to handle the critical orbit indices in $Y_2$.
Consider the spanning forest with vertex set $V'=V(\mathcal{G}')$ and
edges $E'$ (already defined, consisting of blue edges). Each tree in
this forest can be considered as a rooted tree, with its root at the
lexicographically least vertex. Thus, each edge in the forest is
directed blue edge.

Consider any $x\in G$. We explain an $\FL^{\SL}=\Lo$ algorithm for
computing an element $y\in N'$ such that $xy$ fixes all the elements
in $\cup_{i\in Y_2}(\Delta\cap\Omega_i)$. In order to do this we define
an $\FL^{\SL}=\Lo$ subroutine that does the following:

\begin{enumerate}
\item[Step 1.]~~It takes as input an edge $\{u,v\}$ in the spanning
forest $(V',E')$. Using an SL oracle it determines the the unique path
$s=u_0,u_1,\ldots,u_{r-1}=u,v=u_r$ in the forest from a root $s$, this
gives the edge $(u,v)$ an orientation (in the direction of the path
from $s$).

\item[Step 2.]~~Let the actual socle parts determining the edges in
this path be $(K_i,L_i)$ for each edge $(u_i,u_{i+1})$. These can be
identified easily in logspace. Now, suppose we have picked elements
$y_0,\ldots,y_{i-1}$ of the SGS of $N_s$ rel $N_{s+1}$ for the first
$i$ edges of this path. Let the orbit corresponding to $(K_i,L_i)$ be
$\Omega_j$ and let $\{\delta\}=\Delta\cap \Omega_j$.  If
$\delta^x=\mu$ and $\mu^{y_{i-1}}=\nu$, then for the edge
$(u_i,u_{i+1})$ pick the lexicographically least element $y_i$ from
the SGS of $N_s$ rel $N_{s+1}$ such that $\nu^{y_i}=\delta$. This is
clearly possible since $L_i$ is transitive and it is not linked to any
socle part associated to an edge at distance smaller than $i+1$.

\item[Step 3.]~~Output the elements $y_i$ corresponding to the edges
of $G$ in increasing order of distance from their corresponding roots.
This can also be done in $\FL^{\SL}=\Lo$.
\end{enumerate}

Define $y$ as a product of the $y_i$'s computed in the order output by
the above subroutine. It is easy to see by construction that $x'=xy$
will fix each point in $\cup_{i\in Y_2}(\Delta\cap\Omega_i)$.

Now, we can define the group $H''$ as the subgroup generated by these
elements $x'$ for each generator $x$ of $H'$. 

Likewise, let $N''$ be the subgroup of $N'$ obtained by computing $z'$
for each generator $z$ of $N'$ using the above subroutine. It is easy
to see that $N''$ is the pointwise stabilizer of $\cup_{i\in
Y}(\Delta\cap\Omega_i)$. 

We define $H_{s+1}=H''N''$. The construction of $H_{s+1}$ enforces
that if $K$ and $K'$ are any two socle parts (vertices in $V$) that
are connected by a path in $(V,\red\cup\blue)$ then in $H_{s+1}$ the
socle parts $K$ and $K'$ are diagonally linked. We first argue that in
$H_{s+1}$ for each $i\in X$, $|\pr{i}{H_{s+1}}|\leq |G_i|/2$. This is
already true for each $i\in Y$ by construction. Let $i\in X\setminus
Y$.

If $\Soc{G_i}$ is of type (ii), then the corresponding socle part $K$
is in a red clique. Furthermore, since the red clique has a
representative $L$ in $Y$ on which $H_{s+1}$ is cut down, due to the
linking of $K$ and $L$, $H_{s+1}$ is cut down in $K$ (and hence $G_i$)
as well. 

If $\Soc{G_i}$ is of type (iii), let $\Soc{G_i}=K_1\times K_2$. Now,
$\pr{i}{H_{s+1}}=\Diag{K_1\times K_2}$, implying that $H_{s+1}$ is cut
down on this orbit as well.

Let $\Sigma=\cup_{i\in Y}(\Delta\cap\Omega_i)$. By construction we have
$H_s\leq H_{s+1}N_s$. Now, inductively assuming that
$G_{\{\Delta\}}\leq H_s$, notice that
\[
G_{\{\Delta\}}\leq (H_s)_{\{\Sigma\}}\leq
(H_{s+1}N_s)_{\{\Sigma\}}=H_{s+1}N''=H_{s+1}.
\]
This completes the proof for the case when the socle is nonabelian.

\subsection*{When the socle is abelian}

We now turn to the case when $\Soc{G_i}$ is abelian for each $i\in X$,
where $X$ is the set of orbit indices for which $\pr{i}{N_s}$ is
abelian. Assume that the group $H_s$ is already computed.

\paragraph{Finding $Y$} 
Suppose $N_s/N_{s+1}$ is $T$-semisimple. Then $T$ is $\Z_p$ for some
prime $p$. By the O'Nan Scott Theorem~\ref{onan}, for each $i$
$\Soc{G_i}$ is a subgroup of $\Z_p^{c_i}$, where the number of copies
of $\Z_p$ is a constant $c_i$ (that depends on $b$). It follows that
$N_s$ projected on the set of orbits $\Omega_i$ for $i\in X$ is
isomorphic to a subgroup of $\Z_p^r$ where $r=\sum_{i\in
X}c_i=O(m)$. Thus, $N_s$ projected on the set of orbits $\Omega_i$ for
$i\in X$ is actually a subspace $U$ of $\Z_p^r$.  Here, in $\Z_p^r$ we
assume that the coordinates corresponding each orbit $i\in X$ are
adjacent and appear in the increasing order of $i\in X$. Furthermore,
we can compute a basis for this projected subspace $U$ from the
generators of $N_s$ in deterministic logspace. Let
$\v_1,\v_2,\ldots,v_q$ be the computed basis. In $\FL^{\Mod{p}}$ we
can compute a special basis for the vector space $U$ as follows: for
each index $j$, $1\leq j\leq r$ form a column vector
$y^{(j)}=(0,\ldots,0,1,y_{j+1},\ldots,y_r)^T$, where the $1$ occurs in
the $j$th position, the first $j-1$ positions have $0$, and
$y_{j+1},\ldots,y_r$ are indeterminates. Let $A$ denote the matrix
with columns as $\v_1,\v_2,\ldots,\v_q$. In $\FL^{\Mod{p}}$ the
feasibility of $Ax=\y^{(j)}$ can be tested, and if feasible then a
solution $\y^{(j)}$ to $y^{(j)}$ can be computed. These vectors
$\y^{(j)}$ clearly form a basis $B$ for $U$.

Define the subset of orbit indices $Y\subset X$ as follows: include
index $i$ in $Y$ if for some index $j$ corresponding to $\Omega_i$,
there is a solution $y{(j)}\in B$. 

\paragraph{Finding $N$ and $H$}
Now, projecting the vectors in $B$ only on the orbits of $Y$, we again
apply an $\FL^{\Mod{p}}$ computation to obtain a new set of vectors
$B'$ such that $B'$ spans $U$ projected on the orbits of $Y$, and for
each orbit index $i\in Y$ there is a vector $z_i\in B'$ such that
$z_i$ is nontrivial on all orbits of $Y$ except $\Omega_i$.  As
mentioned before, notice here that the algorithm can keep track of the
corresponding entire permutation in $G$ while working with a vector at
each stage. Thus, we also have a $\pi_i\in N_s$ which coincides with
$z_i$ on the orbits $Y$.

Now, let $x$ be a generator of $H_s$. By the above construction we
can obtain an element $y\in N_s$ as a product of the form $\prod_{i\in
Y}\pi_i^{k_i}$ such that $x'=xy\in H_sN_s$ fixes $\Delta\cap
\Omega_i$ for each $i\in Y$.

Let $H$ be the subgroup generated by the above elements $x'$.

Next, notice that the subgroup $N\leq N_s$ consisting of elements that
fix $\cup_{i\in Y}\Delta\cap\Omega_i$ is trivial on all the orbits in
$Y$ (by O'Nan Scott Theorem). Furthermore, the definition of $Y$
forces that $N$ is trivial on all $\Omega_i$, $i\in X$. 

Hence, notice that $N_{s+1}\lhd N\lhd N_s$. Thus, $N_s/N$ is also
$\Z_p$-semisimple. To find a generating set for $N$ we will apply the
algorithm of sifting and normal closure from
Theorem~\ref{thm-normal-closure}. To sift we proceeds as follows: for
each generator $x$ of $N_s$, find an element $y\in N_s$ as a product
of the form $\prod_{i\in Y}\pi_i^{k_i}$ such that $x'=xy$ fixes
$\Delta\cap \Omega_i$ for each $i\in Y$. Let $S'$ be this set of
elements $x'$, which is the sift of $x$ in $N_s$ rel $N$. It remains
to compute $\NCL{N_s}{S'}$. This can be easily done in $\FL^{\Mod{p}}$
by applying the algorithm described in the abelian case of
Theorem~\ref{thm-normal-closure} to the set $S'$.

Now, consider the group $H_{s+1}$ defined as $HN$. Since for each
$i\in X$, $\pr{i}{H\cap N_s}$ is trivial, by Lemma~\ref{lem-N-soc} it
follows that $|\pr{i}{H_{s+1}}|\leq |G_i|/2$ for each orbit $i\in X$.

Notice that by construction of $H_{s+1}$ we have $H_s\leq
H_{s+1}N_s$. Let $\Sigma=\cup_{i\in Y}(\Delta\cap\Omega_i)$. Now,
inductively assuming that $G_{\{\Delta\}}\leq H_s$, notice that
\[
G_{\{\Delta\}}\leq (H_s)_{\{\Sigma\}}\leq
(H_{s+1}N_s)_{\{\Sigma\}}=H_{s+1}N=HNN=HN=H_{s+1}.
\]

This completes the proof of the abelian case, and hence the proof of
Theorem~\ref{thm-cutting-down}.\\

\noindent{\bf Acknowledgments.}~~We thank Jacobo Tor\'an for
discussions with him on applying his result \cite[Lemma 4.7]{tor} to
derive that BCGI is hard for the $\Mod{k}$ hierarchy. The first and
second authors are grateful to Johannes K\"obler for hosting their
visits at Humboldt Universit\"at, Berlin under the DST-DAAD project,
and for many useful discussions on BCGI and related questions.


\begin{thebibliography}{MVV87}

\bibitem[ABO99]{allender99complexity}
Eric Allender, Robert Beals, and Mitsunori Ogihara.
\newblock The complexity of matrix rank and feasible systems of linear
  equations.
\newblock {\em Computational Complexity}, 8:99-126, 1999.

\bibitem[AO96]{allender96relationships}
E.~Allender and M.~Ogihara.
\newblock Relationships among {PL}, \#{L} and the determinant.
\newblock {\em RAIRO Theoretical Informatics and Applications,} 
30(1):1--21, 1996.

\bibitem[ARZ99]{allender99isolation}
E.~Allender, K.~Reinhardt, and S.~Zhou.
\newblock Isolation, matching, and counting: Uniform and nonuniform upper
  bounds.
\newblock {\em Journal of Computer and System Sciences}, 59:164-181, 1999.

\bibitem[Al04]{quaderni} E.~Allender.  \newblock Arithmetic Circuits
and Counting Complexity Classes.  \newblock To appear in {\em Quaderni
di Matematica} series, Edited by Jan Krajicek, 2004.

\bibitem[Bab79]{babai}
L. Babai
\newblock Monte Carlo algorithms in graph isomorphism testing.
\newblock {\em Universitat de Montreal Tech. Report D.M.S}, 79-10, 1979.

%\bibitem[CDL01]{DL}
%A.~Chiu, G.~Davida, and B.~Litow.
%\newblock Division in logspace-uniform {NC}$^1$.
%\newblock {\em RAIRO Theoretical Informatics and Applications}, 35:259-276,
%  2001.

\bibitem[BDHM92]{DH}
G.~Buntrock, C.~Damm, U.~Hertrampf, C.~Meinel.
\newblock Structure and Importance of Logspace-MOD Classes.
\newblock {\em Mathematical Systems Theory,} 25(3): 223-237 (1992).

\bibitem[FHL80]{FHL} 
M. L. Furst, J. E. Hopcroft, E. M. Luks.
\newblock Polynomial-Time Algorithms for Permutation Groups.
\newblock In {\em Proceedings of Foundations of Computer Science,}
IEEE Computer Society, 36-41, 1980.

\bibitem[Ha]{hallbook}
M. Hall.
\newblock {\em The Theory of Groups}.
\newblock Macmillan, New York, 1959.

%\bibitem[HAB02]{HAB}
%W.~Hesse, E.~Allender, and D.~A.~M. Barrington.
%\newblock Uniform constant-depth threshold circuits for division and iterated
%  multiplication.
%\newblock {\em Journal of Computer and System Sciences}, 65:695-716, 2002.

\bibitem[JKMT04]{jkmt} 
B.~Jenner, J.~K\"obler, P.~McKenzie, J.~Tor\'an.
\newblock Completeness results for graph isomorphism. 
\newblock {\em J. Comput. Syst. Sciences,} 66(3): 549-566, 2003.

%\bibitem[KM99]{KM99}
%A.~Klivans  and  D.~van Melkebeek.
%\newblock Graph Isomorphism has subexponential size provers unless the
%polynomial time hierarchy collapses.
%\newblock {\em SIAM Journal of Computing,} 31(5): 1501-1526 (2002).

\bibitem[Lu86]{luks} E.~M.~Luks.  
\newblock Parallel algorithms
for permutation groups and graph isomorphism.  
\newblock In {\em Proceedings of the IEEE Foundations of Computer Science},
IEEE Computer Society, 292-302, 1986.

\bibitem[Luk93]{Lu93}
E.~M. Luks.
\newblock Permutation groups and polynomial time computations.
\newblock {\em DIMACS Series in Discrete Mathematics and Theoretical Computer
  Science}, 11:139--175, 1993.

\bibitem[LM88]{LM}
E.~M.~Luks and P.~McKenzie.
\newblock Parallel algorithms for solvable permutation groups.
\newblock {\em Journal of Computer and System Sciences,} 37, 1988, 39-62.

\bibitem[MC87]{MC}
P.~McKenzie and S.~C.~Cook.
\newblock The parallel complexity of abelian permutation group problems.
\newblock {\em Siam Journal of Computing,} 16:880-909, 1987.

\bibitem[Re04]{rein}
O.~Reingold.
\newblock Undirected graph connectivity is in logspace.
\newblock {\em ECCC Technical Report,} TR94-04, Nov. 2004.

\bibitem[RST84]{RST} 
W.~L.~Ruzzo, J.~Simon, and M.~Tompa.  
\newblock Space-bounded hierarchies and probabilistic computation.  
\newblock {\em Journal of Computer and System Sciences,} 28:216-230, 1984.

\bibitem[Tod91]{toda91counting}
S.~Toda.
\newblock Counting problems computationally equivalent to computing the
  determinant.
\newblock Technical Report CSIM 91-07, Department of Computer Science,
  University of Electro-Communications, Tokyo, Japan, May 1991.

\bibitem[Tor04]{tor}
J.~Tor\'an.
\newblock On the hardness of Graph Isomorphism.
\newblock {\em SIAM Journal of Computing,} 33(5): 1093-1108, 2004.

\bibitem[Vin91]{vinay91counting}
V.~Vinay.
\newblock Counting auxiliary pushdown automata and semi-unbounded arithmetic
  circuits.
\newblock In {\em Structure in Complexity Theory Conference}, pages 270-284,
  1991.

\bibitem[Wie64]{wielandt64finite}
Helmut Wielandt.
\newblock {\em Finite Permutation Groups}.
\newblock Academic Press, New York, 1964.

\end{thebibliography}

\end{document}



