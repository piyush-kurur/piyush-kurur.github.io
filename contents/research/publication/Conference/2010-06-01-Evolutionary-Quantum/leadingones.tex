The pseudo-Boolean function~\leadingones counts the number of one-bits preceding the first zero-bit in a bit-string~$\mathbf{x}\in\{0,1\}^n$, that is, let
\begin{equation}
\leadingones(\mathbf{x}):=\sum_{k=1}^n\prod_{i=1}^k \mathbf{x}(i)\,.
\end{equation}


The following theorem can be deduced from~\cite{DJWoneone}.
\begin{theorem}
Let~$\{\mathbf{x}_t\}_{t\in\NSet}$ be the search points generated by the \ooea or \rls maximizing \leadingones.  Then the expected query time is in~$\Theta(n^2)$.
\end{theorem}

The expected query time decreases considerably when we use quantum acceleration. However, it does not decrease quadratically.
\begin{theorem}
Let~$\{\mathbf{x}_t\}_{t\in\NSet}$ be the search points generated by the \qooea or \qrls minimizing \leadingones. Then the expected query time is in~$\Theta(n^{3/2})$.
\end{theorem}

\begin{proof}
Let us first consider \qrls. For all~$0\leq t \leq T$, let~$k_t=\onemax(\mathbf{x}_{t})$ be the number of one-bits in~$\mathbf{x}_t$, and~$i_t=\leadingones(\mathbf{x}_t)+1$ be the position of the first zero-bit in~$\mathbf{x}_t$ (or~$n+1$, if no such bit exists).

Consider~$t\ge 1$. Then~$\SCal_t$ contains exactly one element, namely the element that is obtained from~$\mathbf{x}_{t-1}$ by flipping the $i_{t-1}$-th bit. Consequently, this element is chosen and afterwards~$k_t=k_{t-1}+1$. Thus,~$T=n-k_0$ and
\[
\EXP[\TQSH\mid k_0]=\sum_{t=1}^{n-k_0}p_t^{-\nicefrac{1}{2}}\,.
\]

Recall that~$p_t$ is the probability that~$\MutOp{\mathbf{x}_{t-1}}\in\SCal_t$. This event happens if the $i_{t-1}$-th bit is flipped, that is, $p_t=1/n$. Therefore, 
\[
\EXP[\TQSH\mid k_0] = \sum_{t=1}^{n-k_0} \sqrt{n} = \sqrt{n}(n-k_0)
\]
Since~$\mathbf{x}_0$ is chosen uniformly at random from~$\{0,1\}^n$ it holds by the Chernoff bounds \cite{MitzemacherU05} that $k_0\leq 2n/3$~with a probability bounded away from zero by a positive constant. Hence, $\EXP[\TQSH] = \Omega(n^{3/2})$. On the other hand, $k_0 \geq 0$, and therefore~$\EXP[\TQSH] = \Oh(n^{3/2})$.

Now, let us turn to \qooea with~$i_t$ and~$k_t$ as for the \qrls. For~$t\ge 1$, the set~$\SCal_t$ consists of all boolean vectors that are obtained from~$\mathbf{x}_{t-1}$ if the $i_{t-1}$-th bit is flipped and none of the bits $1,\ldots,i_{t-1}-1$ is flipped. The probability that~$\MutOp{\mathbf{x}_{t-1}}\in\SCal_t$ is then
\[
p_t = \frac{1}{n}\Big(1-\frac{1}{n}\Big)^{i_{t-1}-1}
\]
Thus, since~$(1-1/n)^{n-1}\ge\euler^{-1}$ and~$0\le i_{t-1}\le n$,
\[
\frac{1}{\euler\,n}\le p_t\le\frac{1}{n}\,.
\]

Unlike for the~\qrls, we cannot guarantee~$k_t=k_{t-1}+1$. However, we know that~$i_t\ge i_{t-1}+1$. Thus, we have~$T\le n$ and 
\[
\EXP[\TQSH]\le \sum_{t=1}^n\sqrt{e\,n}\,,
\]
that is,~$\EXP[\TQSH]\in\Oh(n^{3/2})$.

For the corresponding lower bound, we first make the following observation. Let~$t\ge 1$ and~$i\ge i_{t-1}+1$. Since the probability that~$\mathbf{x}_0(i)=1$ is~$1/2$ and since the~$i$-th bit had no influence on whether~$\mathbf{x}_r\in\SCal_r$ for~$s\le t$, the probability that~$\mathbf{x}_t(i)=1$ is as well~$1/2$ because of the symmetry of the mutation operator.

For~$t\ge 1$, we know that~$\mathbf{x}_t(i_{t-1})=0$ and $\mathbf{x}_t(i)=1$ for~$i\le i_{t-1}-1$. Moreover, we have just seen that~$\mathbf{x}_{t}(i)=1$ with probability~$1/2$ for~$i\ge i_{t-1}+1$. Thus, for all~$\ell\ge 1$ the probability that~$i_t-i_{t-1}=\ell$ is~$2^{-\ell}$ and therefore
\[
\EXP[i_t-i_{t-1}]\le\sum_{\ell\ge 1}\ell\, 2^{-\ell}=2\,.
\] 
Since also~$\EXP[i_0]\leq 2$ for the same reason,~$\EXP[i_t]\le 2(t+1)$ holds for all~$t\ge 0$. Thus~$\EXP[i_{\lfloor n/3\rfloor}]\le 2(n/3+1)$ and the probability that~$i_{\lfloor n/3\rfloor}\le n$ is bounded away from zero by a constant~$c>0$. Thus,~$\Prob[T\ge \lfloor n/3\rfloor]\ge c$ and
\[
\EXP[\TQSH]\ge c\cdot\EXP[\TQSH\mid T\ge \lfloor n/3\rfloor]\ge c\,\lfloor n/3\rfloor\sqrt{n}\,,
\]
that is,~$\EXP[\TQSH]\in\Omega(n^{3/2})$. 
\end{proof}
