In this section we describe the Quantum search algorithm and its reformulation quantum probability amplification in a form that is suitable for our purpose. As before, let $\mathcal{S}$ denote the set $\{ 0, 1 \}^n$ of all $n$-bit strings. Let $\mathcal{S}_0$ be a subset of $\mathcal{S}$ for which we are given a membership oracle, i.e. we are given a oracle $M$ from $\mathcal{S}$ to $\{ 0 , 1 \}$ such that $\mathcal{S}_0 = \{ \mathbf{x} | M(\mathbf{x}) = 1 \}$. Our task is to \emph{search} for a string $\mathbf{x}_0$ in $\mathcal{S}_0$ using queries to $M$. In this setting, we are interested in minimizing the number of queries made to $M$.

In an important breakthrough, Grover~\cite{Grover96} gave a quantum algorithm to search for such an element $\mathbf{x}_0 \in \mathcal{S}_0$ that makes only $\sqrt{|\mathcal{S}|/|\mathcal{S}_0|}$ queries to the oracle $M$. One needs to, however, make the oracle $M$ work for quantum states. The standard approach, which we describe briefly for completeness, is to consider the membership oracle as unitary operator~$U_M$ on the $n$-qubit Hilbert space $\mathcal{H}= \mathbb{C}^{2^{\otimes^n}}$ defined as
\[
U_M \ket{\mathbf{x}} = (-1)^{M(\mathbf{x})} \ket{\mathbf{x}}.
\]
One application of this unitary operator $U_M$ is considered as a single query to the membership oracle.

Let $N$ denote the cardinality of the search space $\mathcal{S}$, and let~the cardinality of the set $\mathcal{S}_0$ be $M$. During initialization, Grover's quantum search algorithm prepares the uniform superposition $\ket{\psi_0}= {1}/{\sqrt{N}}\sum_{\mathbf{x} \in \mathcal{S}}\ket{\mathbf{x}}$. The algorithm iteratively applies the Grover step, a unitary operator which we denote by $G$, to $\ket{\psi_0}$.  Let $\ket{\psi_t}$ denote the state after $t$ applications of $G$, i.e. $\ket{\psi_t} = G^t \ket{\psi_0}$. If we choose some appropriate $t$ in $ \Oh(\sqrt{N/M})$ then on measuring the state $\ket{\psi_t}$ we obtain an element $\mathbf{x} \in \mathcal{S}_0$ with constant probability. More precisely, if we write the state as $\ket{\psi_t} = \sum_{\mathbf{x}\in \mathcal{S}} \alpha_x(t)\ket{\mathbf{x}}$, then for $t= \Oh(\sqrt{N/M})$ we have $\sum_{\mathbf{x}\in\mathcal{S}_0} |\alpha_\mathbf{x}(t)|^2$ is a constant (say $1/2$). The exact form of the Grover step $G$ is not relevant (for details see the text book of Nielsen and Chuang~\cite[Chapter 6]{nielsenchuang:book}) but the crucial point is that $G$ can be constructed using one application of the unitary operator~$U_M$. Hence the Grover search makes $\sqrt{N/M}$ queries to the oracle.

Grover search starts with the uniform superposition as a priori there is no specific reason to prefer one bit string over the other. Instead if we start the search algorithm with the state $\ket{\psi_0} = \sum \alpha_{\mathbf{x}} \ket{x}$, then running time will be $\sqrt{1/p}$ where $p = \sum_{\mathbf{x}\in \mathcal{S}_0} |\alpha_{\mathbf{x}}|^2$ is the probability of picking $\mathbf{x}\in \mathcal{S}_0$ had we measured the initial state $\ket{\psi_0}$ directly. This reformulation due to Brassard \emph{et al}~\cite{brassard98quantum} is often called the \emph{quantum probability amplification} or \emph{quantum amplitude amplification} as a quantum algorithm is able to amplify the probability by making just $\sqrt{1/p}$ queries as opposed to $1/p$ required by a classical algorithm.

Grover's search algorithm, however, comes with a caveat. One needs to stop the Grover iteration after $\Theta(\sqrt{N/M})$ steps, for otherwise the probability of getting a favorable $\mathbf{x}_0 \in \mathcal{S}_0$ actually deteriorates. Thus it appears as if without knowing the count $|\mathcal{S}_0|$, or in the case of probability amplification, the probability $p$ of sampling an $x\in \mathcal{S}_0$ under the given distribution, one cannot use Grover search.  However, using phase estimation, Brassard \emph{et   al}~\cite{brassard98quantum} gave a way to overcome this difficulty with essentially no change in the overall running time. {From} now on, by quantum probability amplification we mean this generalized version where we do not need to know the probabilities.

Let $\ket{\psi}$ be any state in the Hilbert space of $n$-qubits and let $\ket{\varphi} = G\ket{\psi}$. Let $\mathcal{D}_\psi$ and $\mathcal{D}_\varphi$ be the probability distributions obtained on $\mathcal{S}$ on measuring $\ket{\psi}$ and $\ket{\varphi}$, respectively. An important property of the Grover operation $G$ is that the conditional probabilities of obtaining $\mathbf{x}\in \mathcal{S}_0$ given $\mathcal{S}_0$ is the same with respect to either of the distribution $\mathcal{D}_\psi$ and $\mathcal{D}_\varphi$ (see the analysis of Grover's search in Section 6.1.3 of Nielsen and Chuang's book~\cite{nielsenchuang:book}). Thus if we start the Grover search on the state $\ket{\psi_0}$ and perform the Grover search, although the probability of obtaining $\mathbf{x} \in \mathcal{S}_0$ improves, the conditional probabilities of obtaining an element in $\mathcal{S}_0$ given the event that we have obtained an element in $\mathcal{S}_0$ remains the same as in the beginning of the algorithm. 

Given a classical sampling algorithm $A$, consider the quantum algorithm that uses probability amplification starting with the initial state obtained by sampling an $\mathbf{x} \in \mathcal{S}$ using the (quantum version of the) sampling algorithm $A$ followed by measurement and repeats the process until it succeeds in getting an $\mathbf{x}_0 \in \mathcal{S}_0$. Since the probability has been amplified to a constant we would repeat this process at most a constant time. Further since the conditional probability of getting a particular $\mathbf{x}_0 \in \mathcal{S}_0$ does not changes after every Grover step, the quantum algorithm will produce a sample $\mathbf{x}_0\in\mathcal{S}_0$ with probability $\Prob[\mathbf{x}_0|\mathcal{S}_0]$.  Hence we have the following reformulation of probability amplification which is more suitable for our analysis.

\begin{theorem}[Probability Amplification]~\\
\label{thm:probamp}
Let~$\SCal$ be a finite search space, $\SCal_0$ be any subset of $\mathcal{S}$ for which there is a membership oracle $M$, and $A$ a sampling procedure that produces a distribution $\mathcal{D}_A$ on $\mathcal{S}$. Let $p$ be the probability $\Prob_{\mathcal{D}_A}[\mathbf{x} \in\mathcal{S}_0]$ of obtaining a element in $\mathcal{S}_0$ on running $A$.  Then there exists a quantum algorithm that makes on expectation $\Theta(p^{-\nicefrac{1}{2}})$ queries to the membership oracle $M$ and samples an element $\mathbf{x}_0$ in $\mathcal{S}_0$ with a distribution $\mathcal{D}_\psi$ on~$\SCal_0$ given by
\[
  \Prob_{\mathcal{D}_\psi}[\mathbf{x}=\mathbf{x}_0]= \Prob_{\mathcal{D}_A}[\mathbf{x}=\mathbf{x}_0|\mathbf{x}\in\mathcal{S}_0]\,.
\]
\end{theorem}

The above quantum algorithm uses probability amplification starting with the initial state obtained by sampling an $\mathbf{x} \in \mathcal{S}$ using the (quantum version of the) sampling algorithm $A$ followed by measurement and repeats the process until it succeeds in getting an $\mathbf{x}_0 \in \mathcal{S}_0$. Since the probability has been amplified to a constant we would repeat this process at most a constant time each of which costs $\sqrt{1/p}$ queries. Further since the conditional probability of getting a particular $\mathbf{x}_0 \in \mathcal{S}_9$ does not changes, we have the above result.
