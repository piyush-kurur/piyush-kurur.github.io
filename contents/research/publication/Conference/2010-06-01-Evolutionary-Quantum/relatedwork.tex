

\section{Related Work}

Quantum algorithms have drawn much attention and there are many problems for which specialized algorithms have been designed, such as searching \cite{Grover96} \cite{BoyerBHT98}, Element Distinctness \cite{Santha08}, Minimum-Finding \cite{DurrH96} and many others (e.g., \cite{DurrHHM04} \cite{BerzinaDFLS04}, \cite{Zhangthesis06}).

Evolutionary Algorithms (EA) aim not to look specifically into a certain problem, but rather provide some wide-applicable algorithms which may be used if it is not possible to analyze the problem thoroughly - either because the problem complexity is too high, or because the algorithm designer can not spend much time and resources into investigating the problem.

Quantum Evolutionary Algorithms (QEA) have the same objective, but are algorithms that run on a quantum computer. They are not to be confused with Quantum-Inspired Evolutionary Algorithms (QIEA), which use ideas from quantum computing, but are algorithms that run on a classical machine.

It is difficult to test QEAs in practice, because quantum computers do not exist apart from proof-of-concept machines. It is possible to run QEAs by simulating a quantum computer on a classical machine (e.g., \cite{SpectorBBS99}), but this approach is cumbersome since such simulations have exponential running time.

Santha and Szegedy have shown \cite{SanthaS09} that quantum and classical query complexities of local search are polynomially related. Moreover, Magniez et al. have shown \cite{MagniezNRS09} that the speed-up in terms of hitting times is exactly quadratic for a large class of algorithms (``quantum random walks'' in the sense of \cite{Szegedy04} - note that there are different notions of quantum random walks, cf. \cite{MagniezNRS07} for an overview).

For some classes of local search problems, more specific and superior algorithms are available: In particular for search spaces with transition probabilities that are symmetric \cite{Szegedy04}, or ergodic \cite{MagniezNRS07}.

In our paper, we do not prove lower bounds. Conjectures of Aaronson \cite{Aaronson06} imply that the EA-algorithms considered in our paper are asymptotically optimal.  He has shown his conjecture for hypercubes, and Zhang \cite{Zhang06} has generalized the result to product graphs.


%\cite{MagniezNRS09}, \cite{Aaronson06}, \cite{DurrHHM04},
%\cite{DurrH96}, \cite{DurrHHM06}, \cite{HoyerNS02}, \cite{SanthaS04},
%\cite{MagniezNRS07}, \cite{Santha08}, \cite{SanthaS09},
%\cite{SpectorBBS99}, \cite{Zhang06}, \cite{LiR07}, \cite{ZhouZHW05},
%\cite{NarayananM96}, \cite{HanK02}, \cite{LiLR07}, \cite{XiaoYLYZ08},
%\cite{ZhangZPW08}, \cite{AraujoNM08}, \cite{MahdabiJA08},
%\cite{ZhangGW08}, \cite{LiSGS09}, \cite{WangWF09}, \cite{XingJBLQW09},
%\cite{ZhangS06}, \cite{Ambainis06}, \cite{Ambainis08},
%\cite{AmbainisSW09}, \cite{AmbainisCGT09}, \cite{HoyerNS01}


