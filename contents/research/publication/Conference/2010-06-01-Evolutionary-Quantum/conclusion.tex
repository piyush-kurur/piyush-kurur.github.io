In this paper, we have presented an approach to evolutionary algorithms on a quantum computer. Essentially, we keep the mutation and selection process from the classical setting and use Grover search in order to find an improved offspring more quickly. Theorem \ref{thm:probamp} tells us that this does not affect the behavior of the algorithm except for getting faster. The approach is universal, i.e., it works for any mutation operator. Lemma \ref{lem:tq} gives us an easy way to compute the expected running time of the quantum algorithms from non-quantum quantities.

Our method yields at most a quadratic speedup. This is similar to other general settings like unordered search~\cite{BennetBBGV1997,Zalka99} or query complexity of local search on a graph~\cite{Aaronson06}, in which it is proven or conjectured that quantum computers can give at most a quadratic speedup. The analyzed examples \onemax, \leadingones, and \discrepancy show that a substantial speedup is possible (as for \leadingones) but is not guaranteed (as for \discrepancy). The harder it is to improve the objective function, the better will quantum acceleration work.


This has a surprising consequence. In order to be able to move on fitness plateaus, in the Step 2.b) of a \rsh the condition ``$f(\mathbf{y}_t) > f(\mathbf{x}_t)$'' is often replaced by the condition ``$f(\mathbf{y}_t) \ge f(\mathbf{x}_t)$ and~$\mathbf{y}_t\neq\mathbf{x}_t$''. We denote this variant by \rsh{}$^*$. In the examples we analyze, this does not make much difference for the classical case. However, for the quantum algorithms, the results differ substantially. Due to lack of space, we summarized the results in Table~\ref{tabA} and omit the proofs, which are very similar to those we carried out.  The reason for the different results is that by allowing equality of the objective functions we increase the number of valid successor states and thus we increase the probability to find such a state. But quantum enhancement is more powerful if these probabilities are small. See Lemmas \ref{lem:tcl} and \ref{lem:tq} for a precise statement. However, there are ways to keep quantum enhancement powerful and still allow the algorithm to move to a successor state with unchanged objective value. We hope to discuss these issues in further work.


We have chosen the problems \onemax, \leadingones, and \discrepancy for two reasons. Firstly, because they demonstrate a wide range of effects that may occur. Secondly, because they are easy to analyze. However, these are of course only toy problems, and we would like to analyze real combinatorial problems. We aim to do so in subsequent work.
