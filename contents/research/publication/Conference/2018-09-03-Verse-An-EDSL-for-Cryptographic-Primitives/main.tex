\documentclass[sigconf,review,final]{acmart}\settopmatter{printfolios=true}%% For double-blind review submission
%\documentclass[acmsmall,review,anonymous,draft]{acmart}\settopmatter{printfolios=true}%% For double-blind review submission
%% For final camera-ready submission
%%\documentclass[acmlarge]{acmart}\settopmatter{}

%% Some recommended packages.
\usepackage[color]{coqdoc}

% Coq typesetting tweaks.
\renewcommand{\coqdockw}[1]{\textbf{\texttt{#1}}}
\renewcommand{\coqdocvar}[1]{{\color{\coqdocvarcolor}{\textbf{\small\texttt{#1}}}}}
% End of coq typesetting tweaks.

\usepackage{booktabs}   %% For formal tables:
                        %% http://ctan.org/pkg/booktabs
\usepackage{subcaption} %% For complex figures with subfigures/subcaptions
                        %% http://ctan.org/pkg/subcaption


%\makeatletter\if@ACM@journal\makeatother
%% Journal information (used by PACMPL format)
%% Supplied to authors by publisher for camera-ready submission
%% \acmJournal{PACMPL}
%% \acmVolume{1}
%% \acmNumber{1}
%% \acmArticle{1}
%% \acmYear{2017}
%% \acmMonth{1}
%% \acmDOI{10.1145/nnnnnnn.nnnnnnn}
%% \startPage{1}
%% \else\makeatother
%% Conference information (used by SIGPLAN proceedings format)
%% Supplied to authors by publisher for camera-ready submission
\acmConference[PPDP-2018]{3-5 September 2018}{Frankfurt am Main}{Germany}
\acmYear{2018}
\acmISBN{NONE}
\acmDOI{NONE}
%\startPage{1}
%\fi


%% Copyright information
%% Supplied to authors (based on authors' rights management selection;
%% see authors.acm.org) by publisher for camera-ready submission
\setcopyright{none}             %% For review submission
%\setcopyright{acmcopyright}
%\setcopyright{acmlicensed}
%\setcopyright{rightsretained}
%\copyrightyear{2017}           %% If different from \acmYear


%% Bibliography style
\bibliographystyle{ACM-Reference-Format}
%% Citation style
%% Note: author/year citations are required for papers published as an
%% issue of PACMPL.
\citestyle{acmauthoryear}   %% For author/year citations

\usepackage{hyperref}

%%%%%% Setting up the Todo list.

\usepackage{flushend}


%% Todo notes should become invisible in the final version.
\usepackage[obeyFinal]{todonotes}
%%\usepackage[disable,obeyFinal,colorinlistoftodos]{todonotes}
\newcommand{\ppk}[1]{\todo[inline,color=red!20!white]{#1 ---PPK}}
\newcommand{\abhi}[1]{\todo[inline,color=blue!20!white]{#1 ---ABHI}}
\newcommand{\citetodo}[1]{\cite{#1}\todo[color=red!80]{Citation: #1}}
\newcommand{\criticaltodo}[1]{\todo[inline,color=red!80]{#1}}

%% Fixing the listoftodos to work with amsart class.
%% \makeatletter
%% \providecommand\@dotsep{5}
%% \renewcommand{\listoftodos}[1][\@todonotes@todolistname]{%
%%   \@starttoc{tdo}{#1}}
%% \makeatother

%%%%%%%%%% END of todnotes

\begin{document}
\title{Verse: An EDSL for cryptographic primitives}
\author{Abhishek Dang}
\affiliation{
  \department{Department of Computer Science and Engineering}
  \institution{IIT Kanpur}
  \streetaddress{}
  \city{Kanpur}
  \state{Uttar Pradesh}
  \postcode{208016}
  \country{India}
}
\email{ahdang@cse.iitk.ac.in}          %% \email is recommended

\author{Piyush P Kurur}
\affiliation{
  \department{Department of Computer Science and Engineering}
  \institution{IIT Kanpur}
  \streetaddress{}
  \city{Kanpur}
  \state{Uttar Pradesh}
  \postcode{208016}
  \country{India}
}
\email{ppk@cse.iitk.ac.in}          %% \email is recommended

\begin{abstract}
  Cryptographic primitives need high-speed implementations that are
  also resistant to side channel attacks. The absolute control over
  instructions and registers that such implementations demand makes
  assembly language programming a necessity. In this article, we
  describe Verse, a \emph{typed low-level} language \emph{embedded} in
  Coq designed specifically to generate assembly language programs for
  cryptographic primitives. Despite being a low-level language, the
  programming experience is markedly high-level:

  \begin{itemize}
  \item The type system of Verse is rich enough to even prevent
    errors in array indexing and endian conversion.
  \item Being embedded in Coq, we have at our disposal Gallina, the
    underlying functional programming language, as a macro assembler
    for code generation, and Ltac, the tactic language, as an
    automation tool for proof obligations inherent to our type
    system.
  \end{itemize}

  We also provide a generic framework to formulate semantic aspects of
  Verse. This framework has value beyond providing an interpretation
  of Verse in Coq. We demonstrate this versatility by using it to
  localise uninitialised/clobbered variable use, and arithmetic
  overflows.

  %%  As an example, we prove the correctness of
  %% implementation of arithmetic in the finite field
  %% $\mathrm{GF}(2^{130} - 5)$ used in Poly1305.

  %% this mechanism can
  %% be specialised to provide checks beyond type safety. Two examples of
  %% such specialised semantics is available in Verse. provided in verse

  %% refactor and
  %% verify programs. Together with features like Sectioning we present
  %% an interface to verse that resembles high-level language. Verse also
  %% gives a uniform representation for instruction across distinct
  %% architecture making it concise and portable.


  %% Besides the

  %% are prevented in Verse by its type system.



\end{abstract}

\keywords{Coq, EDSL, Cryptography, assembly language}  %% \keywords is optional


\begin{CCSXML}
<ccs2012>
<concept>
<concept_id>10011007.10011006.10011050.10011017</concept_id>
<concept_desc>Software and its engineering~Domain specific languages</concept_desc>
<concept_significance>500</concept_significance>
</concept>
<concept>
<concept_id>10002978.10002986.10002990</concept_id>
<concept_desc>Security and privacy~Logic and verification</concept_desc>
<concept_significance>300</concept_significance>
</concept>
</ccs2012>
\end{CCSXML}

\ccsdesc[500]{Software and its engineering~Domain specific languages}
\ccsdesc[300]{Security and privacy~Logic and verification}

%%%%%%%%%%%%%%%%%%%%%%% Legalese from ACM %%%%%%%%%%%%%%%%%%%%%%%%%%%%%%%%%
%%
%% Their documentation is unclear which of the two works I have added both
%% and commented out the one that does not work.
%%
%%
%%%%%%%%%%%%%%%%%%%%%%%% Option one ########################################
\copyrightyear{2018}
\acmYear{2018}
\setcopyright{acmlicensed}
\acmConference[PPDP '18]{The 20th International Symposium on Principles and
Practice of Declarative Programming}{September 3--5, 2018}{Frankfurt am
Main, Germany}
\acmBooktitle{The 20th International Symposium on Principles and Practice
of Declarative Programming (PPDP '18), September 3--5, 2018, Frankfurt am
Main, Germany}
\acmPrice{15.00}
\acmDOI{10.1145/3236950.3236971}
\acmISBN{978-1-4503-6441-6/18/09}

%%%%%%%%%%%%%%%%%%%%% Option two %%%%%%%%%%%%%%%%%%%%%%%%%%%%%%%%%%%%%%%%%%%%

%% \CopyrightYear{2018}
%% \setcopyright{acmlicensed}
%% \conferenceinfo{PPDP '18,}{September 3--5, 2018, Frankfurt am Main, Germany}
%% \isbn{978-1-4503-6441-6/18/09}
%% \acmPrice{$15.00}
%% \doi{https://doi.org/10.1145/3236950.3236971}

%%%%%%%%%%%%%%%%%%%%%%%%%%%% End of ACM legalese %%%%%%%%%%%%%%%%%%%%%%%%%%%%

\maketitle
\listoftodos{}
\input{contents}
\bibliography{references}
%\appendix
%\section{ChaCha20 in verse}
%\input{chacha20}

\end{document}
