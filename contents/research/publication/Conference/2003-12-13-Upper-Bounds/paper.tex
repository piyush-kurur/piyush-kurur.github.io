%\documentclass{llncs}
\documentclass{article}
\usepackage{amssymb}
\usepackage{amsthm}
\newtheorem{theorem}{Theorem}[section]
\newtheorem{lemma}[theorem]{Lemma}
\newtheorem{proposition}[theorem]{Proposition}
\newtheorem{corollary}[theorem]{Corollary}
\newtheorem{definition}[theorem]{Definition}
\newtheorem{problem}[theorem]{Problem}
%\usepackage{showlabels}
%\usepackage{algorithm2e}
%\usepackage{fullpage}
%\pagestyle{empty}
\newcommand{\Ideal}[1]{\ensuremath{\mathfrak{#1}}}
\newcommand{\Sym}[1]{\ensuremath{\mathbf{Sym}\left(#1\right)}}
\newcommand{\Stab}[2]{\ensuremath{\mathbf{Stab}_{#1}\left(#2\right)}}
\newcommand{\Gal}[1]{\ensuremath{Gal\left(#1\right)}}
\newcommand{\Fix}[2]{\ensuremath{Fix_{#1}\left(#2\right)}}
\newcommand{\Q}[0]{\ensuremath{\mathbb{Q}}}
\newcommand{\Z}[0]{\ensuremath{\mathbb{Z}}}
\newcommand{\C}[0]{\ensuremath{\mathbb{C}}}
\newcommand{\F}[0]{\ensuremath{\mathbb{F}}}
\newcommand{\size}[0]{\ensuremath{\mathrm{size}}}
\newcommand{\mod}[0]{\ensuremath{\mathrm{mod}\ }}
\renewcommand{\angle}[1]{\langle #1\rangle}
\newcommand{\Prob}{\ensuremath{\mathrm{Prob}}}
\newcommand{\Tr}{\mbox{\rm Tr}}

\newcommand{\Frob}[2]
{\ensuremath
  {\left(
      \begin{array}{c}
        #1\\
        \hline
        #2
      \end{array}
    \right)
  }
}
\newcommand{\Artin}[2]
{\ensuremath
  {\left[
      \begin{array}{c}
        #1\\
        \hline
        #2
      \end{array}
    \right]
  }
}

\newcommand{\FBPP}{\mbox{\rm FBPP}}
\newcommand{\UP}{\mbox{\rm UP}}
\newcommand{\PP}{\mbox{\rm PP}}
\newcommand{\BPP}{\mbox{\rm BPP}}
\newcommand{\GapP}{\mbox{\rm GapP}}
\newcommand{\NP}{\mbox{\rm NP}}
\newcommand{\FP}{\mbox{\rm FP}}
\newcommand{\p}{\mbox{\rm P}}
\newcommand{\coNP}{\mbox{\rm co-NP}}
\newcommand{\ParityP}{\mbox{$\oplus$\rm P}}
\newcommand{\ModkP}{\mbox{\rm Mod$_k$P}}
\newcommand{\numP}{\mbox{\rm \#P}}

\title{Upper Bounds on the Complexity of some Galois Theory Problems}

%% \subtitle{(Extended abstract)\footnote{The full version of the paper
%%     is available at www.imsc.res.in/$\sim$ppk.}}

\author{ V.~Arvind and Piyush P Kurur\\
  Institute of Mathematical Sciences, C.I.T Campus,\\
  Chennai 600113, India\\
  {\tt \{arvind,ppk\}@imsc.res.in}
}% \date{}
%\usepackage{fancyhdr}
\begin{document}
\maketitle
%%
%% This is the junk I added to get the exact style of the proceedings version
%%

%% \setcounter{page}{716}
%% \fancypagestyle{plain}{
%% 	\fancyhf{}
%% 	\renewcommand{\headrulewidth}{0pt}
%% 	\fancyfoot[L]{\scriptsize
%% 	T. Ibaraki, N. Katoh, and H. Ono (Eds.): ISAAC 2003,
%% 	LNCS 2906, pp 716--725, 2003. \\ \copyright\ Springer-Verlag Berlin
%% 	Heidelberg 2003}
%% }
%% \pagestyle{myheadings}
%% \markboth{\qquad \normalfont{\footnotesize V. Arvind and P.P Kurur} \hfill}
%% {\hfill \normalfont{\footnotesize Upper Bounds on the Complexity of some
%% Galois Theory Problems\qquad}}
%% \thispagestyle{plain}

\begin{abstract}
  Assuming the generalized Riemann hypothesis, we prove the following
  complexity bounds: The order of the Galois group of an arbitrary
  polynomial $f(x)\in\Z[x]$ can be computed in $\p^{\numP}$.
  Furthermore, the order can be approximated by a randomized
  polynomial-time algorithm with access to an NP oracle. For
  polynomials $f$ with solvable Galois group we show that the order
  can be computed exactly by a randomized polynomial-time algorithm
  with access to an NP oracle. For all polynomials $f$ with abelian
  Galois group we show that a  generator set for the Galois
  group can be computed in randomized polynomial time.
\end{abstract}

\section{Introduction}
A fundamental problem in computational algebraic number theory is to
determine the Galois group of a polynomial $f(x) \in \Q[x]$. Formally,
in this paper we study the computational complexity of the following
problem:

\begin{problem}\label{prob1}
Given a nonzero polynomial $f(x)$ over the rationals $\Q$, 
\begin{enumerate}
\item[(a)]~determine the Galois group of $f$ over $\Q$.
\item[(b)]~determine the order of the Galois group of $f$ over $\Q$. 
\end{enumerate}
\end{problem}


%Given as input $f(x) \in \Q[x]$ there is a deterministic algorithm due
%to Landau \cite{landau:1984:galois} that computes the Galois group of
%$f$ in time polynomial in the cardinality of the Galois group (also
%see \cite{lenstra:1992:algorithm}).  However, this is not an efficient
%algorithm as the Galois group can be of cardinality exponential in
%$n$. It is still open if Problem~\ref{prob1}(a), or even
%Problem~\ref{prob1}(b), has a polynomial (in $\size(f)$) time
%algorithm. Neither is a better upper bound than the exponential-time
%algorithm mentioned above known, nor is any nontrivial hardness result
%known for the problem. Problem~\ref{prob1}(b) is polynomial-time
%reducible to Problem~\ref{prob1}(a).\footnote{Given a generator set
%  for a subgroup $G$ of $S_n$ we can compute $|G|$ in time polynomial
%  in $n$ \cite{luks93permutation}.}
%
%The Galois group of a polynomial is a fundamental object of study in
%algebraic number theory. We recall the celebrated result of Galois: a
%polynomial $f$ over $\Q$ is said to be solvable by radicals if we can
%compute its zeroes from the coefficients by the standard arithmetic
%operations and taking $r$th roots, for any positive integer $r$.
%Galois theorem states that a polynomial $f\in\Q[x]$ is solvable by
%radicals if and only if its Galois group is a solvable group. Landau
%and Miller in \cite{landau85solvability} showed that the problem of
%testing whether the Galois group of a polynomial $f\in\Q[x]$ is
%solvable can be done in polynomial time. However, even when the Galois
%group is solvable, no polynomial-time algorithm is known for
%Problem~\ref{prob1}(a) or Problem~\ref{prob1}(b).

%\subsection*{Summary of results}
%
%In this paper we prove the following new complexity upper bounds for
%some special cases of Problems~\ref{prob1}(a) and (b), assuming the
%generalized Riemann hypothesis (henceforth GRH).
%
%\begin{enumerate}
%\item Given a polynomial $f\in\Q[x]$, the order of its Galois group
%  can be computed by a polynomial time algorithm with one query to a
%  $\numP$ oracle. This yields a \emph{polynomial-space} algorithm for
%  Problem~\ref{prob1}(b). In contrast, we observe here that Landau's
%  algorithm \cite{landau:1984:galois} requires more than polynomial
%  space.
%\item If the Galois group of the polynomial is solvable then we get a
%  randomized algorithm with NP oracle that \emph{exactly} computes the
%  order of its Galois group.
%\item Assuming the GRH, we have a polynomial time randomized algorithm
%  for computing the Galois group for a polynomial $f$ with
%  \emph{abelian} Galois group. Previously, a polynomial-time algorithm
%  was known only for the case when $f$ is \emph{irreducible} and has
%  an abelian Galois group \cite{landau:1984:galois} (also see
%  \cite{lenstra:1992:algorithm}), because in that case the Galois
%  group has only $n$ elements.
%\end{enumerate}
%
%Our main tool is an effective version of the Chebotarev density
%theorem, which holds assuming the GRH.
%
%\subsection{Galois theory background}

An \emph{extension} of a field $K$ is a field $L$ that contains $K$
(written $L/K$). If $L/K$ is a field extension then $L$ is a vector
space over $K$ and its dimension, denoted by $[L:K]$ is called its
\emph{degree}.  If $[L : K]$ is finite then $L/K$ is a \emph{finite}
extension. If $L/M$ and $M/K$ are finite extensions then $[L:K] =
[L:M].[M:K]$.

Let $K[x]$ denotes the ring of polynomials with indeterminate $x$ and
coefficients from the field $K$. A polynomial $f(x) \in K[x]$ is
\emph{irreducible} if it has no nontrivial factor over $K$.  The
\emph{splitting field} $K_f$ of a polynomial $f(x) \in K[x]$ is the
smallest extension $L$ of $K$ such that $f$ factorizes into linear
factors in $L$. An extension $L/K$ is \emph{normal} if for any
irreducible polynomial $f(x) \in K[x]$, $f$ either splits in $L$ or
has no root in $L$. An extension $L/K$ is \emph{separable} if for all
irreducible polynomials $f(x) \in K[x]$ there are no multiple roots in
$L$.  A normal and separable finite extension $L/K$ is called a
\emph{Galois extension}.

An \emph{automorphism} of a field $L$ is a field isomorphism
$\sigma:L\rightarrow L$. The \emph{Galois group} $\Gal{L/K}$ of a field
extension $L/K$ is the subgroup of the group of automorphisms of $L$
that leaves $K$ fixed: i.e. for every $\sigma\in\Gal{L/K}$, $\sigma(a)
= a$ for all $a \in K$. By the Galois group of a polynomial $f\in
K[x]$ we mean $\Gal{K_f/K}$.

% For a subgroup $G$ of automorphisms of $L$, the
%\emph{fixed field} $L^G$ is the largest subfield of $L$ fixed by $G$.
%
%We now state the fundamental theorem of Galois.

%\begin{theorem}{\rm{\cite[Theorem 1.1 Chap.
%     VI]{lang:algebra}}}\label{funda:galois} Let $L/K$ be a Galois
%  extension with Galois group $G$. There is a one-to-one
% correspondence between subfields $E$ of $L$ containing $K$ and
% subgroups $H$ of $G$, given by $E \rightleftharpoons L^H$.  The
% Galois group of $\Gal{L/E}$ is $H$ and $E/K$ is a Galois extension
% iff $H$ is a normal subgroup of $G$. If $H$ is a normal subgroup of
% $G$ and $E = L^H$ then the Galois group of $\Gal{E/K}$ is $G/H$.
%\end{theorem}

Roots of polynomials over $\Q$ are \emph{algebraic numbers}. The
\emph{minimal polynomial} $T\in\Q[x]$ of an algebraic number $\alpha$
is the unique monic polynomial of least degree with $\alpha$ as a
root. \emph{Algebraic integers} are roots of monic polynomials in
$\Z[x]$. A \emph{number field} is a finite extension of $\Q$. For an
algebraic number $\alpha$, $\Q(\alpha)$ denotes the smallest number
field that contains $\alpha$. If $f(x)$ is the minimal polynomial of
$\alpha$ then $\Q(\alpha)$ can be identified with the quotient
$\Q[x]/(f(x)\Q[x])$. Every number field $K$ has an element $\alpha$
such that $K = \Q(\alpha)$ (see \cite[Theorem 4.6
Chap.V]{lang:algebra}). Such elements are called \emph{primitive}
elements of the field $K$.

Let $f\in\Q[x]$ with roots $\alpha_1,\alpha_2,\ldots,\alpha_n\in
\Q_f$. A well known lemma \cite{waerden:1991} states that $\Q_f$ has a
primitive element of the form $\sum_{i=1}^n c_i \alpha_i$ for integers
$c_i$. The proof actually yields a probabilistic version which states
that $\sum_{i=1}^n c_i \alpha_i$ is primitive for most $c_i$.

\begin{lemma}\label{random:prim:element}
  Let $f\in \Q[x]$ be a degree $n$ polynomial with roots
  $\alpha_1,\alpha_2,\ldots,\alpha_n$. For a random choice of integers
  $c_1,c_2,\ldots,c_n$ such that $size(c_i) \leq n^2$ the algebraic
  integer $\theta = \sum_{i=1}^n c_i \alpha_i$ is such that $L =
  \Q(\theta)$ with probability $1- \frac{1}{2^{O(n^2)}}$.
\end{lemma}

A polynomial $f(x) \in \Q[x]$ is said to be solvable by radicals if
the roots of $f$ can be expressed, starting with the coefficients of
$f$, using only field operations and taking $r^{th}$ roots for integer
$r$. Galois showed that a polynomial is solvable by radicals if and
only if its Galois group is solvable.

Let $L$ be a number field and $O_L$ be the ring of algebraic integers
in $L$. We can write $O_L$ as $O_L=\{\sum_{i=1}^N a_i\omega_i\mid
a_i\in\Z\}$ where $\omega_1,\omega_2,\ldots,\omega_N$ is its
$\Z$-basis. The \emph{discriminant} $d_L$ of the field $L$ is defined
as the determinant of the matrix $(\Tr(\omega_i\omega_j))_{i,j}$ where
$\Tr:L\rightarrow \Q$ is the trace map. The discriminant $d_L$ is
always a nonzero integer. Let $T$ be any polynomial of degree $N$.
Then the discriminant $d(T)$ of the polynomial $T$ is defined as $d(T)
= \prod_{i \ne j} (\theta_i - \theta_j)$, where
$\theta_1,\theta_2,\ldots,\theta_N$ are the $N$ distinct roots of $T$
(i.e.\ all the conjugates of $\theta$). The following is important
property that relates $d(T)$ and $d_L$ (\cite[Proposition
4.4.4]{cohen:1993}).

\begin{proposition}
  Let $L$ be a number field and $T$ be the minimal polynomial of a
  primitive element $\theta$ of $L$. Then $d_L\mid d(T)$. More
  precisely, $d(T) = d_L\cdot t^2$, for an integer $t$.
\end{proposition}

Let $\size(a)$ denote the length of the binary encoding of an integer
$a$. For a rational $r = p/q$ such that $gcd(p,q) =1$, let $\size(r) =
\size(p) + \size(q)$. A polynomial is encoded as a list of its
coefficients. For a polynomial $f(x) = \sum a_i x^i\in\Q[x]$ we define
$\size(f) = \sum \size(a_i)$. Thus, for an algorithm taking a
polynomial $f$ as input, the input size is $size(f)$.

For any polynomial $g(x) = a_0 + a_1 x + \ldots + a_n x^n$ in $\Z[x]$,
let $|g|_2 = \sqrt{\sum a_i^2}$. Applying an inequality
\cite{landau:1905:ineq} which bounds every root $\eta$ of $g$ by
$|g|_2$, we obtain the following.

\begin{theorem}\label{bound:d:theta}
  Let $f(x) \in \Z[x]$ be a monic polynomial of degree $n$ with
  splitting field $L$. Let $\alpha_1,\alpha_2,\ldots,\alpha_n$ be the
  roots of $f$. Consider an element of the form $\theta = \sum c_i
  \alpha_i$, $c_i\in \Z$, and let $T$ be the minimal polynomial of
  $\theta$. Then $|d(T)| \leq (2c|f|_2)^{N^2}$, where $c =
  {\mathrm{max}}\{ |c_i| : 1 \leq i \leq n \}$.  As a consequence,
  $d_L \leq (2^{n^2} |f|_2)^{n!^2}$ and $\log{d_L} \leq
  (n+1)!^2.\size(f)$.
\end{theorem}

The Galois group of a polynomial $f(x) \in K[x]$ is completely
determined by its action on the roots of $f$ in $K_f$.  We assume
w.l.o.g throughout this paper that $f$ is square-free. Otherwise, we
can replace $f$ by $f/gcd(f.f')$ which is square-free with the same
Galois group. Thus, if we label the $n$ distinct zeroes of $f$, we can
consider the Galois group as a subgroup of the symmetric group $S_n$.
Notice that this subgroup is determined only up to conjugacy (as the
labeling of the zeroes of $f$ is arbitrary).  Since every subgroup of
$S_n$ has a generator set of size $n-1$ (c.f.\ 
\cite{sims70computational} and \cite{luks93permutation}), we can
specify the Galois group in size polynomial in $n$. By computing the
Galois group of a polynomial $f$ we mean finding a small generator set
(polynomial in $n$) for it as a subgroup of $S_n$.

We now state Landau's result on computing the Galois group of a
polynomial $f$. Its worst case running time is exponential in
$\size(f)$.

\begin{theorem}[\cite{landau:1984:galois}]\label{splitfind}
  Given a polynomial $f\in F[x]$, where the number field $F$ is given
  as a vector space over $\Q$, the Galois group $G$ of $f$ over $F$
  can be computed in time polynomial in $|G|$ and $\size(f)$.
\end{theorem}

The extended abstract is organized as follows: In
Sect.~\ref{sect:cheb} we explain the Chebotarev density theorem in a
form that is useful to us. In Sect.~\ref{ordersection} we give a
polynomial time algorithm making a single query to $\numP$ to compute
the order of the Galois group of a polynomial $f(x) \in \Q$. In
Sect.~\ref{sol:order} we show that if the polynomial is solvable by
radicals the order of its Galois group can be computed by a randomized
algorithm with an $\NP$ oracle.  Finally in Sect.~\ref{sect:abel} we
show that if the Galois group of $f$ is abelian then it can be
computed by a randomized polynomial time algorithm. For the
definitions of various complexity classes the reader can consult any
complexity theory text like \cite{BDG}.



\section{Chebotarev Density theorem}\label{sect:cheb} 

The main tool in the proofs of our complexity results is the
Chebotarev density theorem. In this section we explain the theorem
statement and also state it in a form that is suitable for our
applications.

Let $L$ be a Galois number field and $O_L$ be the ring of algebraic
integers in $L$. Let $n = [L : \Q]$ be the degree of $L$. For any
prime $p \in \Q$ consider the principal ideal $pO_L$ generated by $p$
(which we denote by $p$).  The ideal $p$ factorizes in $O_L$ as $p =
\Ideal{p}_1^{e} \Ideal{p}_2^{e} \ldots \Ideal{p}_g^{e}$ for some
positive integer $e$. For each $i$, $O_L/\Ideal{p}_i$ is a finite
field extension of $\mathbb{F}_p$ with $p^{f}$ elements for some
positive integer $f$. Furthermore $efg=n$.

The prime $p$ is said to be \emph{ramified} in $L$ if $e>1$ and
\emph{unramified} otherwise. It is a basic fact about number fields
that a prime $p$ is ramified in $L$ if and only if $p$ divides the
discriminant of $L$ (see \cite[Theorem 1, pg. 238]{ribenboim:1999}).

Let $p$ be an unramified prime with factorization $p =
\Ideal{p}_1\Ideal{p}_2 \ldots \Ideal{p}_g$ in $O_L$. Corresponding to
the Frobenius automorphism of the finite field $O_L/\Ideal{p}_i$,
there is an element denoted $\Frob{L/\Q}{\Ideal{p}_i}$ in the Galois
group $G=\Gal{L/\Q}$ known as the Frobenius element of $\Ideal{p}_i$,
for $i=1, 2,\ldots,g$. Furthermore, it is known that the set
\[
\Artin{L/\Q}{p} = \left\{ \Frob{L/\Q}{\Ideal{p}} : \Ideal{p} | p
\right\}
\]
is a conjugacy class in the Galois group $G$. For any conjugacy class
$C$ of $G$ let $\pi_C(x)$ be the number of unramified primes less that
$x$ such that $\Artin{L/\Q}{p} = C$. We have the following theorem:

\begin{theorem}[Chebotarev's density theorem]\label{chebo}
  Let $L/\Q$ be a Galois extension and $G=\Gal{L/\Q}$ be its Galois
  group. Then for every conjugacy class $C$ of $G$, $\pi_C(x)$
  converges to $\frac{|C|}{|G|}.\frac{x}{\ln x}$ as $x \to \infty$.
\end{theorem}

In order to apply the above theorem in a complexity-theoretic context,
we need the following effective version due to Lagarias and Odlyzko
\cite{lagarias:1977:effective} proved assuming the GRH.

\begin{theorem}\label{eff:chebo}
  Let $L/\Q$ be a Galois extension and $G=\Gal{L/\Q}$ be its Galois
  group. If the GRH is true then there is an absolute constant $x_0$
  such that for all $x > x_0$:
  \[
  \left|\pi_C(x) - \frac{|C|}{|G|} \frac{x}{\ln x} \right| \leq  \frac{|C|}{|G|}
    x^{1/2} \ln{d_L} + x^{1/2}\ln{x}. |G|.
  \]
\end{theorem}

An unramified prime $p$ such that $\Artin{L/\Q}{p}=\{1\}$ is called a
\emph{split prime}. By definition, $\pi_1(x)$ denotes the number of
split primes $p\leq x$.

\begin{corollary}\label{eff:split:primes}
  Let $G = \Gal{L/\Q}$ for a Galois extension $L/\Q$. If the GRH is
  true then there is an absolute constant $x_0$ such that for all $x >
  x_0$:
  \[
  \left|\pi_1(x) - \frac{1}{|G|} \frac{x}{\ln x} \right| \leq  \frac{1}{|G|}
    x^{1/2} \ln{d_L} + x^{1/2}\ln{x}. |G|.
  \]
\end{corollary}
%\section{newsection}

\section{Computing the order of Galois Groups}\label{ordersection}

Let $f(x) \in \Z[x]$ be a monic polynomial of degree $n$ without
multiple roots and let $L$ denote the splitting field of $f$. Suppose
$\{\alpha_1,\alpha_2,\ldots,\alpha_n\}$ is the set of roots of $f$.

As mentioned before, the Galois group $G=\Gal{L/\Q}$ can be seen as a
subgroup of $S_n$. Each $\sigma\in G$, when considered as a
permutation in $S_n$, can be expressed as a product of disjoint
cycles.  Looking at the lengths of these cycles we get the \emph{cycle
  pattern} $\angle{m_1,m_2,\ldots,m_n}$ of $\sigma$, where $m_i$ is
the number of cycles of length $i$, $1\leq i\leq n$. We have
$n=\sum_{i=1}^n m_i$.

If $p$ is a prime such that $p \nmid d(f)$, we can factorize $f=
g_1 g_2\ldots g_s$ into its distinct irreducible factors $g_i$ over
$\F_p$. Looking at the degrees of these irreducible factors we get the
\emph{decomposition pattern} $\angle{m_1,m_2,\ldots,m_n}$ of $f (\mod
p)$, where $m_i$ is the number of irreducible factors of degree $i$.

We now state an interesting fact from Galois theory (see \cite[page
198]{waerden:1991} and \cite[Theorem 2.9, Chap. VII]{lang:algebra}).

\begin{theorem}\label{waerden}
  Let $f(x) \in \Z[x]$ be a monic polynomial of degree $n$ such that
  $d(f) \neq 0$, and let $L$ denote its splitting field. Let $G =
  \Gal{L/\Q}$. Let $p$ be a prime such that $p \nmid d(f)$. Then
  there is a conjugacy class $C$ of $G$ such that for each $\sigma \in
  C$ the cycle pattern of $\sigma$ is the same as the decomposition
  pattern of $f$ factorized over $\F_p$.  Furthermore, if
  $\{\alpha_1,\alpha_2,\ldots,\alpha_n\}$ are the $n$ roots of $f$ in
  its splitting field and if $\F_{p^m}$ is the extension of $\F_p$
  where $f\ (\textrm{mod } p)$ splits then there is an ordering of the
  roots $\{\alpha'_1,\alpha'_2,\ldots,\alpha'_n\}$ of $f$ in
  $\F_{p^m}$ such that for all indices $k$ and $l$,
  $\sigma(\alpha_k)=\alpha_l$ if and only if the Frobenius
  automorphism $x\mapsto x^p$ of $\F_{p^m}$ maps $\alpha'_k$ to
  $\alpha'_l$.
\end{theorem}

For any prime $p$ that divides the order of the Galois group there is
an element whose order is divisible by $p$. This can happen only if
there is a prime $q$ such that the decomposition pattern of $f\ 
(\textrm{mod } q)$ contains only the integers $p$ and $1$ (using
Theorem~\ref{waerden}). Furthermore using the effective Chebotarev
density theorem (Theorem~\ref{eff:chebo}) we can show that there is a
$q$ with size $size(f)^{O(1)}$ satisfying the above property. So to
check whether $p$ divides the order of the Galois group we guess such
a $q$. This leads to the following theorem.
%
%\begin{lemma}\label{conjugacy}
%  Let $f$ be a degree-$n$ polynomial with $d(f)\neq 0$ and let
%  $\overline{n}=\angle{m_1,m_2,\ldots,m_n}$ be a partition of $n$.  If
%  the GRH is true then
%  \[
%  \left| \pi_{\overline{n}}(x) - \frac{|G_{\overline{n}}|}{|G|}
%\frac{x}{\ln x} \right| \leq
%\frac{|G_{\overline{n}}|}{|G|} x^{1/2}\ln d_L + x^{1/2}\ln x|G|^2,
%  \]
%  where $L$ is the splitting field of $f$.
%\end{lemma}
%
%The above theorem is easily proved by noting that $G_{\overline{n}}$
%is a union of conjugacy classes of $G$, and by applying
%Theorem~\ref{eff:chebo} for each conjugacy class contained in
%$G_{\overline{n}}$. When we add up the inequalities for each conjugacy
%class we obtain the theorem.  Notice that we get $|G|^2$ in the second
%term as an upper bound for $|G||G_{\overline{n}}|$. We now have the following
%theorem:

\begin{theorem}\label{primes-np}
  Assuming GRH, the following problem is in $\NP$: Given a prime
  $p\leq n$, and a monic polynomial $f\in\Z[x]$ with $d(f)\neq 0$ as
  input, test if $p$ divides the order of the Galois group of $f$. As
  a consequence, the set of prime factors of $|\Gal{\Q_f/\Q}|$ can be
  computed in $\p^{\NP}$.
\end{theorem}

%\begin{proof}
%  Let $G$ denote the Galois group of $f$ and $s$ denote $size(f)$. Let
%  $X_p$ denote the set of elements of $G$ of order $p$.  Then $X_p$ is
%  non-empty if and only if $p$ divides $|G|$.  Furthermore, $X_p$ is a
%  union of conjugacy classes of $G$.  Consider the set $Y_x=\{$prime
%  $q\mid q\leq x,$ and $f$ factorizes in $\F_q$ into distinct
%  irreducible factors of degrees $1$ or $p$ with at least one degree
%  $p$ factor $\}$, for any positive integer $x$. Applying
%  Lemma~\ref{conjugacy}, we can see that for $x \geq (n+1)!^6 s^4$ we
%  have $Y_x$ is non-empty iff $X_p$ is nonempty. Now, to test if $p$
%  divides $|G|$, the NP procedure can first guess a prime $q\leq
%  (n+1)!^6 s^4$. To verify that $q\in Y_x$, the procedure next guesses
%  and verifies the factorization of $f$ in $\F_q$, and then checks
%  that each irreducible factor if of degree 1 or $p$ and there is at
%  least one degree $p$ factor.
%  
%  Since all prime factors of $|G|$ are bounded by $n$, using the above
%  NP procedure as oracle we can find all the prime factors of $|G|$ in
%  polynomial time.
%\end{proof}

Now for the main result of this section.

\begin{theorem}\label{order}
  Assuming GRH, the order of the Galois group of a monic polynomial $f
  \in \Z[x]$ can be computed in $\p^{\numP}$.
\end{theorem}
The algorithm first count the number of split primes (with a certain
exponentially small error) less than a suitably large $x$ ($size(x) =
size(f)^{O(1)}$) using a single $\numP$ query.  The order of the
Galois group is the nearest integer to
$\frac{1}{\pi_1(x)}\frac{x}{\ln x}$ which can be computed in
polynomial time.

To the best of our knowledge, this is the first polynomial-space
bounded algorithm for the problem.  Next we consider the approximate
counting problem.

\begin{definition}
  A randomized algorithm ${\cal A}$ is an $r$-approximation algorithm
  for a $\numP$ function $f$ with error probability $\delta <
  \frac{1}{2}$ if for all $x \in \{ 0,1\}^*$:
  \[
  Prob_{y}\left[|1 -\frac{{\cal A}(x,y)}{f(x)}|\leq r(|x|)\right]\geq 1-\delta,
  \]
  where $y$ is a uniformly chosen random string used by the algorithm
  ${\cal A}$ on input $x$.
\end{definition}

Stockmeyer \cite{stockmeyer:1985:approx} showed that for any $\numP$
function there is a $n^{-O(1)}$-approximation $\BPP^{\NP}$ algorithm.
We can use Stockmeyer's result to approximate $\pi_1(x)$ within an
inverse polynomial error and use this approximation instead. This
yields the algorithm in the following theorem.

\begin{theorem}\label{approx:G}
  Let $f(x)\in \Z[x]$ be a degree $n$ polynomial, $G$ be its Galois
  group, and $s$ denote $size(f)$. For any constant $c>0$ there is a
  $\BPP^{\NP}$ algorithm that computes an approximation $A$ of $|G|$
  such that
  \[
  \left(1 - \frac{1}{s^c}\right) A \leq |G| \leq \left(1 +
    \frac{1}{s^c} \right) A.
  \]
  with probability greater than $\frac{2}{3}$.
\end{theorem}

We now derive a useful lemma as an immediate consequence of the above
result.

\begin{lemma}\label{star}
  Let $f$ and $g$ be monic polynomials in $\Z[x]$ with nonzero
  discriminant. Suppose the splitting field $\Q_g$ of $g$ is contained
  in $\Q_f$ of $f$ and $[\Q_f:\Q_g]$ is a prime power $p^l$. There is
  a $\BPP^{\NP}$ algorithm that computes $[\Q_f:\Q_g]$ exactly,
  assuming that $|\Gal{\Q_g/\Q}|$ is already computed.
\end{lemma}
%
%\begin{proof}
%  To see this it suffices to note that as $[\Q_f:\Q_g]$ is a prime
%  power of a small prime $p \leq n=\deg f$, if we approximate
%  $[\Q_f:\Q_g]$ using the $\BPP^{\NP}$ algorithm of
%  Theorem~\ref{approx:G} to within an inverse polynomial fraction, we
%  will compute $[\Q_f:\Q_g]$ exactly. Such an approximation can be
%  computed by first computing $[\Q_f : \Q]$ approximately and dividing
%  it by $[\Q_g : \Q]$, which is already computed by assumption.
%\end{proof}

\section{Computing the order of solvable Galois Groups}\label{sol:order}

In this section we show that if the Galois group $G$ of $f\in\Z[x]$ is
\emph{solvable} then $|G|$ can be computed exactly in $\BPP^{\NP}$,
assuming GRH. In fact, we show that for solvable Galois groups,
finding $|G|$ is polynomial-time reducible to approximating $|G|$.

To begin with we need a test for solvability by radicals.  A naive
application of Galois' theorem gives an exponential time algorithm
(using Theorem.~\ref{splitfind}). An important breakthrough was
achieved by Landau and Miller when they gave a deterministic
polynomial time algorithm to check whether a polynomial is solvable by
radicals without actually computing the Galois group (see.
\cite{landau85solvability}). We make use of results from
\cite{landau85solvability}. We begin by recalling some definitions.

A group $G$ is said to be \emph{solvable} if there is a
\emph{composition series} of $G$, $G = G_0 \rhd G_1 \rhd \ldots \rhd
G_t = {1}$ such that $G_i/G_{i+1}$ is a cyclic group of prime order.
Throughout this section by composition series we mean such a
composition series.
  
A Galois extension $K/F$ is said to be \emph{solvable} if $\Gal{K/F}$
is a solvable group. Let $G = G_0 \rhd G_1 \rhd \ldots \rhd G_t = {1}$
be a composition series of $G$. There is a corresponding tower of
fields $F = E_0 \subseteq E_1 \subseteq \ldots \subseteq E_t = K$ such
that $\Gal{K/E_i} = G_i$. Moreover if $K/F$ is Galois then by the
fundamental theorem of Galois, since $G_i \rhd G_{i+1}$, the extension
$E_{i+1}/E_i$ is Galois.
  
At this point we recall some permutation group theory (c.f.
\cite{wielandt64finite}): Let $G$ be a subgroup of $S_n$ acting on a
set $\Omega = \{1,2,\ldots,n\}$ of $n$ elements. $G$ is said to be
\emph{transitive} if for every pair of distinct elements $i,
j\in\Omega$, there is a $\sigma\in G$ such that $\sigma$ maps $i$ to
$j$, written as $i^{\sigma}=j$. A \emph{block} is a subset
$B\subseteq\Omega$ such that for every $\sigma\in G$ either
$B^{\sigma}=B$ or $B^{\sigma}\cap B=\emptyset$.  If $G$ is transitive
then under $G$-action blocks are mapped to blocks, so that starting
with a block $B_1\subseteq\Omega$ we get a \emph{complete block
  system} $\{B_1,B_2 ,\ldots, B_s\}$ which is a partition of $\Omega$.
Notice that singleton sets and $\Omega$ are blocks for any permutation
group. These are the \emph{trivial} blocks. A transitive group $G$ is
\emph{primitive} if it has only trivial blocks. Otherwise it is called
\emph{imprimitive}. A \emph{minimal block} of an imprimitive group is
a nontrivial block of least cardinality. The corresponding block
system is a \emph{minimal block system}.
 
The following result about solvable primitive permutation groups
\cite{palfy:1982:primitive} has been used to show polynomial time
bounds for several permutation group algorithms
\cite{luks93permutation}.

\begin{theorem}[P\'alfy's bound]{\rm\cite{palfy:1982:primitive}}
  If $G<S_n$ is a solvable primitive group then $|G|\leq n^{3.25}$.
\end{theorem}

Let $f(x) \in \Z[x]$ be a monic irreducible polynomial and let $G$ be
the Galois group $\Gal{\Q_f/\Q}$ which acts transitively on the set of
roots $\Omega = \{ \alpha_1,\alpha_2,\ldots,\alpha_n\}$ of $f$. Let
$\{ B_1,B_2 ,\ldots, B_s\}$ be the minimal block system of $\Omega$
under the action of $G$ and $H$ be the subgroup of $G$ that setwise
stabilizes all the blocks: i.e.\ elements of $H$ map $B_i$ to $B_i$
for each $i$. Let $B_1 = \{ \alpha_1,\alpha_2, \ldots, \alpha_k\}$,
where $k =n/s$. Consider the polynomial $p(x) = \prod_{i=1}^k (x -
\alpha_i) = \sum_{i=0}^k \delta_i x^i$.

In \cite{landau85solvability} it is shown that $p(x) \in
\Q(\alpha_1)[x]$ and there is a polynomial time deterministic
algorithm to find $p(x)$: the algorithm computes each coefficient
$\delta_i$ as a polynomial $p_i(\alpha_1)$ with rational coefficients.
In polynomial time we can compute a primitive element $\beta_1$ of
$\Q(\delta_0,\delta_1,\ldots,\delta_k)$ \cite{landau85solvability} so
that $\Q(\beta_1) = \Q(\delta_0,\delta_1,\ldots,\delta_k)$. Let $g(x)
\in \Z[x]$ be the minimal polynomial of $\beta_1$. In the following
theorem we recall some results from \cite{landau85solvability},
suitably rephrased.

\begin{theorem}{\hfill{~}}\label{g:theorem}
  \begin{enumerate}
  \item The degree of $g(x)$ is $s$.
  \item $H = \Gal{\Q_f/\Q_g}$ and $\Gal{\Q_g/\Q} = G/H$.
  \item The Galois group $\Gal{\Q(B_1)/\Q(\beta_1)}$ acts primitively
    on $B_1$.
  \end{enumerate}
\end{theorem}

Let $\Gal{\Q(B_1)/\Q(\beta)} = G^{B_1} = G_0 \rhd G_1 \rhd \ldots \rhd
G_t = {1}$ be a composition series of the solvable group $G^{B_1}$ and
let $\Q(\beta_1) = K_0 \subseteq K_1 \subseteq \ldots \subseteq K_t =
\Q(B_1)$ be the corresponding tower of subfields of the extension
$\Q(B_1)/\Q(\beta_1)$. Since $K_{i+1}/K_i$ is an extension of prime
degree for each $i$ we have the following proposition.

\begin{proposition}\label{minimalextn}
  For all $0 \leq i < t$ if $K^\prime$ be any field such that
  $K_{i}\subseteq K^\prime \subseteq K_{i+1}$ then either $K^\prime =
  K_i$ or $K^\prime = K_{i+1}$.
\end{proposition}

For each field $K_j$ in the above tower, let $\theta_j$ be a primitive
element, $0 \leq j \leq t$. I.e. $\Q(\theta_j) = K_j$ for each $j$.
Let $h_j(x) \in K_{j-1}[x]$ be the minimal polynomial of $\theta_{j}$
over $K_{j-1}$. We can consider $h_j(x)$ as $h_j(x,\theta_{j-1})$, a
polynomial over $\Q$ in the indeterminate $x$ and the algebraic number
$\theta_{j-1}$ as parameter. As before let $G=\cup_{i=1}^s H\sigma_i$.
For each field $K_j$ let $K_{ij}$ be the conjugate field under the
action of $\sigma_i$. More precisely, let $K_{ij} = K_j^{\sigma_i}$
and $\theta_{ij} = \theta_j^{\sigma_i}$.  We have the following
proposition which follows from the fact that $\sigma_i$ is a field
isomorphism which maps the extension $\Q(B_1)/\Q(\beta_1)$ to
$\Q(B_i)/\Q(\beta_i)$, for each $i$.

\begin{proposition}{\hfill{~}}
  \begin{enumerate}
  \item $K_{i0} \subseteq K_{i1} \subseteq \ldots \subseteq K_{it}$
    forms a tower of fields of the extension $\Q(B_i)/\Q(\beta_i)$
    corresponding to the composition series of
    $\Gal{\Q(B_i)/\Q(\beta_i)}$.
  \item $\Gal{K_{it}/K_{ij}} = \sigma_i^{-1} G_j \sigma_i$.
  \item $K_{ij} = \Q(\theta_{ij})$, where $\theta_{ij} =
    \theta_j^{\sigma_i}$.
  \item The minimal polynomial of $\theta_{ij}$ over the field
    $K_{ij-1}$ is $h_{ij}(x) = h_j(x,\theta_{ij-1})$.
  \end{enumerate}
\end{proposition}

For each $i$, let $\overline{h}_i(x)$ denote the minimal polynomial of
$\theta_i$ over $\Q$ and let $n_i$ be its degree. We have the
following lemma:

\begin{lemma}
  Let $n_i = deg(\overline{h}_i)$ then $n_0 = [\Q(\beta_1):\Q]$ and
  $n_i = p_i n_{i-1}$, where $[K_i:K_{i-1}]=p_i$ for each $i$.
%  \item If $C_i$ be the set of all conjugates of $\theta_i$ then 
%    $\overline{h}_{i+1}(x) = \prod_{\theta \in C_i} h_{i+1}(x,\theta)$.
%  \item $\Q_{\overline{h}_i} \subseteq \Q_{\overline{h}_{i+1}}$, and
%    if $K$ is a Galois number field such that $\Q_{\overline{h}_i}
%    \subseteq K \subseteq \Q_{\overline{h}_{i+1}}$ then either $K =
%   \Q_{\overline{h}_i}$ or $K = \Q_{\overline{h}_{i+1}}$.
\end{lemma}

Let $E_i=\Q_{\overline{h}_i}$, $0\leq i\leq t$. Notice that $\Q_f =
E_t$ and $\Q_g = E_0$. We have the following theorem:

\begin{theorem}\label{main:theorem}
  Let $p_i$ be the order of $G_i/G_{i-1}$.  For every $i$ there is a
  $l_i$ such that $\Gal{E_i/E_{i-1}}$ is an abelian group of order
  $p_i^{l_i}$. Furthermore $\Gal{E_i/E_{i-1}}$ is an elementary
  abelian $p_i$-group.
\end{theorem}

Suppose we know $[\Q_g : \Q]$. Using Lemma~\ref{star} we can compute
$[\Q_f:\Q]$ by finding $[E_i :\Q]$ for each $1 \leq i \leq t$ starting
from $i = 1$.  We will find $[\Q_g :\Q]$ recursively. It is also easy
to generalize this algorithm for reducible polynomials $f(x) \in
\Q[x]$.  This gives the following theorem:

\begin{theorem}\label{second}
  Assuming the GRH, there is a $\BPP^{\NP}$ procedure that takes as
  input a monic polynomial $f\in\Z[x]$ such that $d(f)\neq 0$,
  and computes $|\Gal{\Q_f/\Q}|$ exactly when $\Gal{\Q_f/\Q}$ is
  solvable.
\end{theorem}

\section{Finding the Galois group of an abelian extension}\label{sect:abel}

Let $f$ be a polynomial over $\Z[x]$ such that $\Gal{\Q_f/\Q}$ is
abelian. In this section we give a polynomial-time randomized
algorithm that computes the Galois group (as a set of generators) with
constant success probability.

Suppose $f\in\Z[x]$ is irreducible of degree $n$ with Galois group
$G$. Since $G$ is a transitive subgroup of $S_n$, if $G$ is abelian
then $|G|=n$. Thus, given an irreducible $f\in\Z[x]$, the algorithm of
Theorem~\ref{splitfind} gives a $(\size(f))^{O(1)}$ algorithm for
testing if its Galois group is abelian, and if so, finding the group
explicitly. On the other hand, when $f$ is reducible with abelian
Galois group, no polynomial time algorithm is known for computing the
Galois group (c.f.\ Lenstra~\cite{lenstra:1992:algorithm}). However,
for any polynomial $f$ testing if its Galois group is abelian can be
done in polynomial time: we only need to test if the Galois group of
each irreducible factors of $f$ is abelian.

Let $f$ be a polynomial over $\Z[x]$ such that $\Gal{\Q_f/\Q}$ is
abelian. Let $f = f_1 f_2 \ldots f_t$ be its factorization into
irreducible factors $f_i$. Notice that if $\Gal{\Q_f/\Q}$ is abelian
then $\Gal{\Q_{f_i}/\Q}$ is abelian for each $i$. Consequently, each
$f_i$ is a primitive polynomial. Let $G = \Gal{\Q_f/\Q}$ and let $G_i
= \Gal{\Q_{f_i}/\Q}$ for each $i$.  Notice that $G \leq G_1 \times G_2
\times \ldots G_t$.

Let $n_i$ be the degree of $f_i$. Since each $f_i$ is a primitive
polynomial, $|G_i| = n_i$. Let $\theta_i$ be any root of $f_i$, $1\leq
i\leq t$. Then, $\Q_{f_i}=\Q(\theta_i)$ for each $i$. Factorizing
$f_i$ in $\Q(\theta_i)$, we can express the other roots of $f_i$ as
$A_{ij}(\theta_i)$, where $A_{ij}(x)$ are all polynomials of degree at
most $n_i$, $1\leq j\leq n_i$. We can efficiently find these
polynomials $A_{ij}(x)$ for $1 \leq i \leq t$, $1\leq j\leq n_i$. Thus
we can write $f_i(x) = \prod_{j=1}^{n_i} \left( x - A_{ij}(\theta_i)
\right)$, where $\theta_i$ is one of the roots of $f_i$. We have the
following lemma:

\begin{lemma}\label{abel}
  Let $\theta$ be any root of $f_i$ and let $A_{ij}$ be polynomials of
  degree less than $\deg(f_i)$ such that $f_i(x) = \prod_{j=1}^{n_i}
  ( x - A_{ij}(\theta))$. Then for $1 \leq j < deg(f_i)$,
  we have $A_{ij}(A_{ik}(\theta)) =
  A_{ik}(A_{ij}(\theta))$. Furthermore, for every $\sigma
  \in G_i$ there is an index $k, 1\leq k\leq n_i$ such that for any
  root $\eta$ of $f_i(x)$ we have $\sigma(\eta) = A_{ik}(\eta)$.
\end{lemma}

%\begin{proof}
%  All the roots of the polynomial $f_i$ are given by $\theta_k =
%  A_{ik}(\theta), 1\leq k\leq n_i$.  Since $f_i$ is irreducible, for
%  each $k$ there is a element of $G_i$ that maps $\theta$ to
%  $\theta_k$. Let $\sigma_k$ be the element of $G_i$ that maps
%  $\theta$ to $\theta_k = A_{ik}(\theta)$. We have
%  
%  $A_{ij}(A_{ik}(\theta)) = \sigma_k (A_{ij}(\theta))= \sigma_k
%  \sigma_j(\theta) =\sigma_j \sigma_k(\theta)$, because $G_i$ is
%  abelian. But, $\sigma_j \sigma_k(\theta) =
%  A_{ik}(A_{ij}(\theta))$. Thus, $A_{ij}(A_{ik}(\theta)) =
%    A_{ik}(A_{ij}(\theta))$.
%    
%    Now, consider any root $\eta$ of $f_i$. There is a $j$ such that $
%    \eta = \theta_j=A_{ij}(\theta)$.  Applying the above identity,
%    notice that $\sigma_k$ maps $\eta$ to
%    $A_{ij}(A_{ik}(\theta)) =
%    A_{ik}(A_{ij}(\theta)) = A_{ik}(\eta)$.
%\end{proof}

{From} the above lemma it also follows that for each $i, 1\leq i\leq
t$, the polynomials $A_{ij}$, $1\leq j\leq n_i$ are independent of the
choice of the root $\theta$ of $f_i$ because the Galois group is
abelian.

Now, let $\sigma_{ij}$ denote the unique automorphism of $\Q_{f_i}$
that maps $\theta$ to $A_{ij}(\theta)$ for every root $\theta$ of
$f_i$. Since $G \leq G_1 \times G_2 \times \ldots \times G_t$, any
element $\sigma\in G$ is a $t$-tuple $\sigma =
\angle{\sigma_{1j_1},\sigma_{2j_2},\ldots,\sigma_{tj_t}}$, for indices
$j_1,j_2,\ldots,j_t$. We will apply the Chebotarev density theorem to
determine a generator set for $G$.

Let $q$ be a prime such that $q \nmid d(f)$ and $\F_{q^m}$ be the
extension of $\F_q$ where $f$ splits. Observe that since $G$ is
abelian every conjugacy class of $G$ is a singleton set. Let
$\pi_g(x)$ denote the number of primes $p \leq x$ whose Frobenius
corresponds to $g$. By Theorem~\ref{chebo} $\pi_g(x)$ converges to
$\frac{x}{(\ln x)|G|}$. Furthermore using Theorem~\ref{eff:chebo} we
can show that for a random prime $p \leq x$, the probability that the
Frobenius corresponding to  $p$ is $g$ lies in the range
$\left(\frac{1}{|G|} - \epsilon, \frac{1}{|G|} + \epsilon \right)$,
$\epsilon = \frac{1}{x^{O(1)}}$.
  
Next, fix $i$ and let $\{\alpha_1,\alpha_2,\ldots,\alpha_{n_i}\}$ be
the roots of $f_i$.  By Theorem~\ref{waerden}, there is an ordering
$\{\overline{\alpha}_1,\overline{\alpha}_2,\ldots,\overline{\alpha}_{n_i}\}$
of the roots of $f_i$ in $\F_{q^m}$ such that the Frobenius
automorphism $x\mapsto x^q$ maps $\overline{\alpha}_k$ to
$\overline{\alpha}_l$ if and only if the element $g$ (the unique
Frobenius element corresponding to $q$) maps $\alpha_k$ to $\alpha_l$.
If the element
$g=\angle{\sigma_{1j_1},\sigma_{2j_2},\ldots,\sigma_{tj_t}}$ we can
determine $\sigma_{ij_i}$ as follows: find the splitting field
$\F_{q^k}$ of $f_i$. Since $f_i$ is a primitive polynomial, $k \leq
n_i$, thus $\F_{q^k}$ can be found efficiently.\footnote{ In fact $k
  |n_i$ because $k$ is the order of the corresponding Frobenius
  element which is in the Galois group of $f_i$, and the order of the
  Galois group is $n_i$.}  Now, factorize $f_i$ in $\F_{q^k}$.  Pick
any root $\overline{\theta}\in\F_{q^k}$ of $f_i$.  Then
$\overline{\theta}^q=A_{ij}(\overline{\theta})$ for exactly one
polynomial $A_{ij_i}$, which can be found by trying all of them.  This
gives us $\sigma_{ij_i}$. Thus, we can determine $g$ as a $t$-tuple in
polynomial time, in a manner independent of the choice of the root
$\overline{\theta}$ of $f_i$ in $\F_{q^k}$.

We have the following almost uniform polynomial-time sampling
algorithm from the Galois group $G$: Pick primes $p \nmid d(f)$ less
that a suitably large $x$ and recover corresponding Frobenius.  It can
be shown that if we choose $x \geq (n!)^{10}.\size(f)^2$, the
algorithm samples $g\in G$ with probability in the range
\mbox{$\left(\frac{1}{|G|}-\frac{1}{x^{1/4}},\frac{1}{|G|}+\frac{1}{x^{1/4}}\right)$}.
We require the following lemma to complete the proof

\begin{lemma}
  Suppose we have a (almost) uniform sampling procedure ${\cal A}$
  from a subgroup $G$ of $S_n$. Then for every constant $c>0$, there
  is a polynomial-time randomized algorithm with ${\cal A}$ as
  subroutine that outputs a generator set for $G$ with error
  probability bounded by $2^{-n^c}$.
\end{lemma}
%\begin{proof}
%  To see this, let $g_1,g_2,\ldots,g_m$ be a random sample drawn from
%  $G$ using ${\cal A}$, where $m=n^{O(1)}$ will be chosen later. To
%  each $g_i$ associate the $0/1$ random variable $X_i$ which takes the
%  value $0$ if either $\angle{g_1,\ldots,g_{i-1}}=G$ or
%  $g_i\not\in\angle{g_1,\ldots,g_{i-1}}$ and the value $1$ otherwise.
%  Let $p_i=\Prob[\angle{g_1,\ldots,g_{i-1}}=G]$ and $q_i =
%  \Prob[g_i\not\in\angle{g_1,\ldots,g_{i-1}}|\angle{g_1,\ldots,g_{i-1}}
%  \neq G]$. Since $G$ is a group and ${\cal A}$ is a uniform sampler
%  from $G$, clearly $q_i\geq 1/2$. Thus, we have
%  \[
%  \Prob[X_i=0]=p_i +(1-p_i)q_i\geq 1/2+p_i/2\geq 1/2.
%  \]
%
%  Let $X=\sum_i X_i$. Applying Markov's inequality we get that
%  \[
%  \Prob[X\leq 3m/4]\geq 1/3.
%  \]
%  
%  Hence, letting $m=4(\ln n!)$, the set $\{g_1,g_2,\ldots,g_m\}$
%  generates $G$ with probability $1/3$. The success probability can be
%  boosted by suitably increasing the sample size. Notice that we can
%  use the sifting algorithm for permutation groups
%  (c.f~\cite{luks93permutation}) to prune the generator set to
%  $O(n^2)$ size. This completes the proof of the claim.
%\end{proof}

The above lemma implies the the following theorem.
%We have thus proved the following theorem. The algorithm in
%pseudo-code for finding abelian Galois groups is given in the
%appendix.

\begin{theorem}
  There is a randomized polynomial time algorithm for computing a
  generator set for the Galois group of a polynomial $f\in\Z[x]$ if it
  is abelian.
\end{theorem}

%\bibliographystyle{abbrv}
%\bibliography{bibdata}


\begin{thebibliography}{10}

\bibitem{BDG}
J.~L. Balc\'azar, J.~D\'iaz, and J.~Gabarr\'o.
\newblock {\em Structural Complexity I \& II}.
\newblock ETACS monographs on theoretical computer science. Springer-Verlag,
  Berlin, 1988 and 1990.

\bibitem{cohen:1993}
H.~Cohen.
\newblock {\em A Course in Computational Algebraic Number Theory}.
\newblock Springer-Verlag, Berlin, 1993.

\bibitem{lagarias:1977:effective}
J.~C. Lagarias and A.~M. Odlyzko.
\newblock Effective versions of the {Chebotarev} density theorem.
\newblock In A.~{Fr\"ohlich}, editor, {\em Algebraic {Number Fields}}, pages
  409--464. Academic Press, London, 1977.

\bibitem{landau:1905:ineq}
E.~Landau.
\newblock {Sur quelques th\'eor\`emes de M. Petrovitch relatifs aux z\'eros des
  fonctions analytiques}.
\newblock {\em Bulletin de la Soci\'et\'e de France}, 33:251--261, 1905.

\bibitem{landau:1984:galois}
S.~Landau.
\newblock Polynomial time algorithms for galois groups.
\newblock In J.~Fitch, editor, {\em {EUROSAM 84} Proceedings of International
  Symposium on Symbolic and Algebraic Computation}, volume 174 of {\em Lecture
  Notes in Computer Sciences}, pages 225--236. Springer, July 1984.

\bibitem{landau85solvability}
S.~Landau and G.~L. Miller.
\newblock Solvability by radicals is in polynomial time.
\newblock {\em Journal of Computer and System Sciences}, 30:179--208, 1985.

\bibitem{lang:algebra}
S.~Lang.
\newblock {\em Algebra}.
\newblock {Addison-Wesley Publishing Company, Inc}, third edition, 1999.

\bibitem{lenstra:1992:algorithm}
H.~W. {Lenstra Jr.}
\newblock Algorithms in algebraic number theory.
\newblock {\em Bulletin of the American Mathematical Society}, 26(2):211--244,
  April 1992.

\bibitem{luks93permutation}
E.~M. Luks.
\newblock Permutation groups and polynomial time computations.
\newblock {\em DIMACS Series in Discrete Mathematics and Theoretical Computer
  Science}, 11:139--175, 1993.

\bibitem{palfy:1982:primitive}
P.~{P\'alfy}.
\newblock A polynomial bound for the orders of primitive solvable groups.
\newblock {\em Journal of Algebra}, pages 127--137, July 1982.

\bibitem{ribenboim:1999}
P.~Ribenboim.
\newblock {\em Classical theory of algebraic numbers}.
\newblock Universitext. Springer, 1999.

\bibitem{sims70computational}
C.~C. Sims.
\newblock Computational methods in the study of permutation groups.
\newblock {\em Computational problems in Abstract Algebra}, pages 169--183,
  1970.

\bibitem{stockmeyer:1985:approx}
L.~Stockmeyer.
\newblock On approximating algorithms for \#{P}.
\newblock {\em SIAM Journal of Computing}, 14:849--861, 1985.

\bibitem{waerden:1991}
B.~L. van~der Waerden.
\newblock {\em Algebra}, volume~I.
\newblock Springer-Verlag, seventh edition, 1991.

\bibitem{wielandt64finite}
H.~Wielandt.
\newblock {\em Finite Permutation Groups}.
\newblock Academic Press, New York, 1964.

\end{thebibliography}
\end{document}
