\documentclass[11pt]{article}
\bibliographystyle{alpha}

\usepackage{todonotes}

\usepackage{fullpage}
% Uncomment this when debuging labels. This will display labels
% citations etc when they are refered. Which makes some debuggin
% easier. Comment it for the final version.

%\usepackage{showkeys}
\usepackage{latexsym}
\usepackage{amsmath}
\usepackage{amssymb}
\usepackage{amsthm}
\usepackage{graphicx}
\usepackage{hyperref}
\usepackage{cite}


%theorem environments
\newtheorem{theorem}{Theorem}[section]
\newtheorem{theorem*}{Theorem}
\newtheorem{corollary}[theorem]{Corollary}
\newtheorem{lemma}[theorem]{Lemma}
\newtheorem{observation}[theorem]{Observation}
\newtheorem{proposition}[theorem]{Proposition}
\newtheorem{definition}[theorem]{Definition}
\newtheorem{claim}[theorem]{Claim}
\newtheorem{fact}[theorem]{Fact}
\newtheorem{subclaim}[theorem]{Subclaim}
\theoremstyle{definition}\newtheorem{remark}[theorem]{Remark}

% my custom commands
\newcommand{\inparen}[1]{\left(#1\right)}             %\inparen{x+y}  is (x+y)
\newcommand{\inbrace}[1]{\left\{#1\right\}}           %\inbrace{x+y}  is {x+y}
\newcommand{\insquar}[1]{\left[#1\right]}             %\insquar{x+y}  is [x+y]
\newcommand{\inangle}[1]{\left\langle#1\right\rangle} %\inangle{A}    is <A>
\newcommand{\abs}[1]{\left|#1\right|}                 %\abs{x}        is |x|
\newcommand{\norm}[1]{\left\Vert#1\right\Vert}        %\norm{x}       is ||x||
\newcommand{\union}{\cup}
\newcommand{\Union}{\bigcup}
\newcommand{\intersection}{\cap}
\newcommand{\Intersection}{\bigcap}
\newcommand{\setdef}[2]{\inbrace{{#1}\ : \ {#2}}}
\newcommand{\inv}[1]{#1^{-1}}
\newcommand{\pderiv}[2]{\frac{\partial #1}{\partial #2}}
\newcommand{\iso}{\cong}
\newcommand{\bea}[1]{\begin{eqnarray*} #1 \end{eqnarray*}}
\newcommand{\ceil}[1]{\lceil #1 \rceil}
\newcommand{\floor}[1]{\lfloor #1 \rfloor}
\newcommand{\sas}[4]{#1\longrightarrow #2\longrightarrow
#3\longrightarrow #4\longrightarrow #1}

%Dirac
\newcommand{\bra}[1]{\left\langle #1 \right|}
\newcommand{\ket}[1]{\left| #1 \right\rangle}
\newcommand{\braket}[2]{\langle #1|#2 \rangle}
\newcommand{\outerprod}[2]{\ket{#1}\bra{#2}}

%blackboard letters
\newcommand{\F}{\mathbb{F}}
\newcommand{\N}{\mathbb{N}}
\newcommand{\Q}{\mathbb{Q}}
\newcommand{\Z}{\mathbb{Z}}
\newcommand{\R}{\mathbb{R}}
\newcommand{\C}{\mathbb{C}}



\renewcommand{\R}{\mathcal{R}} %these two are very specific to thsi
\renewcommand{\F}{\mathcal{F}} %file. was just lazy

\newcommand{\mr}{\mathcal{M}_\mathcal{R}} %M_r

\title{Fast Integer Multiplication Using Modular Arithmetic \footnote{A preliminary version appeared in the proceedings of the 40th ACM Symposium on Theory of Computing, 2008.}}
\author{Anindya De\thanks{Research done while the author was at the Dept
    of Computer Science and Engineering, IIT Kanpur}\\
  Computer Science Division\\
  University of California at Berkeley\\
  Berkeley, CA 94720, USA\\
  {\tt anindya@eecs.berkeley.edu}\\
  %
  \and
  Piyush P Kurur%
  \thanks{Research supported through Research I Foundation project
    NRNM/CS/20030163},\\ %
  Dept. of Computer Science and Engineering\\%
  Indian Institute of Technology Kanpur\\%
  Kanpur, UP, India, 208016\\
  {\tt ppk@cse.iitk.ac.in}\\%
  \and%
  Chandan Saha\\%
  Dept. of Computer Science and Engineering\\%
  Indian Institute of Technology Kanpur\\%
  Kanpur, UP, India, 208016\\%
  {\tt csaha@cse.iitk.ac.in}%
  \and  Ramprasad Saptharishi%
  \thanks{Research done while visiting IIT Kanpur %
    under Project FLW/DST/CS/20060225}\\%
  Chennai Mathematical Institute\\%
  Plot H1, SIPCOT IT Park\\%
  Padur PO, Siruseri, India, 603103\\%
  {\tt ramprasad@cmi.ac.in}
}

\date{}

\begin{document}
\maketitle
\begin{abstract}
We give an $N\cdot \log N\cdot 2^{O(\log^*N)}$ time algorithm to
multiply two $N$-bit integers that uses modular arithmetic for
intermediate computations instead of arithmetic over complex numbers
as in F\"{u}rer's algorithm, which also has the same and so far the
best known complexity. The previous best algorithm using modular
arithmetic (by Sch{\"{o}}nhage and Strassen) has complexity $O(N \cdot
\log N \cdot \log\log N)$. The advantage of using modular arithmetic
as opposed to complex number arithmetic is that we can completely
evade the task of bounding the truncation error due to finite
approximations of complex numbers, which makes the analysis relatively
simple. Our algorithm is based upon F\"{u}rer's algorithm, but uses
FFT over multivariate polynomials along with an estimate of the least
prime in an arithmetic progression to achieve this improvement in the
modular setting. It can also be viewed as a $p$-adic version of
F\"{u}rer's algorithm.
\end{abstract}

\section{Introduction}

Computing the product of two $N$-bit integers is nearly a ubiquitous
operation in algorithm design. Being a basic arithmetic operation, it
is no surprise that multiplications of integers occur as intermediate
steps of computation in algorithms from every possible domain of
computer science. But seldom do the complexity of such multiplications
influence the overall efficiency of the algorithm as the integers
involved are relatively small in size and the multiplications can
often be implemented as fast hardware operations. However, with the
advent of modern cryptosystems, the study of the bit complexity of
integer multiplication received a significant impetus. Indeed, large
integer multiplication forms the foundation of many modern day
public-key crystosytems, like RSA, El-Gamal and Elliptic Curve
crytosystems. One of the most notable applications is the RSA
cryptosystem, where it is required to multiply two primes that are
hundreds or thousands of bits long. The larger these primes the harder
it is to factor their product, which in turn makes the RSA extremely
secure in practice.

In this paper, our focus is more on the theoretical aspects of integer
multiplication, it being a fundamental problem in its own right. This
is to say, we will be concerned with the asymptotic bit complexity of
mutiplying two $N$-bit integers with little emphasis on optimality in
practice. We begin with a brief account of earlier work on integer
multiplication algorithms.

\subsection{Previous Work}

The naive approach to multiply two $N$-bit integers leads to an
algorithm that uses $O(N^2)$ bit operations. Karatsuba \cite{K63}
showed that some multiplication operations of such an algorithm can be
replaced by less costly addition operations which reduces the overall
running time of the algorithm to $O(N^{\log_23})$ bit
operations. Shortly afterwards, this result was improved by Toom
\cite{T63} who showed that for any $\varepsilon>0$, integer
multiplication can be done in $O(N^{1+\varepsilon})$ time. This led to
the question as to whether the time complexity can be improved further
by replacing the term $O(N^{\epsilon})$ by a poly-logarithmic
factor. In a major breakthrough, Sch\"{o}nhage and
Strassen~\cite{SS71} gave two efficient algorithms for multiplying
integers using fast polynomial multiplication. One of the algorithms
achieved a running time of $O(N\cdot \log N\cdot \log\log N\ldots
2^{O(\log^*N)})$ using arithmetic over complex numbers (approximated
to suitable precisions), while the other used arithmetic modulo
carefully chosen integers to improve the complexity further to
$O(N\cdot \log N\cdot \log\log N)$ bit operations. The modular
algorithm remained the best for a long period of time until a recent
remarkable result by F\"{u}rer \cite{F07} (see also
\cite{F09}). F\"{u}rer gave an algorithm that uses arithmetic over
complex numbers and runs in $N\cdot\log N\cdot 2^{O(\log^\ast N)}$
time. Till date this is the best time complexity known for integer
multiplication and indeed our result is inspired by F\"{u}rer's
algorithm.

Further details on other approaches and enhancements to previous
integer multiplication algorithms can be found in \cite{F09}.

\subsection{The Motivation}

Sch\"{o}nhage and Strassen introduced two seemingly different
approaches to integer multiplication -- using complex and modular
arithmetic. F\"{u}rer's algorithm improves the time complexity in the
complex arithmetic setting by cleverly reducing some costly
multiplications to simple shift operations. However, the algorithm
needs to approximate the complex numbers to certain precisions during
computation. This introduces the added task of bounding the total
truncation errors in the analysis of the algorithm. On the contrary,
in the modular setting the error analysis is virtually absent or
rather more implicit, which in turn simplifies the overall
analysis. In addition, modular arithmetic gives a discrete approach to
a discrete problem like integer multiplication. Therefore, it seems
natural to ask whether we can achieve a similar improvement in time
complexity of this problem in the modular arithmetic setting. In this
work, we answer this question affirmatively. We give an $N\cdot
\log{N}\cdot 2^{O(\log^{*}{N})}$ time algorithm for integer
multiplication using only modular arithmetic, thus matching the
improvement made by F\"{u}rer.

\subsection*{Overview of our result}

As is the case in both Sch\"{o}nhage-Strassen's and F\"{u}rer's
algorithms, we start by reducing the problem to polynomial
multiplication over a ring $\mathcal{R}$ by properly encoding the
given integers. Polynomials can be multiplied efficiently using
Discrete Fourier Transforms (DFT). However, in order that we are able
to use Fast Fourier Transform (FFT), the ring $\mathcal{R}$ should
have some special roots of unity. For instance, to multiply two
polynomials of degree less than $M$ using FFT, we require a
\emph{principal} $2M$-th root of unity (see
Definition~\ref{def-principal-root} for principal roots). One way to
construct such a ring in the modular setting is to consider rings of
the form $\mathcal{R} = \Z/(2^M + 1) \Z$ as in Sch\"{o}nhage and
Strassen's work ~\cite{SS71}. In this case, the element $2$ is a
$2M$-th principal root of unity in $\mathcal{R}$. This approach can be
equivalently viewed as attaching an `artificial' root to the ring of
integers. However, this makes the size of $\mathcal{R}$ equal to $2^M$
and thus a representation of an arbitrary element in $\mathcal{R}$
takes $M$ bits. This means an $N$-bit integer is encoded as a
polynomial of degree $M$ with every coefficient about $M$ bits long,
thereby making $M \approx \sqrt{N}$ as the optimal choice. Indeed, the
choice of such an $\mathcal{R}$ is the basis of Sch\"{o}nhage and
Strassen's modular algorithm in which they reduce multiplication of
$N$-bit integers to multiplication of $\sqrt{N}$-bit integers and
achieve a complexity of $O(N \cdot \log N \cdot \log \log N)$ bit
operations.

Naturally, such rings are a little too expensive in our setting. We
would rather like to find a ring whose size is bounded by some
polynomial in $M$ and which still contains a principal $2M$-th root of
unity. In fact, it is this task of choosing a suitable ring that poses
the primary challenge in adapting F\"{u}rer's algorithm and making it
work in the discrete setting.

We choose the ring to be $\mathcal{R} = \Z/p^c\Z$, for a prime $p$ and
a constant $c$ such that $p^c = \mathsf{poly}(M)$. The ring
$\Z/p^c\Z$, has a principal $2M$-th root of unity if and only if $2M$
divides $p-1$, which means that we need to find a prime $p$ from the
arithmetic progression $\inbrace{1 + i\cdot 2M}_{i>0}$. To make this
search computationally efficient, we also need the degree of the
polynomials, $M$ to be sufficiently small. This we can achieve by
encoding the integers as multivariate polynomials instead of
univariate ones. It turns out that the choice of the ring as
$\mathcal{R} = \Z/p^c\Z$ is still not quite sufficient and needs a
little more refinement. This is explained in Section \ref{sec:ring}.

The use of multivariate polynomial multiplications along with a small
base ring are the main steps where our algorithm differs from earlier
algorithms by Sch\"{o}nhage-Strassen and F\"{u}rer. Towards
understanding the notion of \emph{inner} and \emph{outer} DFT in the
context of multivariate polynomials, we also present a group theoretic
interpretation of DFT. The use of inner
and outer DFT plays a central role in both F\"{u}rer's as well as our
algorithm. Arguing along the line of F\"{u}rer \cite{F07}, we show that
repeated use of efficient computation of inner DFT's using some
special roots of unity in $\mathcal{R}$ reduces the number of
`bad multiplications' (in comparison to Sch\"{o}nhage-Strassen's algorithm)
and makes the overall process efficient, thereby leading to an $N\cdot \log{N}\cdot 2^{O(\log^{*}{N})}$
time algorithm.


\section{The Basic Setup}

\subsection{The Underlying Ring} \label{sec:ring}

Rings of the form $\mathcal{R} = \Z/(2^M + 1) \Z$ have the nice
property that multiplications by powers of $2$, the $2M$-th principal
root of unity, are mere shift operations and are therefore very
efficient. Although by choosing the ring $\mathcal{R} = \Z/p^c\Z$ we
ensure that the ring size is small, it comes with a price:
multiplications by principal roots of unity are no longer just shift
operations. Fortunately, this can be redeemed by working with rings of
the form $\mathcal{R} = \Z[\alpha]/(p^c, \alpha^m + 1)$ for some $m$
whose value will be made precise later. Elements of $\mathcal{R}$ are
thus $m-1$ degree polynomials over $\alpha$ with coefficients from
$\Z/p^c\Z$. By construction, $\alpha$ is a $2m$-th root of unity and
multiplication of any element in $\mathcal{R}$ by any power of
$\alpha$ can be achieved by shift operations --- this property is
crucial in making some multiplications in the FFT less costly (see
Section~\ref{fourier_analysis}).

Given an $N$-bit number $a$, we encode it as a $k$-variate polynomial
over $\mathcal{R}$ with degree in each variable less than $M$. The
parameters $M$ and $m$ are powers of two such that $M^k$ is roughly
$\frac{N}{\log^2N}$ and $m$ is roughly $\log{N}$. The parameter $k$
will ultimately be chosen a constant (see Section
\ref{complexity_section}). We now explain the details of this encoding
process.

\subsection{Encoding Integers into $k$-variate
  Polynomials}\label{encoding_section}

Given an $N$-bit integer $a$, we first break these $N$ bits into $M^k$
blocks of roughly $\frac{N}{M^k}$ bits each. This corresponds to
representing $a$ in base $q = 2^{\frac{N}{M^k}}$.  Let $a = a_0 +
\ldots + a_{M^k-1}q^{M^k - 1}$, where every $a_i < q$. The number $a$
is converted into a polynomial as follows:
\begin{enumerate}
\item Express $i$ in base $M$ as $i = i_1 + i_2M + \cdots +
  i_kM^{k-1}$. \label{base_M_item}
\item Encode each term $a_iq^i$ as the monomial $a_i\cdot
  X_1^{i_1}X_2^{i_1}\cdots X_k^{i_k}$. As a result, the number $a$
  gets converted to the polynomial $\sum_{i=0}^{M^k - 1}
  a_i\cdot X_1^{i_1}\cdots X_k^{i_k}$.
\end{enumerate}

Further, we break each $a_i$ into $\frac{m}{2}$ equal sized blocks
where the number of bits in each block is $u = \frac{2N}{M^k\cdot m}$.
Each coefficient $a_i$ is then encoded as a polynomial in $\alpha$ of
degree less than $\frac{m}{2}$. The polynomials are then padded with
zeroes to stretch their degrees to $m$. Thus, the $N$-bit number $a$
is converted to a $k$-variate polynomial $a(X)$ over
$\Z[\alpha]/(\alpha^m + 1)$.\\

Given integers $a$ and $b$, each of $N$ bits, we encode them as
polynomials $a(X)$ and $b(X)$ and compute the product polynomial. The
product $a\cdot b$ can be recovered by substituting $X_s =
q^{M^{s-1}}$, for $1\leq s\leq k$, and $\alpha = 2^u$ in the
polynomial $a(X)\cdot b(X)$.  The coefficients in the product
polynomial could be as large as $M^k\cdot m\cdot 2^{2u}$ and hence it
is sufficient to do arithmetic modulo $p^c$ where $p^c > 2M^k\cdot
m\cdot 2^{2u}$. Our choice of
the prime $p$ ensures that $c$ is in fact a constant (see
Section~\ref{complexity_section}). We summarize this discussion as a lemma.

\begin{lemma}\label{lem:encoding-time}
  Multiplication of two $N$-bit integers reduces to
  multiplication of two $k$-variate polynomials, with degree in each
  variable bounded by $M$, over the ring $\Z[\alpha]/(p^c,\alpha^m +
  1)$ for a prime $p$ satisfying $p^c > 2M^k\cdot m \cdot 2^{2u}$,
  where $u=\frac{2N}{M^km}$. Furthermore, the reduction can be
  performed in $O(N)$ time.
\end{lemma}


\subsection{Choosing the Prime}\label{prime_section}

The prime $p$ should be chosen such that the ring $\Z/p^c\Z$ has a
\emph{principal} $2M$-th root of unity, which is required for
polynomial multiplication using FFT. A principal root of unity is
defined as follows.

\begin{definition}\label{def-principal-root}
\emph{\textsf{(Principal root of unity)}} An $n$-th root of unity
$\zeta\in \mathcal{R}$ is said to be primitive if it generates a
cyclic group of order $n$ under multiplication. Furthermore, it is
said to be principal if $n$ is coprime to the characteristic of
$\mathcal{R}$ and $\zeta$ satisfies $\sum_{i=0}^{n-1}\zeta^{ij}=0$ for
all $0< j < n$.
\end{definition}

\noindent
In $\Z/p^c\Z$, a $2M$-th root of unity is principal if and only if
$2M\mid p-1$ (see also Section~\ref{Qp_section}). As a result, we need
to choose the prime $p$ from the arithmetic progression $\inbrace{1 +
  i\cdot 2M}_{i> 0}$, which is potentially the main bottleneck of our
approach. We now explain how to circumvent this problem. \\


An upper bound for the least prime in an arithmetic progression is
given by the following theorem by Linnik \cite{L44}:

\begin{theorem}\label{linnik_theorem}
\emph{\textsf{(Linnik)}} There exist absolute constants $\ell$ and $L$
such that for any pair of coprime integers $d$ and $n$, the least
prime $p$ such that $p\equiv d\bmod{n}$ is less than $\ell n^L$.
\end{theorem}

Heath-Brown \cite{B92} showed that the \emph{Linnik constant} $L\leq
5.5$ (a recent work by Xylouris \cite{X09} showed that $L \leq
5.2$). Recall that $M$ is chosen such that $M^k$ is
$O\inparen{\frac{N}{\log^2N}}$. If we choose $k=1$, that is if we use
univariate polynomials to encode integers, then the parameter $M =
O\inparen{\frac{N}{\log^2N}}$. Hence the least prime $p\equiv
1\pmod{2M}$ could be as large as $N^L$. Since all known deterministic
sieving procedures take at least $N^L$ time this is clearly infeasible
(for a randomized approach see Section~\ref{ERH_section}). However, by
choosing a larger $k$ we can ensure that the least prime $p\equiv
1\pmod{2M}$ is $O(N^\varepsilon)$ for some constant $\varepsilon <
1$. Since primality testing is in deterministic polynomial\footnote{a
  subexponential algorithm would suffice} time\cite{AKS04}, we can find the least prime $p\equiv 1 \pmod{2M}$ in $o(N)$
time.


\begin{lemma}\label{prime_time}
If $k$ is any integer greater than $L+1$, then $M^L =
O\inparen{N^{\frac{L}{L+1}}}$ and hence the least prime $p\equiv
1\pmod{2M}$ can be found in $o(N)$ time.
\end{lemma}

\subsubsection*{Choosing the Prime Randomly}\label{ERH_section}

To ensure that the search for a prime $p\equiv 1\pmod{2M}$ does not
affect the overall time complexity of the algorithm, we considered
multivariate polynomials to restrict the value of $M$; an alternative
is to use randomization.

\begin{proposition} \label{prop:randomprime}
Assuming ERH, a prime $p\equiv 1\pmod{2M}$ can be computed by a
randomized algorithm with expected running time $\tilde{O}(\log^3 M)$.
\end{proposition}
\begin{proof}
Titchmarsh \cite{Titchmarsh} (see also Tianxin \cite{Tianxin})
showed, assuming ERH, that the number of primes less than $x$ in the
arithmetic progression $\{ 1 + i \cdot 2M\}_{i > 0}$ is given by,
\begin{equation*}
\pi(x,2M) = \frac{Li(x)}{\varphi(2M)} + O(\sqrt{x} \log x)
\end{equation*}
for $2M \leq \sqrt{x} \cdot (\log x)^{-2}$, where $Li(x) =
\Theta(\frac{x}{\log x})$ and $\varphi$ is the Euler totient
function. In our case, since $M$ is a power of two, $\varphi(2M) = M$,
and hence for $x \geq 4M^2 \cdot \log^6 M$, we have $\pi(x, 2M) =
\Omega\inparen{\frac{x}{M\log x}}$. Therefore, for an $i$ chosen
uniformly randomly in the range $1 \leq i \leq 2M \cdot \log^6 M$, the
probability that $i\cdot 2M + 1$ is a prime is at least $\frac{d}{\log
  x}$ for a constant $d$. Furthermore, primality test of an $O(\log
M)$ bit number can be done in $\tilde{O}(\log^2 M)$ time using
Rabin-Miller primality test \cite{M76, R80}. Hence, with $x = 4M^2
\cdot \log^6 M$, a suitable prime for our algorithm can be found in
expected $\tilde{O}(\log^3 M)$ time.
\end{proof}

\noindent \textbf{Remark - } We prefer to use Linnik's theorem (Theorem
\ref{linnik_theorem}) instead of Proposition \ref{prop:randomprime}
while choosing the prime in the progression $\{ 1 + i \cdot 2M\}_{i > 0}$
so as to make the results in this paper independent of the ERH and the usage
of random bits.

\subsection{Finding the Root of Unity}\label{root_section}

We require a principal $2M$-th root of unity $\rho(\alpha)$ in
$\mathcal{R}$ to compute the Fourier transforms. This root
$\rho(\alpha)$ should also have the property that its
$\inparen{\frac{M}{m}}$-th power is $\alpha$, so as to make some
multiplications in the FFT efficient (see Section
\ref{fourier_analysis}). The root $\rho(\alpha)$ can be computed by
interpolation in a way similar to that in F\"{u}rer's algorithm
\cite[Section 3]{F07}, except that we need a principal $2M$-th root of
unity $\omega$ in $\Z/p^c\Z$ to start with.

To obtain such a root, we first obtain a $2M$-th root of unity
$\omega_1$ in $\Z/p\Z$. A generator $\zeta$ of $\mathbb{F}_p^{\times}$
can be computed by brute force, as $p$ is sufficiently small, and
$\omega_1 = \zeta^{(p-1)/2M}$ is a principal $2M$-th root of unity in $\Z/p\Z$.
A principal $2M$-root of unity
$\omega_1$ must be a root of the polynomial $f(x) = x^M+1$ in $\Z/p\Z$.  Having obtained $\omega_1$,
we use Hensel Lifting \cite[Theorem 2.23]{Zuckerman}.

\begin{lemma}[\textsf{Hensel Lifting}]
  Let $\omega_s$ be a root of $f(x) = x^M + 1$ in $\Z/p^s\Z$. Then
  there exists a unique root $\omega_{s+1}$ in $\Z/p^{s+1}\Z$ such
  that $\omega_{s+1}\equiv \omega_s\pmod{p^s}$ and $f(\omega_{s+1}) =
  0\pmod{p^{s+1}}$. This unique root is given by $\omega_{s+1} =
  \omega_s - \frac{f(\omega_s)}{f'(\omega_s)}$.\end{lemma}

%%% HANDLING TODO PRES 5

\noindent
It is clear from the above lemma that we can compute a $2M$-th root of
unity $\omega = \omega_c$ in $\Z/p^c\Z$. We will need the following
well-known and useful fact about principal roots of unity in any ring.

\begin{lemma}\cite[Lemma 2.1]{F09}\label{lem:Furer-principal-root}
  If $M$ is a power of $2$ and $\omega^{M} = -1$ in an arbitrary ring
  $\mathcal{R}$. If $M$ is relatively prime to the characteristic of
  $\mathcal{R}$, then $\omega$ is a principal $2M$-th root of unity in
  $\mathcal{R}$.
\end{lemma}

\noindent
Hence it follows that the root $\omega$ of $f(x) = x^M + 1$ in
$\Z/p^c\Z$ is a principal $2M$-th root of unity in $\Z/p^c\Z$.
Furthermore, $\zeta^{(p-1)/2M} = \omega_1 \equiv \omega\bmod p$. Since
$\zeta$ is a generator of $\mathbb{F}_p^\times$, different powers of
$\zeta$ must generate the group $\mathbb{F}_p^\times$ and hence must
be distinct modulo $p$. Hence it follows that different powers of
$\omega$ must be distinct modulo $p$ as well. Therefore, the
difference between any two of them is a unit in $\Z/p^c\Z$ and this
makes the following
interpolation feasible in our setting. \\

\paragraph{Finding $\rho(\alpha)$ from $\omega$:} Since $\omega$ is a
principal $2M$-th root of unity, $\gamma = \omega^{\frac{2M}{2m}}$ is
a principal $2m$-th root of unity in $\Z/p^c\Z$. Notice that,
$\alpha^m + 1$ uniquely factorizes as, $\alpha^m + 1 = (\alpha -
\gamma)(\alpha - \gamma^3) \ldots (\alpha - \gamma^{2m-1})$. The
ideals generated by $(\alpha - \gamma^i)$ and $(\alpha - \gamma^j)$
are mutually coprime as $\gamma^i -\gamma^j$ is a unit for $i\neq j$
and is contained in the ideal generated by $(\alpha - \gamma^i)$ and
$(\alpha - \gamma^j)$.  Therefore, using Chinese Remaindering,
$\alpha$ has the direct sum representation $(\gamma, \gamma^3, \ldots,
\gamma^{2m-1})$ in $\mathcal{R}$. Since we require
$\rho(\alpha)^{\frac{2M}{2m}} = \alpha$, it is sufficient to choose a
$\rho(\alpha)$ whose direct sum representation is $(\omega, \omega^3,
\ldots, \omega^{2m-1})$. Now use Lagrange's formula to interpolate
$\rho(\alpha)$ as,
\begin{equation*}
  \rho(\alpha) = \sum_{i=1, \text{ } i \text{ odd}}^{2m-1}{\omega^i \cdot \prod_{j=1, \text{ } j \neq i, \text{ } j \text{ odd}}^{2m-1}{\frac{\alpha - \gamma^j}{\gamma^i - \gamma^j}}}
\end{equation*}
The inverses of the elements $\gamma^i - \gamma^j$ in $\Z/p^c\Z$ can
be easily computed in $\text{poly}(\log p)$ time. It is also clear
that $\rho(\alpha)^M$, in the direct-sum representation, is
$(\omega^M, \omega^{3M}, \cdots, \omega^{(2m -1)M})$ which is the element $-1$
in $\R$. Hence $\rho(\alpha)$ is
a principal $2M$-th root of unity (by Lemma~\ref{lem:Furer-principal-root}).\\

Besides finding a generator $\zeta$ of $\Z/p\Z$ by brute-force (which
can be performed in $O(p)$ time), all other computations can be done
in $\text{poly}(\log p)$ time. We summarize this as a lemma.

\begin{lemma}\label{lem:root-time}
  A principal $2M$-th root of unity $\rho(\alpha)\in \R$ such that
  $\rho(\alpha)^{2M/2m} = \alpha$ can be computed in deterministic
  time $O(p \cdot \mathrm{poly}(\log(p)))$ (which is $o(N)$ if $p =
  o(N)$).
\end{lemma}

% \subsection*{old}

% We require a principal $2M$-th root of unity $\rho(\alpha)$ in
% $\mathcal{R}$ to compute the Fourier transforms. This root
% $\rho(\alpha)$ should also have the property that its
% $\inparen{\frac{M}{m}}$-th power is $\alpha$, so as to make some
% multiplications in the FFT efficient (see Section
% \ref{fourier_analysis}). The root $\rho(\alpha)$ can be computed by
% interpolation in a way similar to that in F\"{u}rer's algorithm
% \cite[Section 3]{F07}, except that we need a principal $2M$-th root of
% unity $\omega$ in $\Z/p^c\Z$ to start with. To obtain such a root, we
% first obtain a $(p-1)$-th root of unity $\zeta$ in $\Z/p^c\Z$ by
% lifting a generator of $\mathbb{F}_p^{\times}$. The
% $\inparen{\frac{p-1}{2M}}$-th power of $\zeta$ gives us the required
% $2M$-th root of unity $\omega$. A generator of $\mathbb{F}_p^{\times}$
% can be computed by brute force, as $p$ is sufficiently small. Having
% obtained a generator, we use Hensel Lifting \cite[Theorem
%   2.23]{Zuckerman}.

% \begin{lemma}[\textsf{Hensel Lifting}]
% Let $\zeta_s$ be a primitive $(p-1)$-th root of unity in
% $\Z/p^s\Z$. Then there exists a unique primitive $(p-1)$-th root of
% unity $\zeta_{s+1}$ in $\Z/p^{s+1}\Z$ such that $\zeta_{s+1}\equiv
% \zeta_s\pmod{p^s}$. This unique root is given by $\zeta_{s+1} =
% \zeta_s - \frac{f(\zeta_s)}{f'(\zeta_s)}$ where $f(x) = x^{p-1} - 1$.
% \end{lemma}

% \noindent It can be shown that the root $\zeta$ in $\Z/p^c\Z$ thus
% obtained by lifting is principal. Furthermore, different powers of
% $\zeta$ are distinct modulo $p$. Therefore, the difference between any
% two of them is a unit in $\Z/p^c\Z$ and this makes the following
% interpolation feasible in our setting.

% \paragraph{Finding $\rho(\alpha)$ from $\omega$:} Since
% $\omega$ is a principal $2M$-th root of unity, $\gamma =
% \omega^{\frac{2M}{2m}}$ is a principal $2m$-th root of unity in
% $\Z/p^c\Z$. Notice that, $\alpha^m + 1$ uniquely factorizes as,
% $\alpha^m + 1 = (\alpha - \gamma)(\alpha - \gamma^3) \ldots (\alpha -
% \gamma^{2m-1})$, and the ideals generated by $(\alpha - \gamma^i)$ in
% $\mathcal{R}$ are mutually coprime as $\gamma^i - \gamma^j$ is a unit
% for $i \neq j$. Therefore, using Chinese Remaindering, $\alpha$ has
% the direct sum representation $(\gamma, \gamma^3, \ldots,
% \gamma^{2m-1})$ in $\mathcal{R}$. Since we require
% $\rho(\alpha)^{\frac{2M}{2m}} = \alpha$, it is sufficient to choose a
% $\rho(\alpha)$ whose direct sum representation is $(\omega, \omega^3,
% \ldots, \omega^{2m-1})$. Now use Lagrange's formula to interpolate
% $\rho(\alpha)$ as,
% \begin{equation*}
% \rho(\alpha) = \sum_{i=1, \text{ } i \text{ odd}}^{2m-1}{\omega^i \cdot \prod_{j=1, \text{ } j \neq i, \text{ } j \text{ odd}}^{2m-1}{\frac{\alpha - \gamma^j}{\gamma^i - \gamma^j}}}
% \end{equation*}
% The inverses of the elements $\gamma^i - \gamma^j$ can be easily computed in $\Z/p^c\Z$.

\section{Fourier Transform}
\subsection{Inner and Outer DFT} \label{sec:inoutDFT}

Suppose that $a(x) \in \mathcal{S}[x]$ is a polynomial of degree less
than $2M$, where $\mathcal{S}$ is a ring containing a $2M$-th principal
root of unity $\rho$. Let us say that we want to compute the
$2M$-point DFT of $a(x)$ using $\rho$ as the root of unity. In other
words, we want to compute the elements $a(1), a(\rho), \ldots,
a(\rho^{2M-1})$ is $\mathcal{S}$. This can be done in two steps.

\paragraph{Step $1$:}
Compute the following polynomials using $\alpha = \rho^{2M/2m}$.
\begin{eqnarray*}
a_0(x) &=& a(x) \mod (x^{2M/2m} - 1) \\
a_1(x) &=& a(x) \mod (x^{2M/2m} - \alpha) \\
			 &\vdots& \\
a_{2m-1}(x) &=& a(x) \mod (x^{2M/2m} - \alpha^{2m-1}),
\end{eqnarray*}
where $\deg(a_j(x)) < \frac{2M}{2m}$ for all $0 \leq j < 2m$.

\paragraph{Step $2$:}
Note that, $a_j(\rho^{k \cdot 2m + j}) = a(\rho^{k \cdot 2m + j})$ for
every $0 \leq j < 2m$ and $0 \leq k < \frac{2M}{2m}$. Therefore, all
we need to do to compute the DFT of $a(x)$ is to evaluate the
polynomials $a_j(x)$ at appropriate powers of $\rho$.\\

\noindent The idea is to show that both Step $1$ and Step $2$ can be
performed by computation of some `smaller' DFTs. Let us see how.

\paragraph{Performing Step $1$:}
The crucial observation here is the following. Fix an integer $\ell$
in the range $[0, \frac{2M}{2m} - 1]$. Then the $\ell^{th}$
coefficients of $a_0(x), a_1(x), \ldots, a_{2m-1}(x)$ are exactly
$e_\ell(1), e_\ell(\alpha), \ldots, e_\ell(\alpha^{2m-1})$,
respectively, where $e_\ell(y)$ is the polynomial,

\begin{equation*}
e_\ell(y) = \sum_{j=0}^{2m-1}{a_{j \cdot \frac{2M}{2m} + \ell} \cdot y^j}.
\end{equation*}
But then, finding $e_\ell(1), e_\ell(\alpha), \ldots,
e_\ell(\alpha^{2m-1})$ is essentially computing the $2m$-point DFT of
$e_\ell(y)$ using $\alpha$ as the $2m^{th}$ root of unity. Therefore,
all we need to do to find $a_0(x), \ldots, a_{2m-1}(x)$ is to compute
the DFTs of $e_\ell(y)$ for all $0 \leq \ell < \frac{2M}{2m}$. These
$\frac{2M}{2m}$ many $2m$-point DFTs are called the \emph{inner} DFTs.

\paragraph{Performing Step $2$:} In order to find $a_j(\rho^{k \cdot
  2m + j})$, for $0 \leq k < \frac{2M}{2m}$ and a fixed $j$, we first
compute the polynomial $\tilde{a}_j(x) = a_j(x \cdot \rho^j)$ followed
by a $\frac{2M}{2m}$-point DFT of $\tilde{a}_j(x)$ using $\rho^{2m}$
as the root of unity. These $2m$ many $\frac{2M}{2m}$-point DFTs ($j$
running from $0$ to $2m - 1$) are called the \emph{outer} DFTs. The
polynomials $\tilde{a}_j(x)$ can be computed by multiplying the
coefficients of $a_j(x)$ by suitable powers of $\rho$. Such
multiplications are termed as \emph{bad} multiplications (as they would result in recursive calls to integer
multiplication). \\

\noindent The above discussion is summarized in the following lemma.

\begin{lemma} \label{lem:inoutDFT} \emph{\textsf{(DFT time = Inner
      DFTs + Bad multiplications + Outer DFTs)}} \\ Time taken to
  compute a $2M$-point DFT over $\mathcal{S}$ is sum of:
\begin{enumerate}
\item Time taken to compute $\frac{2M}{2m}$ many $2m$-point inner DFTs over $\mathcal{S}$ using $\alpha$ as the $2m$-th root of unity.
\item Time to do $2M$ multiplications in $\mathcal{S}$ by powers of
  $\rho$ (bad multiplications).
\item Time taken to compute $2m$ many $\frac{2M}{2m}$-point outer DFTs over $\mathcal{S}$ using $\rho^{2m}$ as the $\frac{2M}{2m}$-th root of unity.
\end{enumerate}
\end{lemma}

\subsection{Analysis of the FFT}\label{fourier_analysis}

We are now ready to analyse the complexity of multiplying the two
$k$-variate polynomials $a(X)$ and $b(X)$ (see Section
\ref{encoding_section}) using Fast Fourier Transform. Treat $a(X)$ and
$b(X)$ as univariate polynomials in variable $X_k$ over the ring
$\mathcal{S} = \mathcal{R}[X_1, \ldots, X_{k-1}]$. We write $a(X)$ and
$b(X)$ as $a(X_k)$ and $b(X_k)$, respectively, where $\deg(a(X_k))$
and $\deg(b(X_k))$ are less than $M$. Multiplication of $a(X)$ and
$b(X)$ can be thought of as multiplication of the univariates $a(X_k)$
and $b(X_k)$ over $\mathcal{S}$. Also note that, the root
$\rho(\alpha)$ (constructed in Section \ref{root_section}) is a
primitive $2M$-th root of unity in $\mathcal{S} \supset
\mathcal{R}$. Denote the multiplication complexity of $a(X_k)$ and
$b(X_k)$ by $\mathcal{F}(2M, k)$.

Multiplication of $a(X_k)$ and $b(X_k)$ using FFT involves computation
of three $2M$-point DFTs over $\mathcal{S}$ and $2M$ pointwise (or
componentwise) multiplications in $\mathcal{S}$. Let $\mathcal{D}(2M,
k)$ be the time taken to compute a $2M$-point DFT over
$\mathcal{S}$. By Lemma \ref{lem:inoutDFT}, the time to compute a DFT
is the sum of the time for the inner DFTs, the bad multiplications and
the outer DFTs. Let us analyse these three terms separately. We will
go by the notation in Section \ref{sec:inoutDFT}, using $\mathcal{S} =
\mathcal{R}[X_1, \ldots, X_{k-1}]$ and $\rho = \rho(\alpha)$.

\paragraph{Inner DFT time:} Computing a $2m$-point DFT requires $2m
\log (2m)$ additions in $\mathcal{S}$ and $m \log (2m)$
multiplications by powers of $\alpha$. The important observation here
is: since $\mathcal{R} = \Z[\alpha]/(p^c, \alpha^m + 1)$,
multiplication by a power of $\alpha$ with an element in $\mathcal{R}$
can be readily computed by simple cyclic shifts (with possible
negations), which takes only $O(m \cdot \log p)$ bit operations. An
element in $\mathcal{S}$ is just a polynomial over $\mathcal{R}$ in
variables $X_1, \ldots, X_{k-1}$, with degree in each variable bounded
by $M$. Hence, multiplication by a power of $\alpha$ with an element
of $\mathcal{S}$ can be done using $\mathcal{N}_{\mathcal{S}} =
O(M^{k-1} \cdot m \cdot \log p)$ bit operations. A total of $m \log
(2m)$ multiplications takes $O(m \log m \cdot
\mathcal{N}_{\mathcal{S}})$ bit operations. It is easy to see that $2m
\log (2m)$ additions in $\mathcal{S}$ also require the same order of
time.

Since there are $\frac{2M}{2m}$ many $2m$-point DFTs, the total time
spent in the inner DFTs is $O(2M \cdot \log m \cdot
\mathcal{N}_{\mathcal{S}})$ bit operations.

\paragraph{Bad multiplication time:}
Suppose that two arbitrary elements in $\mathcal{R}$ can be multiplied
using $\mathcal{M}_{\mathcal{R}}$ bit operations. Mulitplication in
$\mathcal{S}$ by a power of $\rho$ amounts to $c_{\mathcal{S}} =
M^{k-1}$ multiplications in $\mathcal{R}$. Since there are $2M$ such
bad multiplications, the total time is bounded by $O(2M \cdot
c_{\mathcal{S}} \cdot \mathcal{M}_{\mathcal{R}})$.

\paragraph{Outer DFT time:}
By Lemma \ref{lem:inoutDFT}, the total outer DFT time is $2m \cdot
\mathcal{D}\left(\frac{2M}{2m}, k \right)$.

\paragraph{Total DFT time:}Therefore, the net DFT time is bounded as,
\begin{eqnarray*}
\mathcal{D}(2M, k) &=& O\left(2M \cdot \log m \cdot \mathcal{N}_{\mathcal{S}} +
  2M \cdot c_{\mathcal{S}} \cdot \mathcal{M}_{\mathcal{R}}\right) +
2m \cdot \mathcal{D}\left(\frac{2M}{2m}, k\right) \\
&=& O\left(2M \cdot \log m \cdot \mathcal{N}_{\mathcal{S}} +
  2M \cdot c_{\mathcal{S}} \cdot \mathcal{M}_{\mathcal{R}}\right) \cdot
\frac{\log 2M}{ \log 2m} \\
&=& O\left(M^k \log M \cdot m \log p + \frac{M^k \log M}{\log m} \cdot
  \mathcal{M}_{\mathcal{R}}\right),
\end{eqnarray*}
putting the values of $\mathcal{N}_{\mathcal{S}}$ and $c_{\mathcal{S}}$.

\paragraph{Pointwise multiplications:}
Finally, FFT does $2M$ pointwise multiplications in
$\mathcal{S}$. Since elements of $\mathcal{S}$ are $(k-1)$-variate
polynomials over $\mathcal{R}$, with degree in every variable bounded
by $M$, the total time taken for pointwise multiplications is $2M
\cdot \mathcal{F}(2M, k-1)$ bit operations.

\paragraph{Total polynomial multiplication time:}
This can be expressed as,


\begin{eqnarray} \label{eqn_FFT_complexity}
\mathcal{F}(2M, k) &=& O\left(M^k \log M \cdot m \log p + \frac{M^k \log M}{\log m} \cdot \mathcal{M}_{\mathcal{R}}\right) + 2M \cdot \mathcal{F}(2M, k-1) \nonumber \\
&=& O\left(M^k \log M \cdot m \log p + \frac{M^k \log M}{\log m} \cdot \mathcal{M}_{\mathcal{R}}\right),
\end{eqnarray}
as $k$ is a constant. \\

% \subsection*{Inverse FFT}


% \todo{Pres 9}Computing the inverse of a fourier transform  with $\rho$ as the root
% of unity just amounts to yet another fourier transform with
% $\rho^{-1}$ as the root of unity. This can be seen by observing that a
% fourier transform can be thought of as a matrix multiplication. If $a(x) = a_0 + a_1 x + \cdots a_{2M-1}x^{2M-1}$ and $b_i = a(\rho^i)$ for $i=0\cdots (2M-1)$,  then
% \begin{eqnarray*}
% \insquar{\begin{array}{c}b_0\\b_1\\b_2\\ \vdots \\b_{2M-1}\end{array}} & = &  \insquar{\begin{array}{cccc}
%       1      & 1          & \cdots &  1\\
%       1      & \rho       & \cdots & \rho^{2M-1}\\
%       1      & \rho^2     & \cdots & \rho^{2(2M-1)}\\
%       \vdots & \vdots     & \ddots & \vdots \\
%       1      & \rho^{2M-1}& \cdots & \rho^{(2M-1)(2M-1)}
%     \end{array}}\insquar{\begin{array}{c}a_0\\a_1\\a_2\\ \vdots \\a_{2M-1} \end{array}}.
% \end{eqnarray*}
% Since $\rho$ is a principal root of unity, it is easy to see that
% \begin{eqnarray*}
% \insquar{\begin{array}{c}a_0\\a_1\\a_2\\ \vdots \\a_{2M-1}\end{array}} & = &  \frac{1}{N}\cdot\insquar{\begin{array}{cccc}
%       1      & 1          & \cdots &  1\\
%       1      & \rho^{-1}       & \cdots & \rho^{-(2M-1)}\\
%       1      & \rho^{-2}     & \cdots & \rho^{-2(2M-1)}\\
%       \vdots & \vdots     & \ddots & \vdots \\
%       1      & \rho^{-(2M-1)}& \cdots & \rho^{-(2M-1)(2M-1)}
%     \end{array}}\insquar{\begin{array}{c}b_0\\b_1\\b_2\\ \vdots \\b_{2M-1} \end{array}}.
% \end{eqnarray*}
% which is just another fourier transform with $\rho^{-1}$ as a root of unity instead. \\



We now present an equivalent group theoretic interpretation of the
above process of polynomial multiplication, which is a subject of
interest in itself.

\subsection{A Group Theoretic Interpretation}

A convenient way to study polynomial multiplication is to interpret it
as multiplication in a \emph{group algebra}.

\begin{definition}
\emph{\textsf{(Group Algebra)}}
  Let $G$ be any group. The \emph{group algebra} of $G$ over a ring $R$ is
  the set of formal sums $\sum_{g \in G} \alpha_g g$ where $\alpha_g
  \in R$ with addition defined point-wise and multiplication defined
  via convolution as follows
  $$ \left(\sum_g \alpha_g g\right) \left(\sum_h
  \beta_h h\right) = \sum_{u}\inparen{\sum_{gh=u} \alpha_g \beta_h}u $$
\end{definition}

In this section, we study the Fourier transform over the group algebra
$R[E]$ where $E$ is an \emph{additive abelian group}. Most of this,
albeit in a different form, is well known but is provided here for
completeness \cite[Chapter 17]{Igor}.

In order to simplify our presentation, we will fix the base ring to be
$\C$, the field of complex numbers. Let $n$ be the \emph{exponent} of
$E$, that is the maximum order of any element in $E$. A similar
approach can be followed for any other base ring as long as it has a
principal $n$-th root of unity.

We consider $\C[E]$ as a vector space with basis $\{ x \}_{x \in E}$
and use the Dirac notation to represent elements of $\C[E]$ --- the
vector $\ket{x}$, $x$ in $E$, denotes the element $1 . x$ of $\C[E]$.

Multiplying univariate polynomials over $R$ of degree less than $n$
can be seen as multiplication in the group algebra $R[G]$ where $G$ is
the cyclic group of order $2n$. Say $a(x) = a_0 + a_1 x +
\cdots + a_dx^d$ and $b(x)= b_0 + b_1x \cdots + b_dx^d$ (with $d<n$)
are the polynomials we wish to multiply, they can be embedded in
$\C[\Z/2n\Z]$ as $\ket{a} = \sum_{i=0}^d a_i \ket{i}$ and $\ket{b} =
\sum_{i=0}^d b_i\ket{i}$. It is trivial to see that their product in
the group algebra is the embedding of the product of the
polynomials. Similarly, multiplying $k$-variate polynomials of degree
less than $n$ in each variable can be seen as multiplying in the group
algebra $R[G^k]$, where $G^k$ denotes the $k$-fold product group
$G\times\ldots \times G$.



\begin{definition}
\emph{\textsf{(Characters)}} Let $E$ be an additive abelian group. A
\emph{character} of $E$ is a homomorphism from $E$ to $\C^*$.
\end{definition}

An example of a character of $E$ is the trivial character, which we
will denote by $1$, that assigns to every element of $E$ the complex
number $1$. If $\chi_1$ and $\chi_2$ are two characters of $E$ then
their product $\chi_1 . \chi_2$ is defined as $\chi_1 . \chi_2(x) =
\chi_1(x) \chi_2(x)$.

\begin{proposition}\cite[Chapter 17, Theorem 1]{Igor}\label{prop:dual-isomorphism}
  Let $E$ be an additive abelian group of exponent $n$. Then the
  values taken by any character of $E$ are $n$-th roots of
  unity. Furthermore, the characters form a \emph{multiplicative
    abelian group} $\hat{E}$ which is isomorphic to $E$.
\end{proposition}

An important property that the characters satisfy is the following
\cite[Corollary 2.14]{Isaacs}.

\begin{proposition}\label{prop:schur-orthogonality}
\emph{\textsf{(Schur's Orthogonality)}}
  Let $E$ be an additive abelian group. Then
  \[ \sum_{x \in E} \chi(x) =%
  \begin{cases}
    0 & \textrm{ if $\chi \neq 1$,}\\
    \# E &\textrm{ otherwise}
  \end{cases} \quad \text{and}\quad
  \sum_{\chi \in \hat{E}} \chi(x) =%
  \begin{cases}
    0 & \textrm{ if $x \neq 0$,}\\
    \# E &\textrm{ otherwise.}
  \end{cases}
  \]
\end{proposition}

It follows from Schur's orthogonality that the collection of vectors
$\ket{\chi} = \sum_x \chi(x) \ket{x}$ forms a basis of $\C[E]$. We
will call this basis the \emph{Fourier basis} of $\C[E]$.

\begin{definition}
\emph{\textsf{(Fourier Transform)}}
Let $E$ be an additive abelian group and let $x \mapsto \chi_x$ be an
isomorphism between $E$ and $\hat{E}$. The \emph{Fourier transform}
over $E$ is the linear map from $\C[E]$ to $\C[E]$ that sends
$\ket{x}$ to $\ket{\chi_x}$.
\end{definition}

Thus, the Fourier transform is a change of basis from the point basis
$\{ \ket{x} \}_{x \in E}$ to the Fourier basis $\{
\ket{\chi_x}\}_{x\in E}$. The Fourier transform is unique only up to
the choice of the isomorphism $x \mapsto \chi_x$. This isomorphism is
determined by the choice of the principal root of unity.

It is a standard fact in representation theory that any character
$\chi$ of an abelian group satisfies $\chi_y(x) = \chi_x(y)$ for every
$x,y$. Using this and Proposition~\ref{prop:schur-orthogonality}, it is easy
to see that this transform can be inverted by the map $\ket{x} \mapsto
\frac{1}{n}\ket{\overline{\chi_x}}$. Hence the \emph{Inverse Fourier
  Transform} is essentially just a Fourier transform using $x\mapsto
\overline{\chi_x}$ as the isomorphism between $E$ and $\hat{E}$.

\begin{remark}\label{rem-Fourier-inner}
  Given an element $\ket{f} \in
  \C[E]$, to compute its Fourier transform (or Inverse Fourier transform) it is sufficient to compute
  the \emph{Fourier coefficients} $\{\braket{\chi}{f} \}_{\chi \in
    \hat{E}}$.
\end{remark}

\subsubsection*{Fast Fourier Transform}

We now describe the Fast Fourier Transform for general abelian groups
in the character theoretic setting. For the rest of the section fix an
additive abelian group $E$ over which we would like to compute the
Fourier transform. Let $A$ be any subgroup of $E$ and let $B =
E/A$. For any such pair of abelian groups $A$ and $B$, we have an
appropriate Fast Fourier transformation, which we describe in the rest
of the section.

\begin{proposition}\label{prop-character-lift}
  \begin{enumerate}
  \item Every character $\lambda$ of $B$ can be ``lifted'' to a
    character of $E$ (which will be denoted by $\tilde\lambda$ defined
    as follows $\tilde\lambda(x) = \lambda(x + A)$.
  \item Let $\chi_1$ and $\chi_2$ be two characters of $E$ that when
    restricted to $A$ are identical. Then $\chi_1 = \chi_2 \tilde\lambda$ for
    some character $\lambda$ of $B$.
  \item The group $\hat{B}$ is (isomorphic to) a subgroup of $\hat{E}$
    with the quotient group $\hat{E}/\hat{B}$ being (isomorphic to)
    $\hat{A}$.
  \end{enumerate}
\end{proposition}
\begin{proof}
It is very easy to check that $\tilde\lambda(x) = \lambda(x + A)$ is indeed
a homomorphism from $E$ to $\C$. This therefore establishes that
$\hat{B}$ is a subgroup of $\hat{E}$.

\medskip As for the second, define the map $\tilde\lambda(x) =
\frac{\chi_1(x)}{\chi_2(x)}$. It is easy to check that this is a
homomorphism from $E$ to $\C$. Then, for $x\in E$ and $a\in A$
$$
\tilde\lambda(x + a) = \frac{\chi_1(x)\chi_1(a)}{\chi_2(x)\chi_2(a)} =
\frac{\chi_1(x)}{\chi_2(x)} = \tilde\lambda(x)
$$
And hence $\tilde\lambda$ is equal over cosets over $A$ in $E$ and hence $\chi$
is indeed a homomorphism from $B$ to $\C$.

\medskip The third part follows from
Proposition~\ref{prop:dual-isomorphism} and the fact that any quotient
group of a finite abelian group is isomorphic to a subgroup.
% For the third, every character $\chi\in \hat{E}$ can be restricted
% to $A$ to get a character $\phi\in \hat{A}$. Therefore, the
% restriction map is a natural homomorphism from $\hat{E}$ to
% $\hat{A}$. It suffices to show that this homomorphism is surjective
% and the kernel is $\hat{B}$. Suppose $\varphi$ is an arbitrary
% character of $A$. Let $R = \inbrace{x_b}_{b\in B}$ be a set of coset
% representatives of $A$ in $E$. Then every element $x\in E$ can be
% uniquely written as $x_b + a$ where $x_b\in R$ and $a\in
% A$. Consider the following map:
% $$
% \chi(x_b + a) =  \varphi(a)
% $$ It is easy to verify that this is indeed a character of $E$, whose
% restriction to $A$ is $\varphi$. Therefore, the restriction map is
% surjective. And if $\chi$ is in the kernel of this restriction, then
% $\chi$ and the trivial character are identical on $A$ and therefore
% $\chi = 1\cdot \lambda = \lambda \in \hat{B}$. Thus, the kernel is
% precisely $\hat{B}$ and hence $\hat{A}$ is the quotient of $\hat{E}$
% and $\hat{B}$.
\end{proof}



We now consider the task of computing the Fourier transform of an
element $\ket{f} = \sum f_x \ket{x}$ presented as a list of
coefficients $\{f_x\}$ in the point basis. For this, it is sufficient
to compute the Fourier coefficients $\{\braket{\chi}{f}\}$ for each
character $\chi$ of $E$ (Remark~\ref{rem-Fourier-inner}). To describe
the Fast Fourier transform we fix two sets of cosets representatives,
one of $A$ in $E$ and one of $\hat{B}$ in $\hat{E}$ as follows.

\begin{enumerate}
  \item For each $b \in B$, $b$ being a coset of $A$, fix a coset
    representative $x_b \in E$ such $b = x_b + A$.
  \item For each character $\varphi$ of $A$, fix a character
    $\chi_\varphi$ of $E$ such that $\chi_\varphi$ restricted to $A$ is
    the character $\varphi$. The characters $\{ \chi_\varphi \}$ form
    (can be thought of as) a set of coset representatives of $\hat{B}$
    in $\hat{E}$.
\end{enumerate}

Since $\{ x_b \}_{b \in B}$ forms a set of coset representatives, any
$\ket{f} \in \C[E]$ can be written uniquely as $\ket{f} = \sum f_{b,a}
\ket{x_b + a}$.

\begin{proposition}\label{prop-Fourier-coefficient}
  Let $\ket{f} = \sum f_{b,a}\ket{x_b + a}$ be an element of $\C[E]$.
  For each $b \in B$ and $\varphi \in \hat{A}$ let $\ket{f_b}\in
  \C[A]$ and $\ket{f_\varphi} \in \C[B]$ be defined as
  follows.
  \begin{eqnarray*}
    \ket{f_b} &= & \sum_{a \in A} f_{b,a} \ket {a}\\
    \ket{f_\varphi} & = &\sum_{b \in B} \overline{\chi}_{\varphi}(x_b)
    \braket{\varphi}{f_b} \ket{b}
  \end{eqnarray*}
  Then for any character $\chi = \chi_\varphi\tilde\lambda$ of $E$ the
  Fourier coefficient $\braket{\chi}{f} =
  \braket{\lambda}{f_\varphi}$.
\end{proposition}

\begin{proof}
$$
\braket{\chi}{f} \quad=\quad \sum_{b\in B , a\in A} \overline{\chi_\varphi
  \tilde\lambda(x_b + a)} \cdot f_{b,a}
$$
\noindent
Recall that for any $\tilde\lambda$, that is a lift of a character $\lambda$ of
$B$, acts identically inside cosets of $A$ and hence $\tilde\lambda(x_b + a) = \lambda(b)$. Therefore, the above sum can be
rewritten as follows:
\begin{eqnarray*}
\sum_{b\in B , a\in A} \overline{\chi_\varphi
  \tilde\lambda(x_b + a)} \cdot f_{b,a} & = & \sum_{b\in B} \sum_{a\in A}
\overline{\chi_\varphi(x_b + a)}\overline{\lambda(b)} \cdot
f_{b,a}\\
 & = & \sum_{b\in B} \overline{\lambda(b)}\cdot
\overline{\chi_\varphi(x_b)}\sum_{a\in A}
\overline{\varphi(a)} f_{b,a}
\end{eqnarray*}
The inner sum over $a$ is precisely $\braket{\varphi}{f_b}$ and
therefore we have:
\begin{eqnarray*}
\braket{\chi}{f} & = & \sum_{b\in B}
\overline{\lambda(x_b)}\cdot \overline{\chi_\varphi(x_b)}
\braket{\varphi}{f_b}
\end{eqnarray*}
which can be rewritten as $\braket{\lambda}{f_\varphi}$ as claimed.
\end{proof}


We are now ready to describe the Fast Fourier transform given an
element $\ket{f} = \sum f_x \ket{x}$.

\begin{enumerate}
\item \label{step_inner_dft} For each $b \in B$ compute the Fourier
  transforms of $\ket{f_b}$. This requires $\# B$ many Fourier
  transforms over $A$.
\item \label{step_bad_mult}As a result of the previous step we have
  for each $b \in B$ and $\varphi \in \hat{A}$ the Fourier
  coefficients $\braket{\varphi}{f_b}$. Compute for each $\varphi$ the
  vectors $\ket{f_\varphi} = \sum_{b \in B}
  \overline{\chi}_{\varphi}(x_b) \braket{\varphi}{f_b} \ket{b}$. This
  requires $\# \hat{A} . \# B = \# E$ many multiplications by roots of
  unity.
\item \label{step_outer_dft} For each $\varphi \in \hat{A}$ compute
  the Fourier transform of $\ket{f_\varphi}$. This requires $\#\hat{A}
  = \# A$ many Fourier transforms over $B$.\label{item-Fourier-B}
\item Any character $\chi$ of $E$ is of the
  form $\chi_\varphi \lambda$ for some $\varphi \in \hat{A}$ and
  $\lambda \in \hat{B}$. Using
  Proposition~\ref{prop-Fourier-coefficient} we have at the end of
  Step~\ref{item-Fourier-B} all the Fourier coefficients
  $\braket{\chi}{f} = \braket{\lambda}{f_\varphi}$.
\end{enumerate}

If the quotient group $B$ itself has a subgroup that is isomorphic to
$A$ then we can apply this process recursively on $B$ to obtain a divide and
conquer procedure to compute Fourier transform. In the standard FFT we
use $E = \Z/2^n\Z$. The subgroup $A$ is $2^{n-1}E$ which is isomorphic
to $\Z/2\Z$ and the quotient group $B$ is $\Z/2^{n-1}\Z$.

\subsubsection*{Analysis of the Fourier Transform}

Our goal is to multiply $k$-variate polynomials over $\mathcal{R}$, with the
degree in each variable less than $M$. This can be achieved by
embedding the polynomials into the algebra of the product group $E =
\inparen{\frac{\Z}{2M\cdot \Z}}^k$ and multiplying them as elements of
the algebra. Since the exponent of $E$ is $2M$, we require a principal
$2M$-th root of unity in the ring $\mathcal{R}$. We shall use the root
$\rho(\alpha)$ (as defined in Section~\ref{root_section}) for the
Fourier transform over $E$.

For every subgroup $A$ of $E$, we have a corresponding FFT. We choose
the subgroup $A$ as $\inparen{\frac{\Z}{2m\cdot \Z}}^k$ and let $B$ be
the quotient group $E/A$. The group $A$ has exponent $2m$ and $\alpha$
is a principal $2m$-th root of unity. Since $\alpha$ is a power of
$\rho(\alpha)$, we can use it for the Fourier transform over $A$. As
multiplications by powers of $\alpha$ are just shifts, this makes
Fourier transform over $A$ efficient.

Let $\mathcal{F}(M,k)$ denote the complexity of computing the Fourier transform
over $\inparen{\frac{\Z}{2M\cdot \Z}}^k$. We have
\begin{equation}
\mathcal{F}(M,k) = \inparen{\frac{M}{m}}^k \mathcal{F}(m,k) + (2M)^k
\mathcal{M}_{\mathcal{R}}+(2m)^k\mathcal{F}\inparen{\frac{M}{2m},k}
\label{first_recursive_step}
\end{equation}
where $\mathcal{M}_{\mathcal{R}}$ denotes the complexity of multiplications in
$\mathcal{R}$. The first term comes from the $\# B$ many Fourier transforms
over $A$ (Step~\ref{step_inner_dft} of FFT), the second term
corresponds to the multiplications by roots of unity
(Step~\ref{step_bad_mult}) and the last term comes from the $\# A$
many Fourier transforms over $B$ (Step~\ref{step_outer_dft}).

Since $A$ is a subgroup of $B$ as well, Fourier transforms over $B$
can be recursively computed in a similar way, with $B$ playing the
role of $E$. Therefore, by simplifying the recurrence in
Equation~\ref{first_recursive_step} we get:
\begin{equation}
\mathcal{F}(M,k) = O\inparen{\frac{M^k\log M}{m^k\log m}\mathcal{F}(m,k) +
  \frac{M^k\log M}{\log m}\mathcal{M}_{\mathcal{R}}}
\label{eqn_with_Fa}
\end{equation}

\begin{lemma}\label{lem-Fmk}
$\mathcal{F}(m,k) = O(m^{k+1}\log m\cdot \log p)$
\end{lemma}
\begin{proof}
The FFT over a group of size $n$ is usually done by taking $2$-point
FFT's followed by $\frac{n}{2}$-point FFT's. This involves $O(n\log
n)$ multiplications by roots of unity and additions in base
ring. Using this method, Fourier transforms over $A$ can be computed
with $O(m^k\log m)$ multiplications and additions in $\mathcal{R}$. Since each
multiplication is between an element of $\mathcal{R}$ and a power of $\alpha$,
this can be efficiently achieved through shifting operations. This is
dominated by the addition operation, which takes $O(m\log p)$ time,
since this involves adding $m$ coefficients from $\Z/p^c\Z$.
\end{proof}

Therefore, from Equation~\ref{eqn_with_Fa},
\begin{equation*}
\mathcal{F}(M,k) = O\inparen{M^k\log M\cdot m\cdot \log p + \frac{M^k\log
    M}{\log m}\mathcal{M}_{\mathcal{R}}}.
\end{equation*}

\section{Algorithm and Analysis}
\subsection{Integer Multiplication Algorithm}\label{intmult_section}

We are given two integers $a,b< 2^N$ to multiply. We fix constants $k$
and $c$  whose values are given in
Section~\ref{complexity_section}. The algorithm is as follows:
\begin{enumerate}
\item Choose $M$ and $m$ as powers of two such that $M^k \approx
  \frac{N}{\log^2N}$ and $m \approx \log N$. Find the least prime
  $p\equiv 1\pmod{2M}$ (Lemma~\ref{prime_time}).
\item Encode the integers $a$ and $b$ as $k$-variate polynomials
  $a(X)$ and $b(X)$, respectively, over the ring $\mathcal{R} =
  \Z[\alpha]/(p^c, \alpha^m + 1)$ (Section~\ref{encoding_section}).
\item Compute the root $\rho(\alpha)$ (Section~\ref{root_section}).
\item Use $\rho(\alpha)$ as the principal $2M$-th root of unity to
  compute the Fourier transforms of the $k$-variate polynomials $a(X)$
  and $b(X)$. Multiply component-wise and take the inverse Fourier
  transform to obtain the product polynomial. (Sections
  \ref{sec:inoutDFT} and \ref{fourier_analysis})
\item Evaluate the product polynomial at appropriate powers of two to
  recover the integer product and return it
  (Section~\ref{encoding_section}).
\end{enumerate}

\subsection{Complexity Analysis}\label{complexity_section}

The choice of parameters should ensure that the following constraints
are satisfied:
\begin{enumerate}
\item $M^k = O\inparen{\frac{N}{\log^2N}}$ and $m = O(\log
  N)$.
\item $M^L = O(N^\varepsilon)$, where $L$ is the Linnik constant
  (Theorem~\ref{linnik_theorem}) and $\varepsilon$ is any constant less
  than $1$. Recall that this makes picking the prime by brute force
  feasible (see Lemma~\ref{prime_time}).
\item $p^c > 2M^k\cdot m\cdot 2^{2u}$ where $u = \frac{2N}{M^km}$. This
  is to prevent overflows during modular arithmetic (see
  Section~\ref{encoding_section}).
\end{enumerate}
\noindent
It is straightforward to check that $k > L+1$ and $c > 5(k+1)$ satisfy
the above constraints. Since $L\leq 5.2$, it is sufficient to choose
$k = 7$ and $c = 42$.\\


Let $T(N)$ denote the time complexity of multiplying two $N$ bit
integers. This consists of:

\begin{itemize}
\item Time required to pick a suitable prime $p$,
\item Computing the root $\rho(\alpha)$,
\item Encoding the input integers as polynomials,
\item Multiplying the encoded polynomials,
\item Evaluating the product polynomial.
\end{itemize}

As argued before, the prime $p$ can be chosen in $o(N)$ time. To
compute $\rho(\alpha)$, we need to lift a generator of
$\mathbb{F}_p^{\times}$ to $\Z/p^c\Z$ followed by an interpolation. Since $c$
is a constant and $p$ is a prime of $O(\log N)$ bits, the time
required for Hensel Lifting and interpolation is $o(N)$.

The encoding involves dividing bits into smaller blocks, and
expressing the exponents of $q$ in base $M$
(Section~\ref{encoding_section}) and all these take $O(N)$ time since
$M$ is a power of $2$. Similarly, evaluation of the product polynomial
takes linear time as well. Therefore, the time complexity is dominated
by the time taken for polynomial multiplication.

\subsubsection*{Time complexity of Polynomial Multiplication}

From Equation~\ref{eqn_FFT_complexity}, the complexity of polynomial multiplication is given by,
\[
\mathcal{F}(2M,k) = O\inparen{M^k\log M\cdot m\cdot \log p + \frac{M^k\log
    M}{\log m}\cdot \mathcal{M}_{\mathcal{R}}}.
\]

\begin{proposition}\cite{scho_complex}
  If $\mathcal{M}_{\mathcal{R}}$ denotes the complexity of
  multiplication in $\mathcal{R}$, then $\mathcal{M}_{\mathcal{R}} =
  T\left(O(\log^2{N})\right)$ where $T(x)$ denotes the complexity of
  multiplying two $x$-bit integers.
\end{proposition}
\begin{proof}
  Elements of $\mathcal{R}$ can be viewed as polynomials in $\alpha$
  over $\Z/p^c\Z$ with degree at most $m$. Given two such polynomials
  $f(\alpha)$ and $g(\alpha)$, encode them as follows: Replace
  $\alpha$ by $2^d$, transforming the polynomials $f(\alpha)$ and
  $g(\alpha)$ to the integers $f(2^d)$ and $g(2^d)$ respectively.  The
  parameter $d$ is chosen such that the coefficients of the product
  $h(\alpha) = f(\alpha) g(\alpha)$ can be recovered from the product
  $f(2^d)\cdot g(2^d)$. For this, it is sufficient to ensure that the
  maximum coefficient of $h(\alpha)$ is less than $2^d$.  Since $f$
  and $g$ are polynomials of degree $m$, we would want $2^d$ to be
  greater than $m\cdot p^{2c}$, which can be ensured by choosing $d =
  O\left(\log{N}\right)$. The integers $f(2^d)$ and $g(2^d)$ are
  bounded by $2^{md}$, which is of $O(\log^2 N)$ bits. The product
  $f(\alpha)\cdot g(\alpha)$ can be decoded from the integer product
  $f(2^d)\cdot g(2^d)$ by splitting the bits into $(2m-1)$ blocks of
  $d$ bits each (one for each coefficient of $\alpha^i$) to obtain a
  polynomial in $\Z[\alpha]$ and reducing it modulo $p^c$ and
  $\alpha^m + 1$. Reducing modulo $(\alpha^m + 1)$ can be performed in
  $O(md) = O(\log^2N)$ time. Dividing by $p^c$, which has $O(\log N)$
  bits, can be performed in the same time as multiplying $O(\log N)$
  bit integers using standard techniques (see for example
  \cite[Chapter 4]{Knuth}). Since $T(N) = \Omega(N)$, we have that
  $\mathcal{M}_{\mathcal{R}} = T(O(\log^2 N)) + O(\log N \cdot
  T(O(\log N))) = T(O(\log^2 N))$ bit operations.
\end{proof}

Therefore, the complexity of our integer multiplication algorithm
$T(N)$ is given by,
\begin{eqnarray*}
T(N) & = & O(\mathcal{F}(2M, k)) = O\inparen{M^k\log M\cdot m\cdot \log p + \frac{M^k\log
  M}{\log m} \cdot \mathcal{M}_{\mathcal{R}}}\\
& = & O\inparen{N\log N + \frac{N}{\log N\cdot \log\log N} \cdot T(O(\log^2N))}
\end{eqnarray*}

\noindent Solving the above recurrence leads to the following theorem.

\begin{theorem} \label{thm:mainthmx}
Given two $N$ bit integers, their product can be computed using
$N\cdot \log N\cdot 2^{O(\log^*N)}$ bit operations.
\end{theorem}

\subsubsection*{Performing on multi-tape turing machines}

The upper-bound presented in Theorem \ref{thm:mainthmx} holds for
multi-tape turing machines. The only part of the algorithm that
warrants an explanation is regrouping of the terms in preparation for
the inner and outer DFTs. For the inner DFT, we are given $\ket{f} =
\sum_{a,b} f_{b,a}\ket{x_b + a}$ and we wish to write down $\ket{f_b}
= \sum_{a} f_{b,a}\ket{a}$ for each $b\in B$. The following discussion
essentially outlines how this can be performed on a multi-tape turing
machine using $O(N\log m)$ bit operations. (Recall that the group $E =
(\Z/2M\Z)^k$ and $A = (\Z/2m\Z)^k$ and both $M$ and $m$ are powers of
$2$). \\

We may assume that the coefficients are listed according to the
natural lexicographic order of $(\Z/2M\Z)^k$. To ease the
presentation, we first present the approach for $k=1$. It would be
straightforward to generalize this to larger $k$ by repeated
applications of this approach.


Given as input is a sequence of coefficients $f_g$ for each $g\in
\tilde{E} = \Z/2M\Z$ in the natural lexicographic order on one of the
tapes of the turing machine. We can then ``shuffle'' the input tape
using two passes to order the coefficients according to
$\inbrace{0,M,1,M+1,\dots, M-1,2M-1}$ (by copying the first half with
appropriate blanks, and copying the second half on the second
pass). This results in regrouping the inputs as various cosets of
$\Z/2\Z$. By considering two successive elements as a block and
repeating the shuffling, we obtain various cosets of
$Z/4\Z$. Repeating this process $\log(2m)$ times regroups the elements
according to various cosets of $\Z/2m\Z$.\\

Suppose that $E = (\Z/2M\Z)^k$ and that the coefficients are ordered
according to the natural lexicographic order. By repeating the same
shuffling process for $\log(2m)$ steps, the coefficients are regrouped
according to the cosets of $\inbrace{0}^{k-1} \times (\Z/2m\Z)$ and
hence there are $M/m$ many groups. By considering every $2m$
successive coefficients as one block, each of the $M/m$ groups can
be thought of as coefficients ordered according to the natural
lexicographic order of $(\Z/2M\Z)^{k-1}$. By repeating this for
$(k-1)$ more steps, we can reorder the coefficients according to the
cosets of $(\Z/2m\Z^k)$.

The procedure totally requires $k\log(2m)$ passes over the tapes and
hence can be performed in $O(N\log m)$ time on a $3$-tape\footnote{an
  input tape, an output tape, and a third tape for a counter} turing
machine. With the above discussion, it can be easily seen that the
upper-bound holds for multi-tape turing machines.

\begin{theorem}
Given two $N$ bit integers, their product can be computed using
$N\cdot \log N\cdot 2^{O(\log^*N)}$ bit operations on a multi-tape turing machine.
\end{theorem}


\section{A Comparison with F\"{u}rer's Algorithm} \label{Qp_section}

Our algorithm can be seen as a $p$-adic version of F\"{u}rer's integer
multiplication algorithm, where the field $\C$ is replaced by $\Q_p$,
the field of $p$-adic numbers (for a quick introduction, see Baker's
online notes \cite{Baker}). Much like $\C$, where representing a
general element (say in base $2$) takes infinitely many bits,
representing an element in $\Q_p$ takes infinitely many $p$-adic
digits. Since we cannot work with infinitely many digits, all
arithmetic has to be done with finite precision. Modular arithmetic in
the base ring $\Z[\alpha]/(p^c, \alpha^m + 1)$, can be viewed as
arithmetic in the ring $\Q_p[\alpha]/(\alpha^m + 1)$ keeping a
precision of $\varepsilon = p^{-c}$.

Arithmetic with finite precision naturally introduces some errors in
computation. However, the nature of $\Q_p$ makes the error analysis
simpler. The field $\Q_p$ comes with a norm $\abs{\ \cdot\ }_p$ called
the $p$-adic norm, which satisfies the stronger triangle inequality
$\abs{x+y}_p \leq \max\inparen{\abs{x}_p, \abs{y}_p}$ \cite[Proposition
  2.6]{Baker}. As a result, unlike in $\C$, the errors in computation
do not compound.\\

Recall that the efficiency of FFT crucially depends on a special
principal $2M$-th root of unity in $\Q_p[\alpha]/(\alpha^m + 1)$. Such
a root is constructed with the help of a primitive $2M$-th root of
unity in $\Q_p$. The field $\Q_p$ has a primitive $2M$-th root of
unity if and only if $2M$ divides $p-1$ \cite[Theorem
  5.12]{Baker}. Also, if $2M$ divides $p-1$, a $2M$-th root can be
obtained from a $(p-1)$-th root of unity by taking a suitable power. A
primitive $(p-1)$-th root of unity in $\Q_p$ can be constructed, to
sufficient precision, using Hensel Lifting starting from a generator
of $\mathbb{F}_p^{\times}$.

\section{Conclusion}\label{conclusions_section}

As mentioned earlier, there has been two approaches to multiplying
integers - one using arithmetic over complex numbers and the other
using modular arithmetic. Using complex numbers, Sch\"{o}nhage and
Strassen \cite{SS71} gave an $O(N \cdot \log N \cdot \log\log N\ldots
2^{O(\log^* N)})$ algorithm. F\"{u}rer \cite{F07} improved this
complexity to $N\cdot\log N \cdot2^{O(\log^*N)}$ using some special
roots of unity. The other approach, that is modular arithmetic, can be
seen as arithmetic in $\Q_p$ with certain precision. A direct
adaptation of the Sch\"{o}nhage-Strassen's algorithm in the modular
setting leads to an $O(N \cdot \log N \cdot \log\log N\ldots
2^{O(\log^* N)})$ time algorithm. In this work, we show that by
choosing an appropriate prime and a special root of unity, a running
time of $N\cdot \log N \cdot 2^{O(\log^*N)}$ can be achieved through
modular arithmetic as well. Therefore, in a way, we have unified the
two paradigms. The important question that remains open is:
\begin{itemize}
\item Can $N$-bit integers be multiplied using $O(N \cdot \log N)$ bit operations?
\end{itemize}

\noindent Even an improvement of the complexity to $O(N \cdot \log N
\cdot \log^{*}N)$ operations will be a significant step forward
towards answering this question.

\section*{Acknowledgements}

We are greatly thankful to the anonymous reviewers for their detailed
comments that have improved the presentation of the paper
significantly.


\bibliography{references}
\end{document}
