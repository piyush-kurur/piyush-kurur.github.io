The pseudo-Boolean function~\leadingones counts the number of one-bits
preceding the first zero-bit in a bit-string~$\mathbf{x}\in\{0,1\}^n$,
that is, let

\begin{equation}
\leadingones(\mathbf{x}):=\sum_{k=1}^n\prod_{i=1}^k x_i\,.
\end{equation}


The following theorem can be deduced from~\cite{DJWoneone}.

\begin{theorem}
  The expected optimization times of the (1+1)~EA, the (1+1)~EA$^*$,
  RLS, and RLS$^*$ for maximizing \leadingones are in~$\Theta(n^2)$.
\end{theorem}

For the progressive selection rule, quantum acceleration does not yield
a substantial speed-up. In contrast, for the conservative selection
rule the expected optimization time decreases considerably. However,
it does not decrease quadratically.

\begin{theorem}
  The expected optimization times of the (1+1)~QEA$^*$ and of QLS$^*$
  for maximizing \leadingones are in~$\Theta(n^{3/2})$.  The expected
  optimization times of the (1+1)~QEA or QLS for maximizing
  \leadingones are in~$\Theta(n^{2})$.
\end{theorem}

\begin{proof}
  We start with the conservative selection strategy. Let us first
  consider QLS$^*$. For any non-optimal search point~$\xvec$, the
  mutation step yields a better search point if and only if the first
  zero-bit is flipped. So the progress probability is $p_{\xvec} =
  1/n$. For the (1+1)~QEA$^*$ let $\xvec$ be any non-optimal search
  point. Assume that $x_k$ is the first zero-bit in~$\xvec$. Then
  mutation step yields a better search point if and only if $x_k$ is
  flipped, and all preceding bits are unchanged. Therefore, the
  progress probability is
  \[
  p_{\xvec} = \frac{1}{n}\Big(1-\frac{1}{n}\Big)^{i_{t-1}-1}
  \]
  Thus, since~$(1-1/n)^{n-1}\ge\euler^{-1}$ and~$0\le i_{t-1}\le n$,
  \[
  \frac{1}{\euler\,n}\le p_t\le\frac{1}{n}\,.
  \]
  
  So for both QLS$^*$ and the (1+1)~QEA$^*$, we have shown that
  $1/(\euler\,n)\le p_t\le 1/n$.  Therefore, by
  Lemma~\ref{lem:boundedq},
  \[
  \frac{1}{\sqrt{\euler\,n}}\EXP[T]\le \TQSH \le \frac{1}{\sqrt{n}}\EXP[T],
  \]
  and in particular
  \[
  \EXP[\TQSH] = \Theta(n^{-1/2}\EXP[T]) = \Theta(n^{3/2}).
  \]

  Now consider the progressive selection strategy. The upper bound is
  trivial, as the QSH is always asymptotically at least as fast as
  the corresponding RSH.

  For the lower bound, we partition the search space into two sets 
  \[
  S_1 := \left\{\xvec \mid \leadingones(\xvec) < n/2 \right\},
  \]
  \[
  S_2 := \left\{\xvec \mid \leadingones(\xvec) \geq n/2 \right\}.
  \]

  Let $N(\xvec)$ be the random variable that assigns to each run the
  number of occurrences of $\xvec$ in the trajectory. For
  $i\in\{1,2\}$, let
  \[
  \TRSH^{(i)} = \sum_{\mathbf{x}\in\mathcal{S}_i} \EXP[N(\xvec )]
  \cdot \left(p_{\mathbf{x}}\right)^{-1}
  \]
  and 
  \[
  \TQSH^{(i)} = \sum_{\mathbf{x}\in\mathcal{S}_i} \EXP[N(\xvec )]
  \cdot \left(p'_{\mathbf{x}}\right)^{-1},
  \]
  be the estimated optimization time that RLS, or the (1+1)~EA, or
  QLS, or the (1+1)~QEA spends in region $S_i$
  (cf. Lemma~\ref{lem:regions}).  From~\cite{DJWoneone} it can be
  deduced that $\TRSH^{(1)} \in \Theta(n^2)$ for both RLS and the
  (1+1)~EA.

  Let $\xvec$ be any search point in $S_1$. We want to bound the
  progress probability $p_{\xvec}$. First consider QLS. The
  progressive selection strategy will accept any search point of equal
  fitness, except $\xvec$ itself. Thus if the mutation operator flips
  any bit in the second half of $\xvec$, then the offspring is
  accepted. Therefore,
  \[
  p_{\xvec} \geq 1/2.
  \]

  Now we turn to the (1+1)~QEA. If the mutation operator flips exactly
  one bit in the second half, and no bit in the first half, then the
  offspring will be accepted, as it does not equal $\xvec$ but has at
  least the same fitness. (There are many other ways to produce
  offsprings that are accepted, but this particular way will suffice.)

  Hence, the progress probability is at least
  \[
  p_{\xvec} \ge
  \frac{n}{2}\cdot\frac{1}{n}\cdot\Big(1-\frac{1}{n}\Big)^{n-1}\ge
  \frac{1}{2e}\,.
  \]
  In both cases, we have $p_{\xvec} \geq 1/(2e)$, and therefore by
  Lemma~\ref{lem:boundedq},
  \[
  \TQSH \ge \frac{1}{\sqrt{2e}} \EXP[T] = \Theta(n^2),
  \]
  which proves the lower bound.
\end{proof}
