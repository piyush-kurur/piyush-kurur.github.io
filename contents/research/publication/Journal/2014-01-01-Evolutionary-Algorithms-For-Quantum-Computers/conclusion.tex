In this paper, we have presented an approach to evolutionary
algorithms on a quantum computer where we keep the mutation and
selection process from the classical setting and use quantum
probability amplification (Grover search) in order to find an improved
offspring more quickly. We show that this does not affect the distinct
samples that the algorithm produced on its way to an optimal solution,
the quantum amplification only speed it up.  Furthermore our approach
is universal, i.e., it works for any mutation operator. We also give
methods to estimate the optimization using parameters of the classical
heuristic.

One important property of our heuristic is that it can at most give a
quadratic improvement over the corresponding classical heuristic. This
is similar to other general settings like unordered
search~\cite{BennetBBGV1997,Zalka99} or query complexity of local
search on a graph~\cite{Aaronson06}, in which it is proven or
conjectured that quantum computers can give at most a quadratic
speedup. The analyzed examples \onemax, \leadingones, \discrepancy,
\needle, and \jumpm show that a substantial speedup is possible (as
for \leadingones) but is not guaranteed (as for \discrepancy). The
harder it is to improve the objective function, the better will
quantum acceleration work. 


We have also seen that it is important to choose the selection
strategy carefully, since not only the runtime in the classical
setting but also the speedup due to quantum acceleration depends on
the choice of the selection strategy.  The reason for the different
results is that by allowing equality of the objective functions we
increase the number of valid successor states and thus we increase the
probability to find such a state. But quantum enhancement is more
powerful if these probabilities are small, as is illustrated by
Lemma~\ref{lem:boundedq}. However, we believe there are ways to keep
quantum enhancement powerful and still allow the algorithm to move to
a successor state with unchanged objective value. We hope to discuss
these issues in further work.

%This has a surprising consequence. In order to be able to move on
%fitness plateaus, in the Step 2.b) of a \rsh the condition
%``$f(\mathbf{y}_t) > f(\mathbf{x}_t)$'' is often replaced by the
%condition ``$f(\mathbf{y}_t) \ge f(\mathbf{x}_t)$
%and~$\mathbf{y}_t\neq\mathbf{x}_t$''. We denote this variant by
%\rsh{}$^*$. In the examples we analyze, this does not make much
%difference for the classical case. However, for the quantum
%algorithms, the results differ substantially. Due to lack of space,
%we summarized the results in Table~\ref{tabA} and omit the proofs,
%which are very similar to those we carried out.

To conclude, we demonstrated a wide range of different behaviors of
the progressive and conservative versions of the (1+1)~QEA and QLS on
a number of well-studied basic pseudo-Boolean functions. In the line
of this research, the next step would be to analyze the effects of
quantum acceleration on classical problems in combinatorial
optimization.
