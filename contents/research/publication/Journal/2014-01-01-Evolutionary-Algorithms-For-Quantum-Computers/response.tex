\documentstyle{letter}

\begin{document}

Dear reviewer!\bigskip


Thank you for your comments. We list them below (except for the typos and minor remarks) and give an answer to each.

\begin{enumerate}

\item I am confused by the definition of $M(x)$, which was previously referred to as a 'zero random variable'. On page 22, in Definition 5.9, it is now instead stated that $M(x)$ is 'the random variable that is 0 with probability 1'. On page 23 (line 3) it is stated that '$M(x)$ takes only values in $\{0,1\}$'. While this is true, in fact, $M(x)$ only takes the value $0$ and thus, its expectation $m(x)$ is also $0$. Therefore I doubt that it is the probability that $x$ occurs in the trajectory (as stated on page 23, 1st paragraph). Later on page 23, another different definition for $M(x)$ (i.e. as the sum of the $M_k(x))$ can be found. This needs to be clarified.

{\bf ANSWER:}\\
The definition of $M(x)$ had been split into two parts inside of Definition 5.9: First it was defined for non-optimal search points, then (in the line you mentioned) for optimal search points. Obviously this distinction was not very clear and caused confusion, so we rephrased it.
 
The statement in the following paragraph that ``for the conservative selection rule (...) $m(x)$ is just the probability that $x$ occurs in the trajectory'' is only true for non-optimal $x$, so we added this restriction.

In the proof of Lemma 5.10 (second paragraph) we have split up the definition of $M_k$ and added the subcase ``$M_k(x) := 0$ if $x$ is optimal''. In this way the statement ``$M(x)$ is the sum of the $M_k(x)$'' becomes true for all search points $x$ (non-optimal and optimal).

\item I am quite annoyed by the false statement that 'except of x itself' was removed (as a response to my comment on page 32, line 33ff). In fact, the authors rephrased the statement to 'except x itself in the quantum case' and put it in brackets. However, my comment still remains valid and is not properly answered: The statement 'The progressive selection strategy will accept any search point of equal fitness, except of x itself.' is somehow misleading since RLS is considered and RLS does not create copies. Thus 'except x itself' is not a result of the selection strategy. Similarly, for the (1+1) EA the fact that 'it differs from x' does not influence acceptance. The (1+1) would also accept copies in the progressive variant.

{\bf ANSWER:}\\
We apologize that we indicated wrongly how we handled your comment!

We agree that the RLS can never create x itself as the next search point. We left the phrase ``except of $x$ itself in the quantum case'' because the membership oracle $M_{f,x}$ (the quantum analogue of the progressive selection strategy) explicitly forbids the search point $x$ (page 15, line 1). This is a crucial point: For example, the (1+1)EA has a constant probability of choosing $x$ itself again, and therefore we would get no speed-up at all if the quantum algorithm accepted search points to follow itself.

For this reason we would rather avoid the impression that the same search point is accepted in all cases (since it is not accepted in the quantum case).

If you think that the confusion we may possibly raise by the phrase ``(except x itself in the quantum case)'' is larger than by omitting it, then please tell us and we will remove the phrase.

\item In several places the term 'fitness level' is used without explaining what this actually is. Since this is a well-defined term, I suggest to add a proper definition when first using it.

{\bf ANSWER:}\\
In the second paragraph after proposition 6.3, we added the sentence: ``The search space may be subdivided into regions of equal fitness, which we call~\emph{fitness levels}.''

\item I do not understand the argumentation on the lower bound for OneMax and the classical variant. Why are the values $k=1,...,n/24$ the worst case? The worst case for the lower bound should in fact be the best case for the run of the algorithm. Thus, we need to find values for $k$ that minimize $\sum_{k} n/k$. This is not the case for $k=1,...,n/24$. Note, that in the quantum case you correctly use $k=(23/24)n...n$. However, this would only yield a linear lower bound in the classical case.

{\bf ANSWER:}\\
You are right, the argument was wrong. We replaced the two paragraphs by an outline of the proof of Theorem 6.2. 

\end{enumerate}

Thank you again for all your efforts!\bigskip

Yours sincerely,\\
Daniel Johannsen, Piyush Kurur, Johannes Lengler

\end{document}