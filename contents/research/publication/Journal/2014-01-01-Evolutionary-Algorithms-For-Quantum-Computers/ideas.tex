\documentclass{article}
\usepackage{amsthm}
\usepackage{amssymb}

\newcommand{\ket}[1]{\ensuremath{\left\vert #1 \right \rangle}}
\newcommand{\weight}[1]{\ensuremath{\mathrm{w}\left(#1\right)}}

\begin{document}

\section*{Change in mutation operator}

Let $\Sigma=\{0,1\}$. Consider an $n$ length string $\mathbf{x} \in
\Sigma^n$. A flip in positions $\mathbf{a} \in \Sigma^n$ is defined
using the operator 

\[
U_{\mathbf{a}} \ket{\mathbf{x}} = \ket{\mathbf{x} + \mathbf{a}}
\]

In the quantum setting there is a possibility of applying the phase
flip as well which is defined using the operator $V_\mathbf{b}$ as

\[
V_\mathbf{b} \ket{\mathbf{x}} =
(-1)^{\mathbf{b}^{\mathrm{T}}\mathbf{x}} \ket{\mathbf{x}}.
\]

In general for $\mathbf{a}$ and $\mathbf{b}$ in $\Sigma^n$, let
$W_{\mathbf{a},\mathbf{b}}$ denote the operator
$U_\mathbf{a}V_\mathbf{b}$, then candidate solutions $\ket{x}$ can be
mutated by applying any of the operators $W_{\mathbf{a},\mathbf{b}}$.

The parameter of interest here is the joint weight defined as

\[
\weight{\mathbf{a},\mathbf{b}} = \# \{ i | (a_i,b_i) \neq (0,0) \}.
\]

Local search and EA on an average make only one mutation per
step. This can be modeled by making transformation
$W_{\mathbf{a},\mathbf{b}}$ where the average joint weight
$\weight{\mathbf{a},\mathbf{b}}$.

\end{document}
