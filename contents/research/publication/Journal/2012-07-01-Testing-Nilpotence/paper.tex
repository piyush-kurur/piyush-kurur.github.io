\documentclass[prodmod,acmtalg]{acmsmall}
%\documentclass{article}
%\usepackage{fullpage}

%\usepackage{a4}
%\usepackage{hyperref}
\usepackage{amsmath}
%\usepackage{amsthm}
\usepackage{amssymb}

\usepackage{algorithm2e}

%\newtheorem{theorem}{Theorem}[section]
%\newtheorem{definition}[theorem]{Definition}
%\newtheorem{lemma}[theorem]{Lemma}
%\newtheorem{corollary}[theorem]{Corollary}
%\newtheorem{proposition}[theorem]{Proposition}
\newtheorem{claim}[theorem]{Claim}
%\newtheorem{remark}[theorem]{Remark}

%% \newcommand{\cproof}{\noindent{\it Proof of Claim}}


%% \newenvironment{claimproof}{\cproof. }{\hspace*{\fill}}

\newcommand{\size}[1]{{\ensuremath{\mathrm{size}\left(#1\right)}}}
\newcommand{\Gal}[1]{{\ensuremath{\mathrm{Gal}\left(#1\right)}}}
\newcommand{\Blocks}[1]{{\ensuremath{\mathcal{B}\left(#1\right)}}}
\newcommand{\Gof}[1]{{\ensuremath{\mathrm{G}\left(\scriptstyle #1\right)}}}

\newcommand{\Sym}[1]{{\ensuremath{\mathrm{Sym}\left(#1\right)}}}
\newcommand{\pr}[2]{{\ensuremath{\left.{#1}\right\vert_{#2}}}}

\newcommand{\abs}[1]{\ensuremath{\left\vert#1\right\vert}}

\newcommand{\norm}[1]{\ensuremath{\left\Vert#1\right\Vert}}

\newcommand{\Height}[1]{\ensuremath{\mathrm{H}\left(#1\right)}}

\newcommand{\Q}{\ensuremath{\mathbb{Q}}}
\newcommand{\F}{\ensuremath{\mathbb{F}}}
\newcommand{\Z}{\ensuremath{\mathbb{Z}}}



\title{Testing Nilpotence of Galois Groups in Polynomial Time
\footnote{preliminary version of this paper was presented at the MFCS
  2006 conference~\cite{AK2006}(see also~\cite{K2006})}.}


\author{V. ARVIND
  \affil{Institute of Mathematical Sciences,\\
    CIT Campus, Chennai, India 600113.\\
    {\tt arvind@imsc.res.in}.
  }
  PIYUSH P KURUR\footnote{Most of the work was done as a Ph.D student at the
    Institute of Mathematical Sciences.}
  \affil{Department of Computer Science and Engineering,\\
    Indian Institute of Technology, Kanpur,\\
    Kanpur, UP 208016, India.\\
    {\tt ppk@cse.iitk.ac.in}
  }
}

\begin{abstract}
  We give the first polynomial-time algorithm for checking whether the
  Galois group $\Gal{f}$ of an input polynomial $f(X) \in \Q[X]$ is
  nilpotent: the running time of our algorithm is bounded by a
  polynomial in the size of the coefficients of $f$ and the degree of
  $f$. Additionally, we give a deterministic polynomial-time algorithm
  that, when given as input a polynomial $f(X) \in \Q[X]$ with
  nilpotent Galois group, computes for each prime factor $p$ of $\#
  \Gal{f}$, a polynomial $g_p(X)\in \Q[X]$ whose Galois group of is
  the $p$-Sylow subgroup of $\Gal{f}$.
\end{abstract}

\category{F.2.1}{Numerical Algorithms and Problems}{Computations on
  polynomials}

\terms{Algorithms, Theory}

\keywords{Galois group, nilpotence, univariate polynomials, polynomial
  time}

%pagestyle{plain}
%\usepackage{showlabels}
\begin{document}
\maketitle


\section{Introduction}

Computing the Galois group of a polynomial $f(X)$ with rational
coefficients is a fundamental problem in algorithmic number
theory. For a polynomial $f(X)$ it is well known that the Galois group
$\Gal{f}$ acts faithfully as a permutation group on the roots of
$f(X)$. Thus, for a polynomial $f$ without repeated factors, we can
think of the Galois group of $f$ as a group of permutations on the set
of distinct roots of $f$. In this paper, by computing the Galois
group, we mean finding a generating set for the Galois group as a
permutation group. This characterizes the group only up to a
relabeling of the roots. However, this gives $\Gal{f}$ a compact
representation because any subgroup of $S_n$ has a generating set of
size at most $n-1$ \cite{Jerrum1986}. Furthermore, since there is a
substantial library of efficient algorithms for permutation groups
that takes as input subgroups of $S_n$ given by generating sets, this
compact representation of Galois groups can be computationally
useful. The book by Seress~\cite{seress2003permutationalgo} contains a
comprehensive treatment of permutation group algorithms.

There are algorithms for computing the Galois group of polynomials
over rationals that even go back to the nineteenth century
\cite{TS50}. However, no general polynomial-time algorithm for this
problem is known to date. Asymptotically, the best known algorithm is
due to Landau~\cite{landau84galois}: given a polynomial $f(X)$, as a
list of its coefficients in binary, it takes time polynomial in its
input size and the order of Galois group of $f$. Landau's algorithm
explicitly lists all elements of its Galois group $\Gal{f}$. However,
for a degree $n$ polynomial $f(X)$, $\Gal{f}$ can have $n!$ many
elements. Hence, Landau's algorithm takes exponential time in the
worst case. It is a long standing open problem if there is an
asymptotically faster algorithm. Lenstra's survey
\cite{lenstra92algorithm} discusses this and related problems.

In the absence of efficient asymptotic algorithms, considerable
research has gone into designing practical algorithms for Galois group
computation. The work of Stauduhar~\cite{St73} describes an algorithm
that could compute the Galois group of polynomials up to degree
8. Recent implementations with the computer algebra software KANT
apparently work up to degree 15. This method is described in detail by
Cohen~\cite{Co93}. Stauduhar's method has the drawback of precision
problems as it involves numerical approximation of the roots of the
polynomial. Instead of the numerical approximation, using $p$-adic
approximation Geissler and Kl\"uners~\cite{GK00} gave a variant of
Stauduhar's algorithm that could work for polynomials up to degree
15. Another well-studied practical method~\cite{SM85,MM97}, which
avoids the precision problems due to root approximations, works by
computing certain invariants called the absolute resolvents of the
given polynomial. However, this approach has the drawback of being
computationally more intensive as it involves factoring of very large
degree resolvent polynomials.  An entire special issue~\cite{jsc2000}
is devoted to algorithmic Galois theory, with practical
implementations as the main goal.

Although computing the Galois group of a polynomial remains hard in
general, often it is sufficient to check whether the Galois group
satisfies some specific property. Knowing the nature of the Galois
group of a polynomial can provide insight into the structure of the
roots of the polynomial. There is no better example than the
celebrated work of Galois \cite{galois1830analyse,galois1830note} in
which he shows that a polynomial $f(X)$ over rationals is solvable by
radicals if and only if its Galois group is solvable.

In this paper we study the problem of testing nilpotence of the Galois
group of a polynomial $f(X)$ over $\Q$. Landau's
algorithm~\cite{landau84galois} to compute the Galois group, although
exponential in general, yields efficient algorithms in certain
cases. For example, we can test whether the Galois group is
abelian~\cite{landau84galois} in polynomial time. We give a quick
overview. Assume that the input polynomial $f(X)$ is irreducible;
otherwise factor $f(X)$ by applying the LLL algorithm~\cite{lll} and
test whether the Galois group of each of its irreducible factors is
abelian. Since the Galois group of $f(X)$ is a subgroup of the direct
product of the Galois groups of each of its irreducible factors, this
is clearly sufficient. Any abelian transitive group of $S_n$ is of
order $n$. Hence, if the input polynomial is irreducible and abelian
then its Galois group is polynomially bounded. One can use Landau's
algorithm~\cite{landau84galois} and compute the Galois group
explicitly and verify that it is abelian. Similarly, for solvability
and nilpotence test, we can assume that the input polynomial is
irreducible. However, transitive solvable subgroups of $S_n$ can be of
size exponential in $n$. As far as efficient algorithms are concerned,
this does not give a satisfactory answer to the problem of checking
whether a given polynomial is solvable by radicals. Landau and Miller
made a remarkable breakthrough by giving a polynomial-time algorithm
for checking whether the Galois group of an input polynomial is
solvable\cite{landau85solvability}.

%% The key idea in their algorithm was to
%% check the solvability of $\Gal{f}$ by testing the solvability of some
%% small (i.e. polynomially bounded) extensions of $\Q$.

The class of nilpotent groups is a subclass of the class of solvable
groups that contains the class of abelian groups. Just as in the case
of solvable groups, transitive nilpotent subgroups of $S_n$ can have
order exponential in $n$. This rules out a nilpotence test which
explicitly computes the Galois group. Besides, the Landau-Miller
solvability test does not give a nilpotence test.  A key idea used in
the Landau-Miller algorithm is to reduce the problem of checking the
solvability of the Galois group $G$ of the input polynomial into
checking the the solvability of each factor group $G_{i-1}/G_i$ in
some special composition series $G = G_0 \rhd \ldots \rhd G_n = 1$ of
$G$. Despite each of the groups $G_i$ being large, Landau-Miller could
compute enough information of the factor groups $G_{i-1}/G_i$
implicitly from the input polynomial. However, one cannot infer
nilpotence of $G$, even if all the factor groups $G_{i-1}/G_i$ are
given explicitly. The simplest example that illustrates this are the
groups $S_3$ and $\Z_2\times \Z_3$. The factor groups are $\Z_2$ and
$\Z_3$ in the composition series for both the groups. However, $S_3$
is not nilpotent (though it is solvable) and $\Z_2\times \Z_3$ is
nilpotent (even cyclic). Thus, the composition factors a group alone
do not suffice to determine if it is solvable or nilpotent or even
abelian.

%To see why this is the case, notice that any solvable group $G$ has a
%composition series $G = G_0\rhd \ldots \rhd G_t =1$ such that the
%composition factors $G_{i-1}/G_i$, $0\leq i \leq t$, is a cyclic group
%of order $p_i$. On the other hand, consider any sequence of primes
%$p_1,\ldots,p_t$ and consider the product group $H = \prod_i
%\mathbb{Z}/p_i \mathbb{Z}$. The tower of groups $H_i = \prod_{k=i+1}^t
%\mathbb{Z}/p_k \mathbb{Z}$ give a composition series of $H$ such that
%$H_{i-1}/H_i$ is a cyclic group of order $p_i$.


\subsection*{Overview of our result}

We give the first deterministic polynomial-time algorithm for testing
whether the Galois group of an input polynomial $f(X) \in \Q[X]$ is
nilpotent. The running time of our algorithm is bounded by a
polynomial in $\size{f}$ and thus is polynomial in the input size.
Although our algorithm is polynomial time, it is unlikely to perform
well in practise as it involves factoring polynomials over number
fields. Testing nilpotence has been addressed before from the point of
view of developing practical algorithms. For example,
Fernandez-Ferreiros and Molleda~\cite{pilar2003deciding-nilpotence}
have given an algorithm for testing nilpotence by computing the centre
of the Galois group of $f$. A key step in their algorithm, based on
the Chebotarev density theorem, is to pick primes whose Frobenius give
elements in the centre of $\Gal{f}$.  However, the worst case running
time of this algorithm is polynomial in $\size{f}$ and the order $\#
\Gal{f}$ of the Galois group which is exponential in the input size.

We now give a brief overview of the main idea behind our algorithm.
The main observations that lead to the polynomial-time algorithm are
Theorems~\ref{thm-if-g-nilpotent} and \ref{thm-g-is-nilpotent}, which
together give a characterisation of transitive nilpotent permutation
groups in terms of its block structure. Explicitly testing for this
characterisation will require us to compute the Galois group and hence
is infeasible. As in the Landau-Miller solvability test, however, we
test this implicitly.

The normality of Sylow subgroups of nilpotent groups plays an
important role in the proof of Theorems~\ref{thm-if-g-nilpotent} and
\ref{thm-g-is-nilpotent}. As a byproduct of our main result we obtain
the following additional polynomial-time algorithm: given a polynomial
$f(X)\in \Q[X]$ with nilpotent Galois group, for each prime factor $p$
of $\# \Gal{f}$ we can efficiently compute a polynomial $g_p(X)$ such
that the Galois group $\Gal{g_p(X)}$ is the $p$-Sylow subgroup of
$\Gal{f}$.

\section{Galois theory overview}

In this section we recall some basic Galois theory. A detailed
presentation is available in a standard algebra textbook, like Lang's
book~\cite{lang:algebra}. Let $L$ and $K$ be fields.  We say that $L$
is an \emph{extension} of $K$ and denote it by $L/K$ if $L \supseteq
K$. For an extension $L/K$, $L$ is a vector space over $K$ and by the
\emph{degree} of $L/K$, denoted by $[L:K]$, we mean its dimension. An
extension $L/K$ is \emph{finite} if its degree $[L:K]$ is finite. If
$L/M$ and $M/K$ are finite extensions then $[L:K] = [L:M].[M:K]$. The
ring of polynomials over $K$ with indeterminate $X$ will be denoted by
$K[X]$. This ring is a \emph{unique factorisation domain}.


An element $\alpha$ in an extension $L$ of $K$ is \emph{algebraic} if
there is a polynomial $f(X)$ over $K$ such that $f(\alpha)=0$. For
such an $\alpha$ the \emph{minimal polynomial} over $K$ is the unique
monic polynomial $\mu_\alpha[K](X)$ over $K$ of least degree for which
$\alpha$ is a root. We write $\mu_\alpha(X)$ for $\mu_\alpha[K](X)$
when $K$ is understood. Elements $\alpha$ and $\beta$ in $L$ are
\emph{conjugates} over $K$ if they have the same minimal polynomial
over $K$. The smallest subfield of $L$ containing $K$ and $\alpha$ is
denoted by $K(\alpha)$. If $\alpha$ is algebraic over $K$ then the
field $K(\alpha)$ is isomorphic to the quotient $K[X]/\mu_\alpha[X]$.

Let $f(X)$ be a polynomial over $K$. By the \emph{splitting field} of
$f$ over $K$, denoted by $K_f$ we mean the smallest extension of $K$
containing all the roots of $f$. A finite extension $L/K$ is
\emph{normal} if for all irreducible polynomials $f(X)$ over $K$,
either $f(X)$ splits or has no root in $L$. Any finite normal
extension over $K$ is the splitting field of some polynomial in
$K[X]$. An extension $L/K$ is \emph{separable} if for all irreducible
polynomials $f(X) \in K[X]$ there are no multiple roots in $L$.  A
normal and separable finite extension $L/K$ is a \emph{Galois
  extension}.



The \emph{Galois group} of $L/K$, denoted by $\Gal{L/K}$, is the
subgroup of automorphisms $\sigma$ of $L$ that leaves $K$ fixed, i.e.\
$\sigma(\alpha) = \alpha$ for all $\alpha \in K$. The Galois group of
a polynomial $f(X)$ over $K$, denoted by $\Gal{f}$, is the Galois
group $\Gal{K_f/K}$ of its splitting field. For a subgroup $G$ of
automorphisms of $L$, the \emph{fixed field} $L^G$ is the largest
subfield of $L$ fixed by $G$.  We now state the fundamental theorem of
Galois theory~\cite[Theorem 1.1, Chapter VI]{lang:algebra}.

\begin{theorem}[Fundamental theorem of Galois theory] \label{thm-funda-galois}
  Let $L/K$ be a Galois extension with Galois group $G$. Let
  $\mathcal{F}$ be the set of fields $E$ between $L$ and $K$, i.e. $L
  \supseteq E \supseteq K$ and let $\mathcal{G}$ be the set of
  subgroups of $G$. Then, the maps $E \mapsto \Gal{L/E}$ and $H
  \mapsto L^H$ are inverses of each other and thus gives a one-to-one
  correspondence.  Furthermore, for any field $E \in \mathcal{F}$, the
  extension $E/K$ is Galois if and only if the corresponding Galois
  group $\Gal{L/E}$ is a normal subgroup of $G$. For the Galois
  extension $E/K$, $E\in \mathcal{F}$, the Galois group $\Gal{E/K}$ is
  (isomorphic to) the quotient group $G/\Gal{L/E}$.
\end{theorem}

A \emph{number field} is a finite extension of $\Q$. By the
\emph{degree of a number field} $K$ we mean $[K:\Q]$, i.e. the degree
of $K$ as an extension of $\Q$. An \emph{algebraic number} is a root
of a polynomial over $\Q$ and an \emph{algebraic integer} is an
algebraic number whose minimal polynomial has integral coefficients.
Let $K$ be a number field. A \emph{primitive element} $\eta$ of $K$ is
an algebraic number such that $K = \Q(\eta)$. A polynomial $\mu(X)$
over $\Q$ is a \emph{primitive polynomial} for $K$ if $K =
\Q[X]/\mu(X)$. By the primitive element theorem~\cite[Theorem 4.6,
  Chapter V]{lang:algebra} every number field has a primitive element.
Furthermore, we can assume that this primitive element is an algebraic
integer~\cite[Proposition 1.1, Chapter VII]{lang:algebra} (also refer
to \cite{waerden:1991}). Thus any number field is isomorphic to the
quotient $\Q[X]/\mu(X)$ where $\mu(X)$ is a monic irreducible
polynomial with integral coefficients.

\subsection{Input and Output Representations}

The inputs and outputs of the algorithms we describe in this paper are
objects like algebraic numbers, number fields, Galois groups etc. In
this section we discuss suitable encodings of these objects into
strings.  The complexity of our algorithms are measured in terms of
the sizes of these encodings. We use notation $\size{.}$ to denote
the number of bits required to represent an object.

Integers are encoded in binary and hence for an integer $n$, its size
is given by $\size{n} = \log{n}$. A rational number $r$ is encoded as
a pair of relatively prime integers $a$ and $b$ such that $r =
\frac{a}{b}$. The size of $r$ is therefore $O(\size{a} + \size{b})$.
A polynomial $T(X) = a_0 + \ldots + a_n X^n \in\Q[X]$ is given by a
list of its coefficients.  Thus, $\size{T}$ is defined as $O\left(n +
\sum_i \size{a_i}\right)$.


We now discuss how number fields are encoded. Recall that any number
field can be expressed as a quotient $\Q[X]/\mu(X)$ where $\mu(X)$ is
a primitive polynomial. We assume that a number field $K$ is
represented by giving a primitive polynomial $\mu(X)$ for it. In
addition we will assume that $\mu(X)$ is monic with integral
coefficients. Thus the size of $K$ under this representation is the
size of the polynomial $\mu(X)$. Notice that the size of $K$ depends
on the primitive polynomial chosen to represent $K$.

Let $K = \Q(\eta)$ be a number field of degree $n$ represented via a
primitive polynomial $\mu_\eta(X)$. Any algebraic number $\alpha$ in
$K$ can be expressed uniquely as $\alpha = A_\alpha(\eta)$ where
$A_\alpha(X)$ is a polynomial over $\Q$ of degree less than $n$.  By
$\size{\alpha}$ we mean $\size{A_\alpha(X)}$.  Again the size of
$\alpha$ depends on the chosen primitive element $\eta$ of $K$. For a
polynomial $f(X) = a_0 + \ldots + a_m X^m$ in $K[X]$ we define
$\size{f}$ to be $\sum \size{a_i}$.

Let $f(X)$ be a polynomial of degree $n$ over $\Q$. Landau's algorithm
\cite{landau84galois} for computing its Galois group $\Gal{f}$,
computes the entire multiplication table. Such an algorithm, in the
worst case, will take time exponential in $\size{f}$ as the order of
$\Gal{f}$ can be as large as $n!$. As mentioned in the introduction,
we can consider $\Gal{f}$ as a permutation group on the roots of $f$
and succinctly represent it by $n-1$ many generating permutations
\cite{luks93permutation}. This gives $\Gal{f}$ a representation of
size polynomial in $n$ and it makes sense to ask if $\Gal{f}$ in this
representation is computable in polynomial time. This is an open
problem.

%A drawback with this representation is that $\Gal{f}$ is determined
%only up to conjugacy as the roots of $f$ can be arbitrarily numbered.

%For example polynomial-time algorithms exists for computing the order,
%finding composition series, checking solvability
%etc~\cite{luks93permutation}. Representing $\Gal{f}$ as a permutation
%subgroup of $S_n$ is therefore reasonable from a computational point
%of view.

\subsection{An algorithm for computing primitive elements}

The proof of the primitive element theorem is algorithmic. Many
algorithms in computational number theory make use of this algorithmic
version. In order to keep the present paper self-contained, we give an
algorithmic proof for a version of the theorem suitable for our
purpose. We first recall a key lemma whose proof can be found in van
der Waerden's book \cite{waerden:1991}.

\begin{lemma}\label{lem-combine-fields}
  Let $\alpha$ and $\beta$ be algebraic numbers with conjugates
  $\alpha^{(i)}$, $1\leq i \leq m$, and $\beta^{(j)}$, $1 \leq j \leq
  n$ respectively. Let $c\in\Z$ such that $\alpha^{(i)} + c\beta^{(j)}
  \neq \alpha^{(r)} + c \beta^{(r)}$ for all $ (i,j) \neq (r,s)$ then
  $\Q(\alpha + c\beta) = \Q(\alpha,\beta)$. In particular, there is a
  positive integer $c\in\{1,2,\ldots,m^2n^2+1\}$ such that $\Q(\alpha
  + c\beta) = \Q(\alpha,\beta)$.
\end{lemma}

Next, we give an algorithm to compute minimal
polynomial~\cite{Shoup1999minimalpolynomial}.

\begin{lemma}\label{min-poly-find}
  Let $\alpha$ be an algebraic number with minimal polynomial
  $f(X)\in\Q[X]$. Given a polynomial $g(X)\in\Q[X]$ we can find the
  minimal polynomial for the element $g(\alpha)\in\Q(\alpha)$ in time
  polynomial in $\size{f}+\size{g}$.
\end{lemma}

\begin{proof}
  Let the degree of $f$ be $n$ and let $\beta = g(\alpha)$. Our task
  is to compute the minimal polynomial $\mu_\beta(X)$ of $\beta$. Let
  $m$ be the degree of the polynomial $\mu_\beta(X)$. Since $\beta$ is
  an element of $\Q(\alpha)$, the degree $m$ of its minimal polynomial
  is less than $n$. Furthermore, $m$ is the least integer $i$ less
  than $n$ such that the set of powers $\{ 1,\beta,\ldots,
  \beta^{i-1}, \beta^i \}$ is \emph{linearly dependent} as vectors
  over $\Q$. As $\beta = g(\alpha)$, the set $\{1, \ldots, \beta^i \}$
  is linearly dependent if and only if there are $a_j \in \Q$ such
  that
  \begin{equation}\label{eqn-lindep-beta}
    g^i(X)+\sum_{j=0}^{i-1} a_j g^j(X) = 0 \mod{f(X)}.
  \end{equation}
  Equating the coefficients of $X$ in equation~\ref{eqn-lindep-beta}
  gives us a system of linear equations in $a_j$'s, which can be
  checked for feasibility in time polynomial in $\size{f}+\size{g}$
  using Gaussian elimination for example. Starting with $i = 0$ we
  find the least $i$ for which equation~\ref{eqn-lindep-beta} is
  feasible. Having found the least $i$, which is also the degree $m$
  of $\mu_\beta(X)$, we solve for the unknowns $a_j$. Clearly
  $\mu_\beta(X)$ is the polynomial $X^m + a_{m-1} X^{m-1} + \ldots +
  a_1 X + a_0$.
\end{proof}

In the next lemma we prove an algorithmic version of the primitive
element theorem.

\begin{lemma}\label{lem-primitive-element}
  Let $\alpha$ be an algebraic number with minimal polynomial
  $f(X)\in\Q[X]$ of degree $n$. Let $\gamma_1,\ldots,\gamma_k$ be
  algebraic numbers in $\Q(\alpha)$ given as polynomials
  $\gamma_i=g_i(\alpha), 1\leq i\leq k$, and let
  $K=\Q(\gamma_1,\ldots,\gamma_k)$ be the subfield of $\Q(\alpha)$
  generated by $\gamma_1,\ldots,\gamma_k$. There is a deterministic
  algorithm with running time bounded by a polynomial in
  $\size{f}+\sum_{i=1}^k\size{g_i}$ that computes a polynomial
  $g(X)\in\Q[X]$ such that $g(\alpha)$ is a primitive element for $K$.
\end{lemma}

\begin{proof}
  Consider the tower of fields $\Q\subseteq
  \Q(\gamma_1)\subseteq\ldots \subseteq \Q(\gamma_1,\ldots,\gamma_k) =
  K$.  Our task is to compute a primitive element for $K$. For every
  $1\leq i \leq k$, we will compute polynomials $h_i(X)\in\Q[X]$ such
  that $\eta_i = h_i(\alpha)$ is a primitive element of the subfield
  $\Q(\gamma_1,\ldots,\gamma_i)$.

  For $i = 1$ choose $h_1(X) = g_1(X)$ and $\eta_1 = \gamma_1$.
  Inductively assume that we have computed $h_i(X)$ such that $\eta_i
  = h_i(\alpha)$ is a primitive element of the field
  $\Q(\gamma_1,\ldots,\gamma_i)$. Consider the field
  $\Q(\eta_i,\gamma_{i+1})$. As $\eta_i$ and $\gamma_{i+1}$ are
  elements of $\Q(\alpha)$, their degrees are less than $n$. By
  Lemma~\ref{lem-combine-fields} there is an integer
  $c_{i+1}\in\{1,\ldots,n^4+1\}$ such that $\eta_i +
  c_{i+1}\gamma_{i+1}$ is a primitive element for the field
  $\Q(\eta_i,\gamma_{i+1}) = \Q(\gamma_1,\ldots,\gamma_{i+1})$.

  We now explain how such a constant $c_{i+1}$ can be computed.  For a
  given $c$, to check whether $\eta_i + c \gamma_{i+1}$ is a primitive
  element of $\Q(\eta_i,\gamma_{i+1})$ it suffices to check whether
  there are polynomials $A(X)$ and $B(X)$ of degree at most $n$ such
  that $A(\eta_i + c\gamma_{i+1}) = \eta_i$ and $B(\eta_i + c
  \gamma_{i+1}) = \gamma_{i+1}$. This can be done by checking whether
  the following equations are feasible for unknowns $a_j$ and $b_j$.
  \begin{eqnarray*}
    \sum_{j=0}^{n-1} a_j (h_i(X)+ cg_{i+1}(X))^j & = & h_i(X) \mod{f(X)}\\
    \sum_{j=0}^{n-1} b_j (h_i(X)+ cg_{i+1}(X))^j & = & g_{i+1}(X) \mod{f(X)}
  \end{eqnarray*}

  Equating the coefficients of $X$, this involves checking feasibility
  of a system of linear equations over $\Q$ and thus can be done in
  time polynomial in $\size{h_i}+\size{g_{i+1}}+\size{f}$.  We go over
  all $1 \leq c \leq n^4 +1$ and let $c_{i+1}$ be the least $c$ for
  which the above equations are feasible. The required polynomial
  $h_{i+1}(X)$ is given by $h_{i+1}(X) = h_i(X) + c_{i+1}
  g_{i+1}(X)$. Finally, $h_k(\alpha)$ is a primitive element for
  $K$. The overall running time for the algorithm is bounded by a
  polynomial in the size of $\size{f}+\sum_{j=1}^k \size{g_j}$.
\end{proof}

\begin{remark}\label{rem-prim-elt}
  A similar algorithm can be designed that takes as input the minimal
  polynomials $g_i(X) \in \Q[X]$ of the algebraic numbers $\gamma_i$
  respectively, for $1 \leq i \leq k$ and computes a primitive element
  for the field $K=\Q(\gamma_1,\ldots,\gamma_k)$. The running time for
  this algorithm is polynomial in $[K:\Q]$ and the sizes of the
  minimal polynomials $g_i(X)$.
\end{remark}

\section{Previous complexity results}

We now state some of the known results in computational Galois theory
formally. The following result for computing the Galois group of a
polynomial $f(X)$ that runs in time polynomial in the size of the
Galois group and $f(X)$ is due to Landau~\cite{landau84galois}.

\begin{theorem}[Landau] \label{thm-landau-galois-algo} There is a
  deterministic algorithm that takes as input a number field $K$, a
  polynomial $f(X) \in K[X]$ and a positive integer $b$ in unary, and
  in time bounded by $\size{f}$, $\size{K}$ and $b$, decides if
  $\Gal{K_f/K}$ has at most $b$ elements, and if so computes
  $\Gal{K_f/K}$ by finding the entire multiplication table of
  $\Gal{K_f/K}$ (and hence also by giving the generating set of
  $\Gal{K_f/K}$ as a permutation group on the roots of $f(X)$).
\end{theorem}

The algorithm first computes a primitive element $\theta$ of $K_f$.
Determining $\Gal{f}$ amounts to finding the action of the
automorphisms on $\theta$. Subsequently, Landau and Miller
\cite{landau85solvability} gave their polynomial-time solvability
test.

\begin{theorem}[Landau-Miller]\label{thm-landau-solvability}
  There is a deterministic polynomial-time algorithm that takes as
  input a polynomial $f(X) \in \Q[X]$ and tests if the Galois group
  $\Gal{f}$ of $f$ is solvable.
\end{theorem}

A byproduct of the Landau-Miller algorithm is an algorithm to compute
the primes that divide the order of the Galois group. We summaries this
result for use latter on.

\begin{theorem}[Landau-Miller]\label{thm-landau-primes}
  There is a deterministic polynomial-time algorithm that takes as
  input a polynomial $f(X)$ over $\Q$ and, if $f(X)$ is solvable by
  radicals, computes the prime factors of $\# \Gal{f}$.
\end{theorem}

Thus one can also check in deterministic polynomial time whether the
Galois group is a $p$-group.

\section{Group theoretical Preliminaries}

We recall some group theory. Details can be found in Marshall Hall's
text \cite[Chapter 10]{Hall}. Let $G$ be any group. The \emph{lower
  central series} of $G$ is the sequence of groups $G = G_0 \geq G_1
\ldots \geq G_n \geq \ldots$ where $G_{i+1} = [G_i, G]$. A group $G$
is said to be \emph{nilpotent} if its lower central series is of
finite length, i.e.\ $G_c=\{1\}$ for some nonnegative integer $c$,
where $1$ denotes the identity element. The least $c$ such that
$G_c=\{1\}$ is called the \emph{class of nilpotence} of a nilpotent
group $G$. If the group $G$ is finite then it is nilpotent if and only
if all its $p$-Sylow subgroups are normal. It follows from Sylow's
theorem that a finite nilpotent group is a product of its $p$-Sylow
subgroups. It is this characterisation of finite nilpotent groups that
will be useful for us in our nilpotence test.

%For a nilpotent group $G$ is a normal series of length $c$, a positive
%integer.  The length $c$ of the lower central series is called the
%\emph{class of nilpotence} of the group $G$.

We recall some permutation group theory from Wielandt's
book~\cite{wielandt64finite}. Let $\Omega$ be a finite set. The
\emph{symmetric group} $\Sym{\Omega}$ is the group of all permutations
on $\Omega$.  By a \emph{permutation group on $\Omega$} we mean a
subgroup of $\Sym{\Omega}$. For $\alpha \in \Omega$ and $g \in
\Sym{\Omega}$, let $\alpha^g$ denote the image of $\alpha$ under the
permutation $g$. For $A \subseteq \Sym{\Omega}$, $\alpha^A$ denotes
the set $\{ \alpha^g : g \in A\}$. In particular, for
$G\leq\Sym{\Omega}$ the \emph{$G$-orbit} containing $\alpha$ is
$\alpha^G$. The $G$-orbits form a partition of $\Omega$.  Given
$G\leq\Sym{\Omega}$ by a generating set $A$ and $\alpha \in \Omega$,
there is a polynomial-time algorithm to compute
$\alpha^G$~\cite{luks93permutation}.

Sometimes we need to consider a more general group action on a set
$\Omega$. In the generalised setting, we say $G$ \emph{acts} on
$\Omega$ if there is a group homomorphism $\varphi: G\longrightarrow
\Sym{\Omega}$. The kernel $Ker(\varphi)$ of this action is the
subgroup of $G$ whose image under $\varphi$ is the identity element
(which pointwise fixes $\Omega$). If $Ker(\varphi)$ is trivial we say
that the action is \emph{faithful}. In this paper, when we say $G$ is
a permutation group on a set $\Omega$ we mean a faithful action unless
explicitly stated. The only exceptions arise when we restrict the
group $G$ to a subset of $\Omega$, typically an orbit or a block.
%When we consider a group $G$ acting on $\Omega$ with kernel $H$, the
%quotient group $G/H$ acts faithfully on $\Omega$ via the natural
%action $\alpha^{gH} = \alpha^g$. We will concentrate only on this
%faithful action, keeping in mind that what we say is applicable to the
%quotient rather than the whole group.

%This is not a problem for our needs in this article.

For $\Delta \subseteq \Omega$ and $g \in \Sym{\Omega}$, $\Delta^g$
denotes $\{ \alpha^g : \alpha \in \Delta \}$. The set-wise stabilizer
of $\Delta$, i.e. $\{ g \in G : \Delta^g = \Delta\}$, is denoted by
$G_\Delta$. If $\Delta$ is the singleton set $\{ \alpha \}$ we write
$G_\alpha$ instead of $G_{\{\alpha\}}$.   An often used result is the
orbit-stabilizer formula stated below~\cite[Theorem
  3.2]{wielandt64finite}.

\begin{theorem}[Orbit-stabilizer formula]\label{thm-orbit-stabilizer}
  Let $G\leq\Sym{\Omega}$ be a permutation group and let $\alpha$ be
  any element of $\Omega$ then the order of the group $G$ is given by
  $\# G = \# G_\alpha \cdot \# \alpha^G$.
\end{theorem}

A permutation group $G\leq\Sym{\Omega}$ is \emph{transitive} if there
is a single $G$-orbit. Suppose $G \leq \Sym{\Omega}$ is transitive.
Then a non-empty subset $\Delta$ of $\Omega$ is a \emph{$G$-block} if
for all $g \in G$ either $\Delta^g = \Delta$ or $\Delta^g \cap
\Delta=\emptyset$.  For every $G$, $\Omega$ is a block and each
singleton $\{\alpha\}$ is a block. These are the \emph{trivial blocks}
of $G$. A transitive group $G$ is \emph{primitive} if it has only
trivial blocks and it is \emph{imprimitive} if it has nontrivial
blocks. We state a useful proposition that is easy to prove from these
definitions.

\begin{proposition}\label{prop-block-subblock}
  Let $G\leq \Sym{\Omega}$ be a transitive permutation group.
\begin{itemize}
\item[(a)] If $\Delta\subset\Omega$ is a $G$-block then $G_\Delta$ is
  transitive on $\Delta$. I.e.\ for $\alpha,\beta\in\Delta$ there is a
  $g\in G_\Delta$ such that $\alpha^g=\beta$.
\item[(b)] Let $\Sigma\subset\Omega$ be a $G$-block. Then
  $\Delta\subset \Sigma$ is a $G$-block if and only if $\Delta$ is a
  $G_\Sigma$-block.
\end{itemize}
\end{proposition}

A $G$-block $\Delta$ is a \emph{maximal subblock} of a $G$-block
$\Sigma$ if $\Delta \subset \Sigma$ and there is no $G$-block
$\Upsilon$ such that $\Delta \subset \Upsilon \subset \Omega$. Let
$\Delta$ and $\Sigma$ be two $G$-blocks. A chain $\Delta = \Delta_0
\subset \ldots \subset \Delta_t = \Sigma$ is a \emph{maximal chain} of
$G$-blocks between $\Delta$ and $\Sigma$ if for all $i$, $\Delta_i$ is
a maximal subblock of $\Delta_{i+1}$.

For a $G$-block $\Delta$ and $g \in G$, $\Delta^g$ is also a $G$-block
and $\# \Delta = \# \Delta^g$. Let $\Delta$ and $\Sigma$ be two
$G$-blocks such that $\Delta \subseteq \Sigma$.  The
\emph{$\Delta$-block system of $\Sigma$}, is the collection
\[
\Blocks{\Sigma/\Delta} = \{ \Delta^g : g \in G \textrm{ and } \Delta^g
\subseteq \Sigma \}.
\]
The set $\Blocks{\Sigma/\Delta}$ is a partition of $\Sigma$.  It
follows that $\# \Delta$ divides $\# \Sigma$ and by \emph{index} of
$\Delta$ in $\Sigma$, which we denote by $[\Sigma:\Delta]$, we mean
$\# \Blocks{\Sigma/\Delta} = \frac{\# \Sigma}{\# \Delta}$. We will use
$\Blocks{\Delta}$ to denote $\Blocks{\Omega/\Delta}$. We state the
connection between blocks and subgroups \cite[Theorem
7.5]{wielandt64finite}.

\begin{theorem}[Galois correspondence of blocks]\label{thm-blocks-galois}
  Let $G\leq\Sym{\Omega}$ be transitive and $\alpha\in\Omega$. For $G
  \geq H \geq G_\alpha$ the orbit $\Delta = \alpha^H$ is a $G$-block
  and $G_\Delta = H$. The correspondence $\alpha^H = \Delta
  \rightleftharpoons G_\Delta = H$ is a one-to-one correspondence
  between $G$-blocks $\Delta$ containing $\alpha$ and subgroups $H$ of
  $G$ containing $G_\alpha$. Furthermore for $G$-blocks $\Delta
  \subseteq \Sigma$ we have $[G_\Sigma : G_\Delta] = [\Sigma :
  \Delta]$.
\end{theorem}

Let $G\leq\Sym{\Omega}$ be transitive and $\Delta$ and $\Sigma$ be two
$G$-blocks such that $\Delta\subseteq \Sigma$. Let
$\Gof{\Sigma/\Delta}$ denote the group $\{ g \in G : \Upsilon^g =
\Upsilon \textrm{ for all } \Upsilon \in \Blocks{\Sigma/\Delta} \}$.
We write $G^\Delta$ for the group $\Gof{\Omega/\Delta}$.  The next
three lemmas are well known in the permutation group theory community,
however we prove them here for completeness and to fix the notation.

\begin{lemma}\label{lem-gofsigma}
  Let $G\leq\Sym{\Omega}$ be a permutation group and let $\Delta$ and
  $\Sigma$ be two $G$-blocks such that $\Sigma \supseteq \Delta$. Then
  \begin{enumerate}
  \item The group $\Gof{\Sigma/\Delta}$ is the largest normal subgroup
    of $G_\Sigma$ contained in $G_\Delta$. In particular, $G^\Delta$
    is a normal subgroup of $G$.

  \item The quotient group ${G_\Sigma}/{\Gof{\Sigma/\Delta}}$ is a
    faithful permutation group on $\Blocks{\Sigma/\Delta}$ and is
    primitive when $\Delta$ is a maximal subblock.
  \end{enumerate}
\end{lemma}

\begin{proof}
  For any $g \in G_\Sigma$, since $g$ set-wise stabilises $\Sigma$,
  $g$ permutes the elements of $\Blocks{\Sigma/\Delta}$. Hence for any
  $\Upsilon \in \Blocks{\Sigma/\Delta}$ we have
  $\Upsilon^{g^{-1}\Gof{\Sigma/\Delta}g} = \Upsilon$. Thus,
  $\Gof{\Sigma/\Delta}$ is a normal subgroup of $G_\Sigma$.

  Now consider any $N\subseteq G_\Delta$ which is a normal subgroup of
  $G_\Sigma$. Since $\Delta^N = \Delta$, and since $G_\Sigma$ acts
  transitively on $\Blocks{\Sigma/\Delta}$, for any $\Upsilon \in
  \Blocks{\Sigma/\Delta}$ there is a $g \in G_\Sigma$ such that
  $\Upsilon = \Delta^g$. Therefore, $\Upsilon^N = \Delta^{gN} =
  \Delta^{Ng}= \Upsilon$ for each $\Upsilon \in
  \Blocks{\Sigma/\Delta}$. Thus $N\subseteq \Gof{\Sigma/\Delta}$.
  Since $\Gof{\Sigma/\Delta}\unlhd G_\Sigma$ we have proved part 1.

  Consider the action of $G_\Sigma$ on
  $\Blocks{\Sigma/\Delta}$. Clearly, $\Gof{\Sigma/\Delta}$ is the
  kernel in this action. Therefore $G_\Sigma/\Gof{\Sigma/\Delta}$ acts
  faithfully on $\Blocks{\Sigma/\Delta}$. Notice that $G_\Sigma$ is
  transitive on $\Blocks{\Sigma/\Delta}$ as it is transitive on
  $\Sigma$. Further it can be easily verified that nontrivial
  $G_\Sigma/\Gof{\Sigma/\Delta}$-blocks of $\Blocks{\Sigma/\Delta}$
  are in 1-1 correspondence with the $G$-blocks $\Gamma$ such that
  $\Delta\subset \Gamma \subset\Sigma$. Thus,
  $G_\Sigma/\Gof{\Sigma/\Delta}$ is primitive if and only if $\Delta$
  is a maximal subblock of $\Sigma$.

\end{proof}


In our algorithms, we often need to check certain properties of the
groups $G_\Delta$ and $G^\Delta$. The group $G$ in this context is the
Galois group of the input polynomial $f$ and $\Delta$ is a $G$-block
on its action on the roots of $f$. However, explicit computation of
these groups are impossible in polynomial time. The next lemma helps
us reduce this problem to the study of certain natural quotient
groups.


\begin{lemma}\label{lem-gsupdelta}
  Let $G\leq\Sym{\Omega}$ be a permutation group. Let $\Delta$ and
  $\Sigma$ be two $G$-block such that $\Delta \subseteq \Sigma$. Then
  the quotient group $G^\Sigma/G^\Delta$ can be embedded \emph{into}
  the product group $\left( G_\Sigma/\Gof{\Sigma/\Delta}\right)^l$ for
  some positive integer $l$.
\end{lemma}
\begin{proof}

  For the proof, let the $\Sigma$-block system $\Blocks{\Sigma}$ be
  $\{\Sigma_1,\ldots,\Sigma_m\}$ where $\Sigma = \Sigma_1$. Notice
  that $G^{\Sigma_i} = G^{\Sigma} = G^{\Sigma_j}$ for all $1 \leq i
  \leq j \leq m$. Let $\Delta=\Delta_1,\ldots,\Delta_m$ be any $m$
  elements of the $\Delta$-block system $\Blocks{\Delta}$ such that
  $\Delta_i \subseteq \Sigma_i$.

  Consider the action of $G^\Sigma$ on $\Blocks{\Delta}$.  Clearly the
  kernel of this action is $G^\Delta$. Therefore, the quotient group
  $H=G^\Sigma/G^\Delta$ acts faithfully on $\Blocks{\Delta}$. Notice
  that the orbits under the action are precisely
  $\Blocks{\Sigma_i/\Delta_i}$ for $1 \leq i \leq m$. It is thus easy
  to see that $H$ can be embedded in the product $\prod H_i$ where
  $H_i$ is the restriction of $H$ on to
  $\Blocks{\Sigma_i/\Delta_i}$. Notice that $G^\Sigma \subseteq
  G_{\Sigma_i}$ for all $i$ and the kernel of the action of
  $G_{\Sigma_i}$ on $\Blocks{\Sigma_i/\Delta_i}$ is
  $\Gof{\Sigma_i/\Delta_i}$ (Lemma~\ref{lem-gofsigma}). Therefore,
  $G^\Sigma/G^\Delta$ can be embedded \emph{into} a subgroup of
  the product group $\prod_i
  G_{\Sigma_i}/\Gof{\Sigma_i/\Delta_i}$. The lemma then follows from
  the fact that $G_{\Sigma_i}$ is isomorphic to $G_\Sigma$ (in fact
  $G_{\Sigma_i}$ and $G_\Sigma$ are $G$-conjugates) and
  $\Gof{\Sigma/\Delta}$ is isomorphic to $\Gof{\Sigma_i/\Delta_i}$.

\end{proof}

The next lemma connects orbits of normal subgroups and blocks.

\begin{lemma}\label{lem-orbit-normal}
  Let $G\leq\Sym{\Omega}$ be transitive and $N\unlhd G$. Let
  $\alpha\in\Omega$. Then the $N$-orbit $\alpha^N$ is a $G$-block and
  the collection of $N$-orbits is an $\alpha^N$-block system of
  $\Omega$ under $G$ action. If $N\neq\{ 1 \}$ then $\# \alpha^N>1$.
  Furthermore, if $G_\alpha\leq N\neq G$ then the $\alpha^N$-block
  system is nontrivial implying that $G$ is not primitive.
\end{lemma}
\begin{proof}
  Let $\alpha\in\Omega$ and $g \in G$. If $\beta = \alpha^g$ then the
  set $(\alpha^N)^g = \alpha^{Ng} = \alpha^{gN} = \beta^N$. Thus
  $(\alpha^N)^g$ and $\alpha^N$ are $N$-orbits, and hence are
  identical or disjoint.  So, $\alpha^N$ is a $G$-block and the
  $N$-orbits form the $\alpha^N$-block system of $G$.  If $\# \alpha^N
  = 1$ then all the $N$ orbits are of cardinality $1$ and hence $N$
  fixes all element of $\Omega$. This is possible if and only if $N =
  1$.

  Finally, suppose that $G_\alpha \leq N$. Then by the
  Orbit-Stabilizer formula (Theorem~\ref{thm-orbit-stabilizer}) $\# G
  =\#\Omega\cdot\# G_\alpha$ and $\# N=\#\alpha^N\cdot\# G_\alpha$.
  Thus, if $\{1\}\neq N\neq G$ and $G_\alpha \leq N$ then $1 < \#
  \alpha^N < \# \Omega$ and hence $\alpha^N$ is a nontrivial
  $G$-block.
\end{proof}

\section{Nilpotent permutation groups}\label{sec-nilpotent-perm-group}

In this section, we prove some properties of transitive nilpotent
groups that will be required for our nilpotence test. Recall that a
finite group $G$ is nilpotent if and only if for all prime factors $p$
of $\# G$, there is a unique normal $p$-Sylow subgroup for $G$. Let
$G_p$ denote this unique $p$-Sylow subgroup of $G$. Any orbit of the
Sylow subgroup $G_p$ is a $G$-block as $G_p$ is normal in $G$
(Lemma~\ref{lem-orbit-normal}). These blocks play a crucial role in
our nilpotence test and hence we give them a name.

\begin{definition}[Sylow blocks]\label{def-sylow-block}
  Let $G$ be a transitive nilpotent permutation group on $\Omega$ and
  let $p$ be a prime dividing the order of $G$.  A subset of $\Omega$
  is called a \emph{$p$-Sylow block} of $G$ if it is an orbit of the
  $p$-Sylow subgroup $G_p$.
\end{definition}

We prove the following lemma about Sylow blocks.

\begin{lemma}\label{lem-sylow-block-ppower}
  Let $G\leq\Sym{\Omega}$ be a transitive nilpotent permutation group.
  For every prime $p$ that divides $\# G$ any $p$-Sylow block is of
  cardinality $p^l$ for some $l >0$.
\end{lemma}
\begin{proof}
  Let $\Sigma \subseteq \Omega$ be any $p$-Sylow block. Since $\Sigma$
  is an orbit of a nontrivial normal subgroup $G_p$ of $G$, by
  Lemma~\ref{lem-orbit-normal}, we have $\# \Sigma > 1$. Furthermore
  if $\alpha$ is any element in $\Sigma$, since $\Sigma =
  \alpha^{G_p}$ by the orbit-stabiliser formula
  (Theorem~\ref{thm-orbit-stabilizer}) we have $\# \Sigma \cdot \#
  (G_{p})_{\alpha} = \# G_p$. Hence $\# \Sigma$ is a power of $p$. So,
  $\# \Sigma = p^l$ for some $l >0$.
\end{proof}

We now prove the following lemma about the cardinality of Sylow
blocks.

\begin{lemma}\label{lem-sylow-cardinality}
  Let $G\leq\Sym{\Omega}$ be a transitive nilpotent permutation group.
  For any prime $p$, $p$ divides $\# G$ if and only if it divides $\#
  \Omega$, and for any $p$-Sylow block $\Sigma$ of $G$, $\# \Sigma$ is
  the highest power of $p$ that divides $\# \Omega$.
\end{lemma}
\begin{proof}
  Consider any $\alpha \in \Omega$. Since $G$ is transitive, we known
  that $\alpha^G = \Omega$. By the orbit-stabiliser theorem
  (Theorem~\ref{thm-orbit-stabilizer}) we have $\# \Omega \cdot \#
  G_\alpha = \# G$. Therefore, every prime factor of $\# \Omega$
  divides $\# G$. Conversely, if $p$ divides $\# G$ then for any
  $p$-Sylow block $\Sigma$, $p$ divides $\# \Sigma$ by
  Lemma~\ref{lem-sylow-block-ppower}. Hence $p$ divides $\# \Omega$ (a
  consequence of Theorem~\ref{thm-blocks-galois} for blocks $\Omega$
  and $\Sigma$).

  We now prove that the cardinality of the $p$-Sylow block $\Sigma$ is
  the highest power of $p$ that divides $\# \Omega$. Consider the
  blocks $\Omega$ and $\Sigma$. By Theorem~\ref{thm-blocks-galois} we
  have $[\Omega: \Sigma] = [G: G_\Sigma]$. Since $\Sigma$ is a $G_p$
  orbit we have $G_p \leq G_\Sigma$ and hence $p$ does not divide
  $[G:G_\Sigma]$. So, $p$ does not divide $[\Omega:\Sigma] =
  \frac{\# \Omega}{\# \Sigma}$.
\end{proof}

As mentioned before, the Sylow blocks play an important role in our
algorithm. The fact that any Sylow block $\Sigma$ has prime power
cardinality helps us in studying the Sylow subgroups of $G_\Sigma$. We
prove the following lemma about any block of prime power cardinality.

\begin{lemma}\label{lem-same-sylow}
  Let $G\leq\Sym{\Omega}$ be a transitive nilpotent permutation group.
  Let $\Delta$ be any $G$-block such that $\# \Delta$ is a power of a
  prime $p$ and let $q\neq p$ be another prime dividing $\# G$. Let
  $G_{\Delta,q}$ denote the $q$-Sylow subgroup of $G_\Delta$. Then
  $G_{\Delta,q}$ fixes all points of the block $\Delta$ (i.e. for all
  $g$ in $G_{\Delta,q}$ and $\alpha$ in $\Delta$, $\alpha^g =
  \alpha$). As a result the $q$-Sylow subgroups $G_{\Delta,q}$ and
  $G_{\alpha,q}$ are equal.
\end{lemma}
\begin{proof}
  Let $\alpha\in\Delta$. Consider the blocks $\Delta$ and $\{\alpha
  \}$. By Theorem~\ref{thm-blocks-galois} we have $G_\alpha \leq
  G_\Delta$, and the index $[G_\Delta: G_\alpha] = \# \Delta$ is a
  power of $p$. Consequently, for a prime $q\neq p$ the highest power
  of $q$ that divides $\# G_\alpha$ and $\# G_\Delta$ are same, say
  $q^r$. Further, note that both $G_\Delta$ and $G_\alpha$ are
  nilpotent as they are subgroups of a nilpotent group $G$. Therefore,
  they have unique $q$-Sylow subgroups $G_{\Delta,q}$ and
  $G_{\alpha,q}$, respectively which must be of size $q^r$. Hence
  $G_{\Delta,q}=G_{\alpha,q}$, implying that $\alpha^g=\alpha$ for all
  $g\in G_{\alpha,q}$.
\end{proof}

We derive an important consequence of Lemma~\ref{lem-same-sylow}.

\begin{lemma}\label{lem-gsup-sylow}
  Let $G\leq\Sym{\Omega}$ be a transitive nilpotent permutation group
  and $\Sigma$ be any $p$-Sylow block of $G$. The group $G^{\Sigma}$
  is the (unique) $p$-Sylow subgroup $G_p$ of $G$.
\end{lemma}
\begin{proof}
  Recall that $\Sigma$ is an orbit of $G_p$ and the $\Sigma$-block
  system $\Blocks{\Sigma}$ is the collection of all $G_p$-orbits and
  hence all $p$-Sylow blocks of $G$. Therefore, for all $\Gamma \in
  \Blocks{\Sigma}$, we have $\Gamma^{G_p} = \Gamma$ and hence $G_p
  \leq G^{\Sigma}$. The group $G^\Sigma$ being a subgroup of a
  nilpotent group $G$ is itself nilpotent and is therefore the product
  of its Sylow subgroups. So, to prove that $G^\Sigma = G_p$ it is
  sufficient to prove that for all primes $q\neq p$ the $q$-Sylow
  subgroup $G^\Sigma_q$ of $G^{\Sigma}$ is trivial.

  Note that $G^\Sigma=\bigcap_{\Gamma \in \Blocks{\Sigma}}
  G_\Gamma$. Hence, $G^\Sigma_q \leq G_{\Gamma,q}$ for all $\Gamma \in
  \Blocks{\Sigma}$ and by Lemma~\ref{lem-same-sylow} it follows that
  for all $g$ in $G^\Sigma_q$ and $\alpha\in\Gamma$ we have
  $\alpha^g=\alpha$. Since $\bigcup_{\Gamma\in \Blocks{\Sigma}} \Gamma
  = \Omega$, for all $\alpha$ in $\Omega$ and $g$ in $G^\Sigma_q$,
  $\alpha^g=\alpha$. This is only possible if $G^\Sigma_q$ is the
  trivial group $\{1\}$.
\end{proof}

We now show that the subblock structure of $G$ under a $p$-Sylow block
is similar to the subblock structure of a transitive $p$-group.


\begin{lemma}\label{lem-sylow-subblocks}
  Let $G\leq\Sym{\Omega}$ be a transitive nilpotent permutation group
  and let $p$ be a prime factor of $\# G$. Let $\Sigma$ be any
  $p$-Sylow block of $G$ then for any subset $\Delta \subseteq
  \Sigma$, $\Delta$ is a $G$-block if and only if $\Delta$ is
  $G_p$-block under the transitive action of $G_p$ on $\Sigma$.
\end{lemma}
\begin{proof}
  Clearly if $\Delta \subseteq \Sigma$ is a $G$-block then it is also
  a $G_p$-block as $G_p \leq G$.

  Conversely, suppose $\Delta$ is a $G_p$-block. We first argue that
  $\Delta$ is a $G_\Sigma$-block. Recall that $G_p$ is the $p$-Sylow
  subgroup of $G_\Sigma$ as well, and by Lemma~\ref{lem-same-sylow}
  for a prime $q\neq p$ the $q$-Sylow subgroups of $G_\Sigma$
  pointwise fix each element of $\Sigma$ (and hence each element of
  $\Delta$). Since $G_\Sigma$ is a product of its Sylow subgroups, it
  follows that $\Delta$ is a $G_\Sigma$-block.  Hence, by
  Proposition~\ref{prop-block-subblock} $\Delta$ is a $G$-block as
  well.
\end{proof}

The previous lemma indicates that to study the subblock structure
under a $p$-Sylow block it is sufficient to understand the subblock
structure of a transitive $p$-group.  We now recall a result about
blocks of transitive $p$-groups (see e.g.\ Luks~\cite[Lemma
1.1]{luks82bounded}).

\begin{lemma}\label{lem-luks-pgroups}
  Let $G\leq\Sym{\Omega}$ be a transitive permutation $p$-group and
  $\Delta$ be a maximal $G$-block.  Then $[\Omega : \Delta]=p$ and the
  group $G_\Delta$ is a normal subgroup of $G$ with the quotient group
  $G/G_\Delta$ being the cyclic group of order $p$.
\end{lemma}

The above lemma has the following corollary.

\begin{corollary}\label{cor-luks-pgroups}
  Let $G$ be a transitive permutation $p$-group on $\Omega$ and let
  $\Delta$ be any $G$-block. Suppose that $\Gamma$ is a minimal
  $G$-block containing $\Delta$ then $[\Gamma : \Delta] = p$ and
  $G_\Delta$ is a normal subgroup of $G_\Gamma$ with quotient
  $G_\Gamma/G_\Delta$ a cyclic group of order $p$.
\end{corollary}
\begin{proof}
  Consider the transitive action of $G_\Gamma$ on $\Gamma$. If
  $\Delta'$ is any $G_\Gamma$-block between $\Gamma$ and $\Delta$ then
  $G_\Delta \leq G_{\Delta'} \leq G_\Gamma$. Hence by
  Theorem~\ref{thm-blocks-galois}, $\Delta'$ is a $G$-block.  This
  contradicts the minimality of $\Gamma$. So, $\Delta$ is a maximal
  $G_\Gamma$ block of $\Gamma$. The corollary then follows from
  Lemma~\ref{lem-luks-pgroups} for the group $G_\Gamma$.
\end{proof}

We now translate the above result on the block structure of a
$p$-group into that of a nilpotent group.

\begin{theorem}\label{thm-if-g-nilpotent}
  Let $G$ be any transitive nilpotent group on $\Omega$ and let $p$ be
  a prime dividing $\# G$ (and hence $\# \Omega$). Let $p^m$ be the
  highest power of $p$ that divides $\# \Omega$ and let $\Delta$ be a
  $G$-block of cardinality a power of $p$. Then there is a $p$-Sylow
  block $\Sigma$ such that $\Delta \subseteq \Sigma$. Furthermore, if
  $\# \Delta < p^m$ then for any minimal $G$-block $\Delta'$ such that
  $\Delta \subset \Delta' \subseteq \Sigma$ we have:
  \begin{enumerate}
  \item The index of the blocks $[\Delta':\Delta]$ is $p$,
  \item The group $G_{\Delta}$ is a normal subgroup of $G_{\Delta'}$ and
  \item The quotient group $G_{\Delta'}/G_\Delta$ is the cyclic group
  of order $p$.
  \end{enumerate}
\end{theorem}
\begin{proof}
  First we show that $\Delta$ is a subset of a $p$-Sylow block.  Let
  $\alpha$ be any element of $\Delta$ and let $\Sigma$ be the
  $p$-Sylow block $\alpha^{G_p}$. We claim that $\Delta\subseteq
  \Sigma$. By Proposition~\ref{prop-block-subblock},
  $\Delta=\alpha^{G_\Delta}$. By Lemma~\ref{lem-same-sylow}, for any
  prime $q\neq p$ the $q$-Sylow subgroup $G_{\Delta,q}$ of $G_\Delta$
  fixes each point of $\Delta$. Since $G_\Delta$ is the product of its
  Sylow subgroups, it follows that
  $\Delta=\alpha^{G_\Delta}=\alpha^{G_{\Delta,p}}\subseteq
  \alpha^{G_p}=\Sigma$.

  Now consider any minimal $G$-block $\Delta'$ between $\Sigma$ and
  $\Delta$. By Lemma~\ref{lem-sylow-subblocks}, $\Delta'$ is a minimal
  $G_p$-block between $\Delta$ and $\Sigma$. Therefore, using
  Corollary~\ref{cor-luks-pgroups}, we have $[\Delta':\Delta]=p$ and
  $G_{\Delta,p}$ is a normal subgroup of $G_{\Delta',p}$ such that
  their quotient group $G_{\Delta',p}/G_{\Delta,p}$ is a cyclic group
  of order $p$. For all primes $q$ different from $p$ by
  Lemma~\ref{lem-same-sylow} we have $G_{\Delta',q} =
  G_{\Delta,q}$. Since $G_{\Delta'}$ and $G_\Delta$ are a product of
  their Sylow subgroups we have $G_\Delta$ is a normal subgroup of
  $G_{\Delta'}$ with the quotient $G_{\Delta'}/G_{\Delta} =
  G_{\Delta',p}/G_{\Delta,p}$, a cyclic group of order $p$.
\end{proof}

\begin{theorem}\label{thm-g-is-nilpotent}
  Let $G$ be a transitive permutation group on $\Omega$. Let $\alpha$
  be any element of $\Omega$. Suppose that for all primes $p$ dividing
  $\# G$ we have a chain $\{ \alpha \} = \Delta_0 \subset \ldots
  \subset \Delta_m$ of $G$-blocks satisfying the follow properties
  \begin{enumerate}
  \item The index $[\Delta_{i+1}:\Delta_i] =
    p$,\label{item-p-cardinality}
  \item The group $G_{\Delta_i}$ is a normal subgroup of
    $G_{\Delta_{i+1}}$ and\label{item-normality}
  \item The prime $p$ does not divide the order of
    $G/G^{\Delta_m}$.\label{item-p-not-dividing}
  \end{enumerate}
  Then $G$ is nilpotent.
\end{theorem}

\begin{proof}
  For each prime factor $p$ of $\# G$ we will show that any $p$-Sylow
  subgroup of $G$ is normal.  Since $G^{\Delta_m}$ is normal in $G$
  (by Lemma[Part 1]~\ref{lem-gofsigma}) and by
  part~\ref{item-p-not-dividing}, $p$ does not divide
  $\#G/G^{\Delta_m}$, it is sufficient to prove that $G^{\Delta_m}$ is
  a $p$-group. We prove inductively that for all $0 \leq i \leq m$ the
  group $G^{\Delta_i}$ is of cardinality $p^{l_i}$ for some $l_i$. As
  the base case, $G^{\Delta_0} = \{ 1 \}$ and hence $\# G^{\Delta_0} =
  p^{l_0}$ where $l_0 = 0$.

  Suppose that our hypothesis is true for all $i \leq r$. To prove
  that $\# G^{\Delta_{r+1}}$ is $p^{l_{r+1}}$ for some $l_{r+1}$ it is
  sufficient to prove that the quotient group
  $G^{\Delta_{r+1}}/G^{\Delta_r}$ is a $p$-group since by the
  inductive assumption $\# G^{\Delta_r}= p^{l_r}$. {From}
  Lemma~\ref{lem-gsupdelta} the quotient group
  $G^{\Delta_{r+1}}/G^{\Delta_r}$ is a subgroup of
  $\left(G_{\Delta_{r+1}}/\Gof{\Delta_{r+1}/\Delta_r}\right)^l$ for
  some integer $l$. We will prove that
  $G_{\Delta_{r+1}}/\Gof{\Delta_{r+1}/\Delta_r}$ is a cyclic group of
  order $p$ which clearly is sufficient.

  The group $\Gof{\Delta_{r+1}/\Delta_r}$ is the largest subgroup of
  $G_{\Delta_r}$ that is normal in $G_{\Delta_{r+1}}$
  (Lemma~\ref{lem-gofsigma}). By part~\ref{item-normality},
  $G_{\Delta_r}$ itself is normal in $G_{\Delta_{r+1}}$. Hence the
  groups $\Gof{\Delta_{r+1}/\Delta_r}$ and $G_{\Delta_r}$ are
  equal. Furthermore, $[G_{\Delta_{r+1}}:G_{\Delta_r}] =
  [\Delta_{r+1}:\Delta_r] = p$ (part~\ref{item-p-cardinality}). So,
  the quotient group $G_{\Delta_{r+1}}/\Gof{\Delta_{r+1}/\Delta_r}$,
  which is $G_{\Delta_{r+1}}/G_{\Delta_r}$, is a cyclic group of order
  $p$.
\end{proof}



\begin{remark}
  Both Theorems \ref{thm-if-g-nilpotent} and \ref{thm-g-is-nilpotent}
  play an important role in our algorithm described in the next
  section.  Theorem~\ref{thm-if-g-nilpotent} guarantees for nilpotent
  groups that each chain of $G$-blocks (whose sizes are a power of
  $p$) can be extended to a maximal chain that terminates at a
  $p$-Sylow block and any pair of adjacent blocks have index $p$. This
  allows the algorithm to grow the chain of blocks in any manner.
  Theorem~\ref{thm-g-is-nilpotent} ensures the correctness.
\end{remark}

\section{The Polynomial-Time Nilpotence Test}\label{sec-nilpotence-test}

In this section, our goal is to give an algorithm that takes as input
a polynomial $f(X)$ over $\Q$ and checks whether its Galois group
$\Gal{f}$ is nilpotent. Now, $\Gal{f}$ is nilpotent if and only if for
each irreducible factor $h(X)$ of $f(X)$, the Galois group $\Gal{h}$
is nilpotent. This is true because nilpotent groups are closed under
subgroups, products and quotients. Hence, in order to test the
nilpotence of $Gal{f}$ it suffices to check for each irreducible
factor $h$ of $f$ that its Galois group $\Gal{h}$ is
nilpotent. Furthermore, using the LLL algorithm~\cite{lll}, all the
irreducible factors of $f(X)$ can be computed in polynomial
time. Therefore, for nilpotence testing, we assume without loss of
generality that the input polynomial $f(X)$ is irreducible.

Let $G$ be $\Gal{f}$. We consider $G$ as a subgroup of $\Sym{\Omega}$,
where $\Omega$ is the set of roots of $f(X)$. Since $f$ is irreducible
all its roots are distinct, and for any two roots $\alpha$ and $\beta$
of $f$ there is an element $\sigma$ in the Galois group $G$ of $f$
such $\alpha^\sigma = \beta$. Therefore, the Galois group $G$ is a
transitive subgroup of $\Sym{\Omega}$.

We first outline the main idea. For all primes $p$ dividing $\# G$ if
we can test the existence of a tower of $G$-blocks $\{\alpha\} =
\Delta_0 \subseteq \ldots \subset \Delta_m$ satisfying the conditions
of Theorem~\ref{thm-g-is-nilpotent} then $G$ is nilpotent. It is not
clear whether we can test these conditions by explicitly computing the
$G$-blocks. That seems possible only if we can already compute the
Galois group $G$. Instead, our approach will be to test the conditions
of Theorem~\ref{thm-g-is-nilpotent} by considering the fixed field of
each group $G_{\Delta_i}$. For a $G$-block $\Delta$, let $\Q_\Delta$
denote the fixed field of the splitting field $\Q_f$ under the
automorphisms in $G_\Delta$. More precisely,
\[
\Q_\Delta=\{\beta\in\Q_f\mid \beta^g=\beta \textrm{ for all } g \in
G_\Delta\}.
\]

The following proposition is a consequence of the fundamental theorem
of Galois theory (Theorem~\ref{thm-funda-galois}).

\begin{proposition}\label{prop-Q-delta}
  Let $\Q_\Delta$ denote the fixed field of $\Q_f$ under automorphisms
  in $G_\Delta$. Then:
  \begin{enumerate}
  \item The Galois group $\Gal{\Q_f/\Q_{\Delta}}$ is the group $G_\Delta$.
  \item Let $\alpha$ be any root of the polynomial $f(X)$ such that
    $\alpha \in \Delta$ then $\Q_{\Delta}$ is a subfield of
    $\Q(\alpha)$.
  \item If $\mu_{\Delta}(X)$ is a primitive polynomial for
    $\Q_{\Delta}$ then its Galois group $\Gal{\mu_\Delta}$ is
    $G/G^{\Delta}$.
  \end{enumerate}
\end{proposition}
\begin{proof}
  Part~1 follows directly from the fundamental theorem of Galois
  theory as $\Q_{\Delta}$ of is the fixed field of $\Q_f$ under
  $G_\Delta$. To prove that $\Q_\Delta \subseteq \Q(\alpha)$ notice
  that the Galois groups $\Gal{\Q_f/\Q_{\Delta}}$ and
  $\Gal{\Q_f/\Q(\alpha)}$ are $G_\Delta$ and $G_\alpha$
  respectively. As $\alpha \in \Delta$ the group $G_\alpha$ is a
  subgroup of $G_\Delta$. Hence by the fundamental theorem of Galois
  theory $\Q_\Delta \subseteq \Q(\alpha)$.

For the third part, notice that if $\mu_{\Delta}(X)$ is the primitive
polynomial of $\Q_{\Delta}$ then its splitting field $\Q_{\mu_\Delta}$
is the normal closure of $\Q_\Delta$ in $\Q_f$. Hence, the Galois
group $\Gal{\Q_f/\Q_{\mu_\Delta}}$ is the largest normal subgroup of
$G$ that is contained in $G_{\Delta}$. By Lemma~\ref{lem-gofsigma}
(putting $\Sigma = \Omega$ in the lemma) it follows that the Galois
group $\Gal{\Q_f/\Q_{\mu_\Delta}}$ is precisely
$G^{\Delta}$. Consequently, the Galois group $\Gal{\mu_\Delta} =
\Gal{\Q_{\mu_\Delta}/\Q}$ is $G/G^\Delta$.
\end{proof}

A direct consequence of Proposition~\ref{prop-Q-delta} and
Theorem~\ref{thm-funda-galois} is the following.

\begin{proposition}\label{prop-Q-delta-nilpotent}
  A tower of $G$-blocks $\{ \alpha \} = \Delta_0 \subset \ldots
  \subset \Delta_m$ satisfies the conditions of
  Theorem~\ref{thm-g-is-nilpotent} if and only if the tower of fixed
  field $\Q(\alpha) = \Q_{\Delta_0} \supset \ldots \supset
  \Q_{\Delta_m}$ satisfies the following conditions:
    \begin{enumerate}
    \item The degree of the extension
      $\Q_{\Delta_i}/\Q_{\Delta_{i+1}}$ is $p$.
    \item The extension $\Q_{\Delta_i}/\Q_{\Delta_{i+1}}$ is normal.
    \item For any block $\Delta$ if $\mu_{\Delta}$ denote a primitive
      polynomial of the field $\Q_\Delta$, then the prime $p$ does not
      divide the order of the Galois group $\Gal{\mu_{\Delta_m}}$ of
      the primitive polynomial $\mu_{\Delta_m}(X)$ of $\Q_{\Delta_m}$.
    \end{enumerate}
\end{proposition}

We will first check whether $f(X)$ is solvable by radicals by using
the Landau-Miller test. Clearly, if $f(X)$ is not solvable by radicals
then $G$ is not nilpotent. If $f(X)$ is solvable by radicals then so
is each polynomial $\mu_{\Delta_i}(X)$ for $1 \leq i \leq m$. Hence,
applying the Landau-Miller algorithm \cite{landau85solvability} we can
compute all the prime factors of $\#\Gal{f}$ and $\#
\Gal{\mu_{\Delta_m}}$ (Theorem~\ref{thm-landau-primes}). Thus, if we
can compute in polynomial time the primitive polynomials
$\mu_{\Delta_i}(X)$ of the fields $\Q_{\Delta_i}$ for $1 \leq i \leq
m$ then we will have a polynomial-time algorithm to verify the
conditions of Proposition~\ref{prop-Q-delta-nilpotent}. The following
theorem is due to Landau and Miller~\cite{landau85solvability}
restated in a form suitable for our application. For completeness, we
present a proof in our notation.

\begin{theorem}[Landau-Miller]\label{thm-enlarge-block}
  Let $f(X)\in\Q[X]$ be irreducible, $G=\Gal{f}$ be its Galois group
  and $\Omega$ be the set of roots of $f$ over the algebraic closure
  $\overline{\Q}$. Let $\Delta\subseteq\Omega$ be any $G$-block and
  $\alpha\in \Delta$. There is an algorithm that takes as input a
  polynomial $p_\Delta(X) \in \Q[X]$ such that
  $\Q_\Delta=\Q(p_\Delta(\alpha))$, runs in time polynomial in
  $\size{f}$ and $\size{p_\Delta}$, and for each $G$-block $\Sigma$
  such that $\Delta$ is a maximal block of $\Sigma$, computes a
  polynomial $p_\Sigma(X) \in \Q[X]$ such that
  $\Q_\Sigma=\Q(p_\Sigma(\alpha))$. Furthermore, the size of the
  computed polynomial $\size{p_\Sigma}$ is bounded by a polynomial in
  $\size{f}$ and is independent of $\size{p_\Delta}$.
\end{theorem}

\begin{remark}
  A couple of remarks are in order before we proceed to the
  proof. Notice that the algorithm for computing $p_\Sigma$ takes as
  input polynomials $f$ as well as $p_\Delta$. However, the theorem
  stipulates that the \emph{size} of the output polynomial $p_\Sigma$
  is polynomially bounded in just the $\size{f}$ and \emph{not} on the
  other polynomial $\size{p_\Delta}$. This property of the algorithm
  is crucial because we will recursively apply this algorithm to a
  tower of blocks, where the tower length can be logarithmic in
  $\deg(f)$. So, if $\size{p_\Sigma}$ had been a polynomial in
  $\size{p_\Delta}$ the overall algorithm would have incurred a
  polynomial size growth at every level of the tower making it
  superpolynomial.

  Another point about the algorithm is that the field $\Q(\alpha)$ is
  identified with the quotient $\Q[X]/f(X)$.  Thus, elements of
  $\Q(\alpha)$ are polynomials in $\alpha$ with rational
  coefficients. The algorithm will work with such polynomials
  representing elements of $\Q(\alpha)$.
\end{remark}

\begin{proof}
  Consider the Galois group $G$ as a permutation group over the roots
  $\Omega$. For $\Delta\subseteq\Omega$ let $T_\Delta(X)$ denote the
  polynomial
  \[
  T_\Delta(X) = \prod_{\eta \in \Delta} (X - \eta).
  \]

\begin{claim}\label{claim-prim-poly}
  For the $G$-block containing $\alpha$ if the polynomial
  $T_\Delta(X)$ defined above is $\delta_0 + \ldots + \delta_r X^r$.
  Then field $\Q_\Delta$ is the field $\Q(\delta_0,\ldots,\delta_r)$.
  Here $\delta_i\in\Q(\alpha)$ are polynomials in $\alpha$ with
  coefficients in $\Q$.
\end{claim}
\begin{proof}[of claim]
  Let $K$ be the field $\Q(\delta_0,\ldots,\delta_r)$ and let $H$ be
  the Galois group $\Gal{\Q_f/K}$. For the claim it is sufficient to
  prove that $H = G_\Delta$. Consider any automorphism $\sigma \in
  G_\Delta$. Since $\sigma$ permutes the roots of $\Delta$ among
  themselves $\sigma (T_\Delta(X)) = T_\Delta$. So, $\sigma$ has to
  fix each of the coefficients $\delta_i$ of $T_\Delta$ and hence
  fixes $K$. Conversely, consider any automorphism $\tau$ of $H$ and
  let $\Delta'$ be the block $\Delta^\tau$.  Since $\tau$ fixes $K$ we
  have $\tau (T_\Delta(X)) = T_{\Delta'}$. As $\Delta$ and $\Delta'$
  are $G$-blocks, this can only happen if $\Delta' = \Delta$ for
  otherwise $T_\Delta(X)$ and $T_{\Delta'}(X)$ have no common
  roots. Therefore, $\tau \in G_\Delta$.
\end{proof}

Given the coefficients of the polynomial $T_\Delta$, notice that we
can compute $p_\Delta$ in polynomial time by applying the primitive
element theorem (see Lemma~\ref{lem-primitive-element}). To see this,
observe that $\Q_\Delta$ is a subfield of $\Q_\alpha$. Hence
$[\Q_\Delta:\Q]\leq \deg(f)$. Therefore, the algorithm given by the
primitive element theorem for computing the coefficients of $p_\Delta$
is polynomial-time bounded. Further, since $T_\Delta(X)$ is a factor
of the polynomial $f(X)$, by a well-known result of Mignotte
\cite{Mi74}, each of the $\delta_i$'s have size bounded by a
polynomial in $\size{f}$.

\begin{claim}\label{claim-factor-blocks}
  Let $\Delta$ be a $G$-block containing $\alpha$. The irreducible
  factor of $f$ over $\Q_\Delta$ which has $\alpha$ as root is
  $T_\Delta$. Let $\Sigma$ be any $G$-block such that $\Sigma
  \supseteq \Delta$. If $g$ is an irreducible factor of $f$ over
  $\Q_\Delta$ then $\Sigma$ contains a root of $g$ if and only if it
  contains all the roots of $g$.
\end{claim}
\begin{proof}[of claim]
  Let $g$ be an irreducible factor of $f(X)$ over $\Q_\Delta$.  The
  roots of $g$ form a $G_\Delta$-orbit of $\Omega$. Conversely, for
  any $G_\Delta$-orbit $\Omega'$ the polynomial $T_{\Omega'}(X)$ is an
  irreducible factor of $f(X)$ over $\Q_\Delta$. Hence the irreducible
  factor of $f$ over $\Q_\Delta$ that has $\alpha$ as root is
  $T_\Delta$.

  For a $G$-block $\Sigma$ containing $\Delta$ we have $G_\Sigma \geq
  G_\Delta$. Hence, any orbit of $G_\Delta$ is completely contained
  inside an orbit of $G_\Sigma$. As the roots of any irreducible
  factor $g(X)$ of $f(X)$ over $\Q_\Delta$ form an orbit of
  $G_\Delta$, it is completely contained inside a $G_\Sigma$
  orbit. Hence, if one of the roots of $g$ is in $\Sigma$, the orbit
  $\alpha^{G_\Sigma}$ of $G_\Sigma$, then all roots are in
  $\Sigma$. This proves the claim.
\end{proof}

Let $\Delta$ be a $G$-block containing $\alpha$ and assume that we
have already computed the polynomial $p_\Delta$. Further, assume that
$f$ factors as $g_0\ldots g_r$ over $\Q_\Delta=\Q(p_\Delta(\alpha))$
(which can be computed in polynomial time by Landau's factorisation
algorithm \cite{landau85factoring}). One of these factors say $g_0$ is
$T_\Delta$. Consider any $G$-block $\Sigma$ such that $\Delta$ is a
maximal $G$-subblock of $\Sigma$.  There is a factor $g_i$ such that
$\Sigma$ contains a root, and hence all the roots
(Claim~\ref{claim-factor-blocks}) of $g_i$.  Let $\Sigma_i$ be the
smallest $G$-block containing $\Delta$ and all the roots of $g_i$.  We
give a polynomial-time algorithm to compute $T_{\Sigma_i}$.
Theorem~\ref{thm-enlarge-block} then follows from this algorithm.

\begin{claim}\label{claim-enlarge-block}
  Let $\Delta$ be a $G$-block containing $\alpha$. Given a polynomial
  $p_\Delta$ such that $\Q_\Delta=\Q(p_\Delta(\alpha))$ as a subfield
  of $\Q(\alpha)$ and an irreducible factor $g$ of $f$ over
  $\Q_\Delta$ we can compute in polynomial time $T_\Sigma$ as a
  polynomial in $\Q(\alpha)[Y]$, where $\Sigma$ is the smallest
  $G$-block containing $\Delta$ and the roots of $g$.
\end{claim}
\begin{proof}[of Claim]
  We are given $\Q_\Delta$ as a subfield of $\Q(\alpha)$. The
  coefficients of factors of $f$ over $\Q_\Delta$ are polynomials in
  $\alpha$. Let the factorisation of $f$ over $\Q_\Delta$ be $f=g_0
  \ldots g_r$, where $g_0 = T_\Delta$ and $g=g_1$.  Denote the set of
  roots of $g_i$ by $\Phi_i$, for each $i$. Then $\Phi_i$'s are the
  orbits of $G_\Delta$ and by Claim~\ref{claim-factor-blocks}, the
  polynomial $T_\Sigma$ is precisely the product of $g_i$ such that
  $\Phi_i\subseteq \Sigma$.

  Let $\beta$ denote a root of $g(X)$, and $\sigma \in \Gal{\Q_f/\Q}$
  be an automorphism such that $\sigma$ maps $\alpha$ to
  $\beta$. Notice that $\sigma$ is an isomorphism between the fields
  $\Q(\alpha)$ and $\Q(\beta)$.  Let $\Sigma$ be the smallest
  $G$-block containing $\Delta$ and $\Phi_1$. {From}
  Theorem~\ref{thm-blocks-galois} and the Galois correspondence of
  blocks (Theorem~\ref{thm-blocks-galois}) we know that $G_\Sigma$ is
  generated by $G_\Delta\cup\{\sigma\}$.

  If generators for $G_\Delta$ and the automorphism $\sigma$ are
  known, then the block $\Sigma$ can be computed by transitive closure
  of procedure as in Algorithm~\ref{algo-compute-Sigma}. The
  correctness of this algorithm follows directly from
  Claim~\ref{claim-factor-blocks}.

  \begin{algorithm}
    \caption{Computing $\Sigma$}\label{algo-compute-Sigma}%
    \label{alg-compute-sigma}
    Let $S:= \{\Delta,\Phi_1\}$

    \While{new orbits get added to $S$} {%
      Compute $S':=\{\Phi^\sigma\mid \Phi\in S\}$

      \lIf{$\Phi_j\cap \Phi^\sigma\neq\emptyset$ for some
	$\Phi^\sigma\in S'$} %
	  {include $\Phi_j$ in $S$}%
    }

    Output $\bigcup\{\Phi\mid \Phi\in S\}$%
  \end{algorithm}

  Our goal is to get a polynomial-time algorithm for computing
  $T_\Sigma$ from the above procedure that defines $\Sigma$.  First,
  we compute the extension field $\Q(\alpha,\beta)=\Q(\gamma)$: we do
  this by first factoring $f$ over $\Q(\alpha)$. Let $h$ be an
  irreducible factor of $g$ over $\Q(\alpha)$. Then
  $\Q(\alpha,\beta)=\Q (\alpha)[X]/h(X)$. As
  $[\Q(\alpha,\beta):\Q]\leq n^2$, we can compute a primitive element
  $\gamma$ in polynomial time.\footnote{Note that we need to invoke
    Remark~\ref{rem-prim-elt} for this computation.} Furthermore, in
  polynomial time we can find polynomials $r_1$ and $r_2$ such that
  $\alpha=r_1(\gamma)$ and $\beta=r_2(\gamma)$.

  For all $1 \leq i \leq r$, let $\sigma$ map the polynomials $g_i$ in
  $\Q(\alpha)[X]$ to the polynomials $g_i^\sigma$ in $\Q(\beta)[X]$,
  obtained by symbolically replacing $\alpha$ by $\beta$ in each
  coefficient of $g_i$.  In Algorithm~\ref{alg-compute-sigma}, testing
  if $\Phi_j\cap\Phi_i^\sigma\neq\emptyset$ amounts to finding if
  $gcd(g_j,g_i^\sigma)$ is nontrivial. To make this gcd computation
  possible, we must express $g_j$ and $g_i$ over $\Q(\gamma)$, which
  we do by replacing $\alpha$ by $r_1(\gamma)$ and $\beta$ by
  $r_2(\gamma)$.  We can now give the algorithm for computing
  $T_\Sigma$.

  \begin{algorithm}
    \caption{Computing $T_\Sigma$}%

    \label{alg-enlarge-block}

    Let $S:= \{T_\Delta,g\}$%

    \While{new factors get included in $S$}%
	  { %

	    Compute $S' := \{ h^\sigma\mid h\in
	    S\}$\

	    \For{each factor $g_j$ and $h^\sigma \in S'$}
		{%

		  Express $g_j(X)$ and $h^\sigma(X)$ as polynomials
		  over the field $\Q(\gamma)$.

		  \lIf{$gcd(g_j,h^\sigma)$ is nontrivial}
                      {include $g_j$ in $S$}%

		}%


	  }%

		Output $T_\Sigma:=T_\Delta\cdot\prod_{g_i \in S}
    g_i$
  \end{algorithm}

  It is clear that Algorithm~\ref{alg-enlarge-block} is
  polynomial-time bounded. The preceding discussion and the procedure
  for defining $\Sigma$ imply that the algorithm correctly computes
  $T_\Sigma$. This proves Claim~\ref{claim-enlarge-block}.
\end{proof}

\end{proof}

\subsection*{The Algorithm}

The complete nilpotence test is given in
Algorithm~\ref{algo-nilpotent}. We show that the algorithm is correct
and that its running time is polynomially bounded in its input size in
the rest of the section.

\begin{algorithm}
  \SetKw{Print}{print}
  \caption{Nilpotence test}\label{algo-nilpotent}
  \KwIn{A polynomial $f(X) \in \Q[X]$ of degree $n$}
  \KwOut{\emph{Accept} if $\Gal{f}$ is nilpotent;\emph{Reject} otherwise}

  Verify that $f(X)$ is solvable using the Landau-Miller test.\;

  \lnl{step-compute-ps} Compute the set $P$ of all the prime factors
  of $\# \Gal{f}$\;

  Let $G\leq\Sym{\Omega}$ denote the Galois group of $f$, where
  $\Omega$ is the set of roots of $f$.

  \lnl{step-for-loop}
  \For{\KwSty{every} $p \in P$} {

   \lnl{step-p-div-n}
   \lIf{$p$ does not divide $n$}{ \emph{Reject}}

    Let $p^m$ be the highest power of $p$ dividing $n$.

    $\Q_{\Delta_0} := \Q(X)/f(X)$

    \lnl{step-build-tower}
    \For{$i = 0$ \KwSty{to} $m -1$}
    {

      \lnl{step-all-min-blocks} By Theorem~\ref{thm-enlarge-block}
      compute $\Q_\Gamma$ for all minimal $G$-blocks $\Gamma$
      containing $\Delta_i$.

      \lnl{step-prime-extension} Among the fields $\Q_\Gamma$
        computed above check if there a field $K$ such that
        $\Q_{\Delta_i}/K$ is a normal extension of degree
        $[\Q_{\Delta_i} : K] = p$.

      \lIf{no such field exists}{\emph{Reject}}

      \lElse{ $\Q_{\Delta_{i+1}} := K$ }
    }

    Let $\mu_{\Delta_m}(X)$ be the primitive polynomial for
    $\Q_{\Delta_m}$

    \lnl{step-check-psylow}
    \lIf{$p$ divides $\#\Gal{\mu_{\Delta_m}}$}{\emph{Reject}}
}\emph{Accept}
\end{algorithm}

\begin{proposition}\label{prop-runs-ptime}
  Algorithm~\ref{algo-nilpotent} runs in time polynomial in
  $\size{f}$.
\end{proposition}

\begin{proof}
The Landau-Miller solvability test for $\Gal{\Q_f/\Q}$ and the
algorithm of Theorem~\ref{thm-landau-primes} are polynomial time
bounded \cite{landau85solvability}. The nilpotence test first verifies
that $\Gal{\Q_f/\Q}$ is solvable by applying the Landau-Miller test.
Now, since the field $\Q_{\Delta_m}$ is contained in $\Q_f$, its
primitive polynomial $\mu_{\Delta_m}(X)$ will also split in $\Q_f$
implying that $\Q_{\mu_{\Delta_m}} \subset \Q_f$. Hence
$\Gal{\Q_{\mu_{\Delta_m}}/\Q}=\Gal{\mu_{\Delta_m}}$ is also a solvable
group. Hence, in steps~\ref{step-compute-ps} and
\ref{step-check-psylow} we can apply the algorithm of
Theorem~\ref{thm-landau-primes} to compute all the prime factors of
$\# \Gal{f}$ and $\# \Gal{\mu_{\Delta_m}}$ in polynomial time.

Step~\ref{step-all-min-blocks} can be carried out in polynomial time
by Theorem~\ref{thm-enlarge-block}, and
Step~\ref{step-prime-extension} can be carried out in polynomial time
(applying the algorithmic version of the primitive element theorem as
explained in Remark~\ref{rem-prim-elt}).

Clearly all other steps within the loop starting at
line~\ref{step-build-tower} can be carried out in polynomial
time. Hence the overall algorithm is polynomial-time bounded.
\end{proof}

We now prove the correctness of the algorithm in the next two
propositions.

\begin{proposition}\label{prop-nilpotent-accepts}
  If $f(X)$ is an input irreducible polynomial of degree $n$ such that
  $\Gal{f}$ is nilpotent then Algorithm~\ref{algo-nilpotent} accepts
  $f$.
\end{proposition}

\begin{proof}
  Let $G$ be the Galois group $\Gal{f}$ and let $\Omega$ be the set of
  roots of $f$.  Since $f$ is of degree $n$, $\# \Omega = n$ and by
  Lemma~\ref{lem-sylow-cardinality} every prime factor of $\# G$
  divides $n$.  Therefore, algorithm never rejects $f$ at
  step~\ref{step-p-div-n}. Now, for the loop starting in
  line~\ref{step-build-tower} we show that if $G$ is nilpotent the
  algorithm always succeeds in finding a field $K$, from among the
  candidate fields $\Q_\Gamma$, in
  step~\ref{step-prime-extension}. Notice that at the $i^{th}$
  iteration the block $\Delta_i$ is of cardinality $p^i$ and $i<
  m$. Hence, by Theorem~\ref{thm-if-g-nilpotent}, $\Delta_i$ is
  contained in some $p$-Sylow block say $\Sigma$ and there is a
  minimal $G$-block $\Delta$ containing $\Delta_i$ that has the
  following three properties:
  \begin{enumerate}
  \item The index $[\Delta:\Delta_i]$ is  $p$.
  \item The group $G_{\Delta_i}$ is a normal subgroup of $G_\Delta$.
  \item The quotient $G_{\Delta}/G_{\Delta_i}$ is cyclic of order $p$.
  \end{enumerate}

  Consider the field $\Q_\Delta$. Since $\Delta$ is a minimal block
  that properly contains $\Delta_i$, the field $\Q_\Delta$ is among
  the fields computed in Step~\ref{step-all-min-blocks}. We claim that
  $\Q_\Delta$ is a suitable choice for $K$ in
  step~\ref{step-prime-extension}. The groups $G_{\Delta_i}$ and
  $G_\Delta$ are the Galois groups $\Gal{\Q_f/\Q_{\Delta_i}}$ and
  $\Gal{\Q_f/\Q_\Delta}$, respectively. Hence, by
  Theorem~\ref{thm-funda-galois}, we have $\Q_{\Delta_i}/\Q_\Delta$ is
  a normal extension of degree $[\Q_{\Delta_i}:\Q_\Delta] =
  [G_\Delta:G_{\Delta_i}] = p$. As the algorithm goes over all minimal
  $G$-blocks $\Gamma$ containing $\Delta_{i}$, it will always succeed
  in finding a field $K$ in step~\ref{step-prime-extension}.

  Finally, at the end of the loop, the index $i$ becomes $m$ and
  $\Delta_m$ is of order $p^m$. Since $p^m$ is the highest power of
  $p$ dividing $\# \Omega$, by Theorem~\ref{thm-if-g-nilpotent} the
  block $\Delta_m$ is a $p$-Sylow block. By
  Proposition~\ref{prop-Q-delta}
  $G/G^{\Delta_m}=\Gal{\mu_{\Delta_m}}$.  Further, by
  Lemma~\ref{lem-gsup-sylow} the group $G^{\Delta_m}$ is the unique
  $p$-Sylow subgroup of the nilpotent group $G$, which implies that
  $p$ does not divide $\# G/G^{\Delta_m}$. Hence the input passes the
  test in step~\ref{step-check-psylow}.
\end{proof}

\begin{proposition}\label{prop-accepts-nilpotent}
  If Algorithm~\ref{algo-nilpotent} accepts the input polynomial
  $f(X)$ then $\Gal{f}$ is nilpotent.
\end{proposition}
\begin{proof}
  Let $G$ be the Galois group of $f$. We claim that if the algorithm
  accepts the input then for every prime $p$ dividing $\# G$ we have a
  maximal chain of $\{\alpha\} = \Delta_0 \subset \ldots \subset
  \Delta_m$ with the following properties
  \begin{enumerate}
  \item The index of the block $[\Delta_{i+1}: \Delta_i] = p$,
  \item The group $G_{\Delta_i}$ is a normal subgroup of
    $G_{\Delta_{i+1}}$ and
  \item The prime $p$ does not divide $G/G^{\Delta_m}$.
  \end{enumerate}

  This is because in step~\ref{step-prime-extension} we have verified
  that $\Q_{\Delta_i}$ is a normal extension of $\Q_{\Delta_{i+1}}$ of
  degree $p$. Hence by the fundamental theorem of Galois theory their
  Galois groups $G_{\Delta_i}$ and $G_{\Delta_{i+1}}$ are such that
  $G_{\Delta_i}$ is a normal subgroup of $G_{\Delta_{i+1}}$ and
  $[G_{\Delta_{i+1}}:G_{\Delta_i}] = [\Delta_{i+1}: \Delta_i] =
  p$. Furthermore, in step~\ref{step-check-psylow} we have verified
  that $p$ does not divide the order of $\Gal{\mu_{\Delta_m}} =
  G/G^{\Delta_m}$. Therefore, $G$ satisfies all the properties of
  Theorem~\ref{thm-g-is-nilpotent} and hence is nilpotent.
\end{proof}

Propositions~\ref{prop-runs-ptime}, \ref{prop-nilpotent-accepts} and
\ref{prop-accepts-nilpotent} together show the following.

\begin{theorem}\label{thm-main}
  There is a deterministic polynomial-time algorithm that takes as
  input $f(X)$ over $\Q$ and decides whether the Galois group of $f$
  is nilpotent.
% in time bounded by a polynomial in $\size{f}$.
\end{theorem}

\section{Computing Sylow polynomials}

In the last two sections we saw that Sylow subgroups play a crucial
role in the nilpotence testing algorithm. In this section we explore
whether any further information regarding Sylow subgroups of nilpotent
Galois groups can be computed. In this context we make the following
definition.

\begin{definition}[Sylow polynomials]
  Let $f(X)$ be any polynomial over $\Q$ with nilpotent Galois group
  $G$. Let $p$ be a prime that divides the order of $G$. By a
  $p$-Sylow polynomial we mean a polynomial $g(X)$ over $\Q$ such that
  $g(X)$ splits in the splitting field $\Q_f$ of $f$ and the $\Gal{g}$
  is (isomorphic to) the $p$-Sylow subgroup $G_p$ of $G$.
\end{definition}

In this section we describe a polynomial-time algorithm that, given as
input a polynomial $f(X)\in\Q[X]$ with nilpotent Galois group $G$,
computes a $p$-Sylow polynomial for each prime factor $p$ of $\# G$.
An immediate consequence is that for polynomials $f(X)$ with nilpotent
Galois group there are Sylow polynomials of polynomially bounded
degree.

In the following lemma we show that for this problem it suffices to
consider irreducible polynomials $f(X)$.

\begin{lemma}\label{lem-sylow-product}
  Let $f(X)\in\Q[X]$ be a polynomial with nilpotent Galois group and
  let $f_1,f_2,\ldots,f_k$ be all distinct irreducible factors of
  $f(X)$. For each $i$, let $g_i$ be a $p$-Sylow polynomial for $f_i$.
  Then the product polynomial $g_1g_2\cdots g_k$ is a $p$-Sylow
  polynomial for $f(X)$.
\end{lemma}

\begin{proof}
  Let $g(X)=g_1(X)g_2(X)\cdots g_k(X)$. Since $g_i$ is a $p$-Sylow
  polynomial of $f_i$, by definition we $g_i$ splits in the field
  $\Q_{f_i}$, and its Galois group $\Gal{g_i}$ is isomorphic to the
  unique $p$-Sylow subgroup $G^{(i)}_p$ of $G^{(i)}=\Gal{f_i}$. Hence
  the Galois group $\Gal{\Q_{f_i}/\Q_{g_i}}$ is isomorphic to the
  quotient group $G^{(i)}/G^{(i)}_p$. Therefore, $p$ does not divide
  the order of the Galois group $\Gal{\Q_{f_i}/\Q_{g_i}}$.

  For each $i$ we have $\Q_{g_i}\subseteq \Q_{f_i}\subseteq \Q_f$. Let
  $\overline{\Q}$ denote the algebraic closure of $\Q$. Observe that
  $\Q_f$ is the smallest subfield of $\overline{\Q}$ containing
  $\Q_{f_i}$ for each $f_i$. Likewise, $\Q_g$ is the smallest subfield
  of $\overline{Q}$ containing each $\Q_{g_i}$. It follows that $\Q_g$
  is a subfield of $\Q_f$.

  Furthermore, we can observe that every $\sigma\in\Gal{\Q_g/\Q}$ must
  map $g_i$ to $g_i$ for each $i$. Hence, for each $i$, $\sigma$
  restricted to $\Q_{g_i}$ is in $\Gal{\Q_{g_i}/\Q}$. Clearly,
  $\sigma$ is nontrivial if and only if it is nontrivial on some
  $\Q_{g_i}$.  Therefore, since each $\Gal{\Q_{g_i}/\Q}$ is a
  $p$-group it follows that $\Gal{\Q_g/\Q}$ is also a
  $p$-group.

  Similarly, consider an element $\pi\in\Gal{\Q_f/\Q_g}$. We can see
  that for each $i$, $\pi$ restricted to $\Q_{f_i}$ is in
  $\Gal{\Q_{f_i}/\Q_{g_i}}$ and $\pi$ is nontrivial only if it is
  nontrivial on some $\Q_{f_i}$.

  As a result the Galois group $\Gal{\Q_f/\Q_g}$ is isomorphic to a
  subgroup of the product group $\prod_i
  \Gal{\Q_{f_i}/\Q_{g_i}}$. Hence, $p$ is not a factor of $\#
  \Gal{\Q_f/\Q_g}$ as it is not a factor of $\#
  \Gal{\Q_{f_i}/\Q_{g_i}}$ for $1 \leq i \leq k$. It follows that
  $\Gal{\Q_g/\Q}$ is isomorphic to the $p$-Sylow subgroup of
  $\Gal{\Q_f/\Q}$. Hence $g$ is a $p$-Sylow polynomial for $f$.
\end{proof}

We now consider the problem of computing a $p$-Sylow polynomial for an
irreducible polynomial $f(X)$ with nilpotent Galois group. To this end
we generalise the notion of Sylow blocks
(Definition~\ref{def-sylow-block}).

\begin{definition}[Generalised Sylow block]
  Let $G$ be a transitive nilpotent permutation group on $\Omega$ and
  let $\{p_1,p_2,\ldots,p_k\}$ be the set of all prime factors of $\#
  G$.  For $\emptyset\subset P\subseteq \{p_1,p_2,\ldots,p_k\}$, a
  subset $\Sigma\subseteq \Omega$ is called a \emph{$P$-Sylow block}
  if it is an orbit of $G_P = \prod_{p \in P}G_p$.
\end{definition}

Since $G_P$ is a normal subgroup of $G$, each orbit of $G_P$ is a
block of $G$.

The next three results state some properties of $P$-Sylow blocks of
transitive nilpotent groups that we require for the algorithm. These
are generalisations of Lemma~\ref{lem-sylow-cardinality},
Lemma~\ref{lem-gsup-sylow} and Theorem~\ref{thm-if-g-nilpotent}
respectively. We give brief proofs for these since they are on the
same lines as their counterparts in
Section~\ref{sec-nilpotence-perm-group}. As a matter of fact, all
results in Section~\ref{sec-nilpotent-perm-group} for Sylow blocks
have straightforward generalisations to $P$-Sylow blocks.

% Lemma~\ref{lem-gen-sylow-subblocks},
%Lemma~\ref{lem-gen-gsup-sylow} and
% Theorem~\ref{thm-gen-if-g-nilpotent} which

\begin{lemma}\label{lem-gen-sylow-subblocks}
  Let $G\leq\Sym{\Omega}$ be a transitive nilpotent permutation group
  and let $P$ be any set of primes that divide the order of $G$. Let
  $m_p$ denote the highest power of $p$ that divides $\# \Omega$. Then
  any $P$-Sylow block $\Sigma$ has cardinality $\prod_{p\in P}
  p^{m_p}$.
\end{lemma}

\begin{proof}
  By Theorem~\ref{thm-blocks-galois}
  $[\Omega~:~\Sigma]=[G~:~G_\Sigma]$. Recall that $G_\Sigma=\{g\in
  G\mid \Sigma^g=\Sigma\}$. Since $\Sigma$ is a $G_P$ orbit it follows
  that $G_P$ is a subgroup of $G_\Sigma$. Hence for every prime $p\in
  P$, $p$ does not divide $\# G/\# G_\Sigma$, and hence $p$ does not
  divide $\# \Omega/\# \Sigma$ for each $p\in P$.  By the nilpotence
  of $G$, each $p\in P$ divides $\# \Omega$. Therefore, $p^{m_p}$
  divides $\# \Omega$ for each $p\in P$.

  On the other hand, since $\Sigma$ is a $G_P$-orbit, $G_P$ is
  transitive on $\Sigma$. Hence, by the Orbit-Stabilizer formula
  (Theorem~\ref{thm-orbit-stabilizer}) $\# \Sigma$ divides $\# G_P$
  which means $\# \Sigma$ has no prime factors other than from $P$.
  Putting it together, it follows that $\Sigma$ has cardinality
  $\prod_{p\in P} p^{m_p}$.
\end{proof}

\begin{lemma}\label{lem-gen-gsup-sylow}
  Let $G\leq\Sym{\Omega}$ be a transitive permutation group and let
  $P$ be any set of primes that divide the order of $G$. Let $\Sigma$
  be any $P$-Sylow block of $G$ then $G^\Sigma$ is the product
  $\prod_{p \in P} G_p$.
\end{lemma}

\begin{proof}
  As observed in the previous lemma, $G_P$ is a subgroup of
  $G_\Sigma$.  Since $\Sigma$ is a block for $G$, consider the block
  system generated by $G$-action on $\Sigma$ and let $\Sigma'$ be any
  other block in this system. For some $g\in G$ we have
\[
G_{\Sigma'} ~=~gG_\Sigma g^{-1}.
\]
Since $G_P$ is a normal subgroup of $G$, $gG_Pg^{-1}=G_p$, and hence
$G_P$ is a subgroup of $G_{\Sigma'}$ for each $\Sigma'$ in the block
system which implies $G_P$ is a subgroup of $G^\Sigma$.

Suppose that $G_P\neq G^\Sigma$. Then some prime $p\not\in P$ divides
$\# G^\Sigma$. Let $G^\Sigma_p$ be the $p$-Sylow subgroup of
$G^\Sigma$ (which is unique and hence a normal subgroup since
$G^\Sigma$ is also nilpotent). Suppose $G^\Sigma_p$ has nontrivial
action on some block $\Sigma'$ in the block system. If $O\subset
\Sigma'$ is a nontrivial $G^\Sigma_p$-orbit then it is a
$G^\Sigma$-block contained in $\Sigma'$. Further, $G^\Sigma$ is
transitive on $\Sigma'$ since $G_P\leq G^\Sigma$. Hence $|O|$ divides
$|\Sigma'|$ which is impossible since $|O|$ is a power of $p$ and
$|\Sigma'|$ does not have $p$ as factor. This is a
contradiction. Hence $G_P= G^\Sigma$.
\end{proof}

\begin{theorem}\label{thm-gen-if-g-nilpotent}
  Let $G$ be any transitive permutation group on $\Omega$ and let $P$
  be a set of primes that divide the order of $G$. Let $\Gamma$ be any
  $G$-block such that each prime that divides $\# \Gamma$ is in
  $P$. Then there is a $P$-Sylow block $\Sigma$ of $G$ such that
  $\Gamma \subset \Sigma$. Furthermore, for a prime $p \in P$ if $p$
  divides $\frac{\# \Omega}{\# \Gamma}$ then we have a $G$-block
  $\Gamma'$ such that $\Gamma\subset \Gamma'\subseteq \Sigma$ such that
  \begin{enumerate}
  \item The index $[\Gamma': \Gamma]$ is $p$.
  \item The group $G_\Gamma$ is a normal subgroup of $G_{\Gamma'}$.
  \item The quotient group $G_{\Gamma'}/G_{\Gamma}$ is a cyclic group
    of order $p$.
  \end{enumerate}
\end{theorem}

\begin{proof}
  Let $\Gamma$ be any $G$-block and $\alpha\in\Gamma$. Let
  $\Sigma=\alpha^{G_P}$ which, by definition, is a $P$-Sylow block.
  We claim that $\Gamma\subseteq \Sigma$. Let $\{p_1,p_2,\ldots,p_k\}$
  be the set of all distinct prime factors of $\# G$ and let
  $\overline{P}$ denote the complement of $P$ in this set.

  Now consider the group $G_\Gamma$, which is transitive on $\Gamma$
  by Proposition~\ref{prop-block-subblock}. Since $G_\Gamma$, being
  nilpotent, is the direct product of its Sylow subgroups, we can
  write
\[
G_\Gamma~=~G_{\Gamma,P}\times G_{\Gamma,\overline{P}},
\]
where $G_{\Gamma,P}$ is the product of the $p$-Sylow subgroups of
$G_\Gamma$ for $p\in P$ and $G_{\Gamma,\overline{P}}$ is similarly
defined for $\overline{P}$. Now, every element of
$G_{\Gamma,\overline{P}}$ must pointwise fix $\Gamma$ because
$G_{\Gamma,\overline{P}}$ is a normal subgroup of $G_\Gamma$ and a
nontrivial orbit of $G_{\Gamma,\overline{P}}$ will have size whose
prime factors are all from $\overline{P}$ on the one hand, and on the
other hand the orbit size must divide $\# \Gamma$. It follows that
$G_{\Gamma,P}$ is transitive on $\Gamma$ and hence
$\alpha^{G_{\Gamma,P}}=\Gamma$. Since $G_{\Gamma,P}$ is contained in
$G_P$ it follows that $\Gamma\subseteq \Sigma$.

Suppose $p\in P$ divides $\# \Omega/\# \Gamma$. To prove the three
parts of the theorem, we will consider the action of the group $G$ on
the block system $\Blocks{\Omega/\Gamma}$. Since $G$ is nilpotent and
transitive on $\Blocks{\Omega/\Gamma}$ and $p$ divides $\# \Omega/\#
\Gamma$ we can apply Theorem~\ref{thm-if-g-nilpotent}. The block
$\Delta$ in Theorem~\ref{thm-if-g-nilpotent} is set to be the
singleton set $\{\Gamma\}$. By Theorem~\ref{thm-if-g-nilpotent} there
is a $p$-Sylow block
$\Sigma'=\{\Gamma_1=\Gamma,\Gamma_2,\ldots,\Gamma_t\}$ containing
$\Gamma$ and a minimal $G$-block $\Delta'\subset \Sigma'$ of size $p$.
Without loss of generality, let
\[
\Delta'=\{\Gamma_1=\Gamma,\Gamma_2,\ldots,\Gamma_p\}.
\]
Let $\Gamma'=\bigcup_{i=1}^p\Gamma_i$. It is easy to verify that all
the three conditions in the statement follow from the corresponding
conditions for $\Delta'$ and $\Delta$ in
Theorem~\ref{thm-if-g-nilpotent}.
\end{proof}

Let $\{p_1,p_2,\ldots,p_k\}$ be the set of all distinct prime factors
of $\# \Gal{f}$. In order to compute the $p_i$-Sylow polynomial we set
$P_i=\{p_1,p_2,\ldots,p_k\}\setminus\{p_i\}$ and compute a tower of
blocks $\{\alpha\}=\Delta_0\subset \Delta_1\subset\ldots\subset
\Delta_m$ where $\Delta_m$ is the $P_i$-Sylow block containing the
point $\alpha$. By computing the blocks $\Delta_i$ we mean, as in
Section~\ref{sec-nilpotence-test}, that we compute a primitive
polynomial $\mu_{\Delta_i}(X)$ for the field $\Q_{\Delta_i}$, $1\leq
i\leq m$. Since the Galois group of $\mu_{\Delta_i}$ is
$G/G^{\Delta_i}$ (by Proposition~\ref{prop-Q-delta}), and
$G^{\Delta_m}=G_{P_i}$ it follows that the Galois group of
$\mu_{\Delta_m}$ is $G/G^{\Delta_m}=G/G_{P_i}$ which is isomorphic to
$G_{p_i}$, the $p_i$-Sylow subgroup of $G$. Hence $\mu_{\Delta_m}$ is
a $p_i$-Sylow polynomial for $f$.

We now give the complete algorithm for computing the $p$-Sylow
polynomial for an irreducible polynomial $f(X)$ with nilpotent Galois
group.

\begin{algorithm}
  \caption{Computing a $p_i$-Sylow polynomial}\label{algo-sylow-poly}

  \KwIn{An irreducible polynomial $f(X) \in \Q[X]$ of degree $n$ with
    nilpotent Galois group and a prime factor $p_i$ of $n$.}

  \KwOut{A $p_i$-Sylow polynomial of $f$}

  Let $\{p_1,p_2,\ldots,p_k\}$ be the set of all prime factors of $\#
  \Gal{f}$ (these are exactly the prime factors of $n$ by
  Lemma~\ref{lem-sylow-cardinality}).

  Let $\Q_{\Delta_0} = \Q(X)/f(X)$ and $\mu_{\Delta_0}(X) = f(X)$.

  Let $G$ denote the Galois group of $f$.

  $r:=0$

  $\Q_{\Delta_0} := \Q(X)/f(X)$

  \lnl{step-one}

  \For{$j \in \{1,\ldots, i -1, i+1, \ldots,k\}$} {

    Let $p_j^{m_j}$ be the highest power of $p_j$ dividing $n$.

    \lnl{step-two}
    \For{$\ell = 1$ \KwSty{to} $m_j$}
    {

      \lnl{step-three} Using Theorem~\ref{thm-enlarge-block}
      compute $\Q_\Gamma$ for all minimal $G$-blocks $\Gamma$
      containing $\Delta_r$.

      \lnl{step-four} Among the fields $\Q_\Gamma$ computed
      above find a field $\Q_{\Delta_{r+1}}$ such that
      $\Q_{\Delta_r}/\Q_{\Delta_{r+1}}$ is a normal extension of
      degree $p_j$.

      $r:=r+1$
    }
 }
    Let $\mu_{\Delta_m}(X)$ be the primitive polynomial for
    $\Q_{\Delta_m}$, where $m=\sum_{j\neq i} m_j$

  \KwRet{$\mu_{\Delta_m}(X)$}
\end{algorithm}

Clearly Algorithm~\ref{algo-sylow-poly} runs in time polynomial in
$\size{f}$. By Theorem~\ref{thm-gen-if-g-nilpotent} it follows that
step~\ref{step-four} is always possible. Therefore, it follows from
Lemma~\ref{lem-gen-sylow-subblocks} that at the end of the two loops
we have a primitive polynomial $\mu_{\Delta_m}(X)$ for $\Q_{\Delta_m}$
where $\Delta_m$ is a $P_i$-Sylow block of $G$.  The Galois group
$\Gal{\mu_{\Delta_m}}$ is $G/G^{\Delta_m}$ by
Proposition~\ref{prop-Q-delta}. Hence the Galois group of
$\mu_{\Delta_m}(X)$ is the $p_i$-Sylow subgroup of $G$ as claimed.
Algorithm~\ref{algo-sylow-poly} thus computes the $p_i$-Sylow
polynomial for the irreducible polynomial $f(X)$. Together with
Lemma~\ref{lem-sylow-product} we have the following theorem.

\begin{theorem}\label{thm-sylow-poly}
  There is a deterministic polynomial-time algorithm that given a
  polynomial $f(X)$ with nilpotent Galois group and any prime $p$
  dividing the order of the Galois group $\Gal{f}$, computes a
  polynomial $g(X)$ such that $g(X)$ splits in $\Q_f$ and the Galois
  group of $g(X)$ is isomorphic to the $p$-Sylow subgroup of
  $\Gal{f}$.
\end{theorem}

In fact the same ideas yield a more general observation by
modifying the Algorithm~\ref{algo-sylow-poly} to work with
any arbitrary set $P$ of prime factors of $\# \Gal{f}$.

\begin{theorem}
  There is a deterministic polynomial-time algorithm that given a
  polynomial $f(X)$ with nilpotent Galois group and any subset $P$ of
  primes dividing the order of the Galois group $\Gal{f}$, computes a
  polynomial $g(X)$ such that $g(X)$ splits in $\Q_f$ and the Galois
  group of $g(X)$ is (isomorphic to) the subgroup $G_P = \prod_{p \in
    P} G_p$ where $G_p$ denotes the unique $p$-Sylow subgroup of
  $\Gal{f}$.
\end{theorem}

\section{Concluding Remarks}

Computing the Galois group of a polynomial $f(X)\in\Q[X]$ efficiently
remains a challenging open problem. However, it is possible to test
certain properties like commutativity and solvability efficiently. We
have added nilpotence testing to this list. In this context, an
intriguing problem is whether we can efficiently test if the Galois
group of $f(X)$ is supersolvable (refer Hall's text~\cite[Chapter
10]{Hall} for a definition).  Supersolvable groups are a proper
subclass of solvable groups and contain nilpotent groups.  It is not
clear if we can adapt either the Landau-Miller solvability
test~\cite{landau85solvability} or our nilpotence test to an efficient
supersovablility test. It would be interesting to even give a
conditionally efficient algorithm, e.g.\ assuming the generalised
Riemann hypothesis.

Finally, we note that our nilpotent test can be generalised to obtain
a polynomial-time algorithm to test if the Galois group of a
polynomial $f(X) \in K[X]$ is nilpotent, where $K$ is a number field
presented by giving a primitive polynomial $\mu(X)$ of $K$ over $\Q$,
and the running time is polynomially bounded in $\size{f}$ and
$\size{\mu}$. This generalised nilpotence test requires some standard
polynomial-time algorithms like factoring of univariate polynomials
and gcd computations over $K$ which are already
known~\cite{landau85factoring}.\\

\noindent{\bf Acknowledgments.~}
We are indebted to the referees for their meticulous refereeing with
many important suggestions and corrections.

\bibliographystyle{acmsmall}
\bibliography{reference}
\end{document}
